\documentclass{beamer}
\usepackage{amsmath,graphics}
\usepackage{amssymb}

\usetheme{default}
\usepackage{xcolor}

\definecolor{solarizedBase03}{HTML}{002B36}
\definecolor{solarizedBase02}{HTML}{073642}
\definecolor{solarizedBase01}{HTML}{586e75}
\definecolor{solarizedBase00}{HTML}{657b83}
\definecolor{solarizedBase0}{HTML}{839496}
\definecolor{solarizedBase1}{HTML}{93a1a1}
\definecolor{solarizedBase2}{HTML}{EEE8D5}
\definecolor{solarizedBase3}{HTML}{FDF6E3}
\definecolor{solarizedYellow}{HTML}{B58900}
\definecolor{solarizedOrange}{HTML}{CB4B16}
\definecolor{solarizedRed}{HTML}{DC322F}
\definecolor{solarizedMagenta}{HTML}{D33682}
\definecolor{solarizedViolet}{HTML}{6C71C4}
%\definecolor{solarizedBlue}{HTML}{268BD2}
\definecolor{solarizedBlue}{HTML}{134676}
\definecolor{solarizedCyan}{HTML}{2AA198}
\definecolor{solarizedGreen}{HTML}{859900}
\definecolor{myBlue}{HTML}{162DB0}%{261CA4}
\setbeamercolor*{item}{fg=myBlue}
\setbeamercolor{normal text}{fg=solarizedBase03, bg=solarizedBase3}
\setbeamercolor{alerted text}{fg=myBlue}
\setbeamercolor{example text}{fg=myBlue, bg=solarizedBase3}
\setbeamercolor*{frametitle}{fg=solarizedRed}
\setbeamercolor*{title}{fg=solarizedRed}
\setbeamercolor{block title}{fg=myBlue, bg=solarizedBase3}
\setbeameroption{hide notes}
\setbeamertemplate{note page}[plain]
\beamertemplatenavigationsymbolsempty
\usefonttheme{professionalfonts}
\usefonttheme{serif}

\usepackage{fourier}

\def\vec#1{\mathchoice{\mbox{\boldmath$\displaystyle#1$}}
{\mbox{\boldmath$\textstyle#1$}}
{\mbox{\boldmath$\scriptstyle#1$}}
{\mbox{\boldmath$\scriptscriptstyle#1$}}}

\definecolor{OwnGrey}{rgb}{0.560,0.000,0.000} % #999999
\definecolor{OwnBlue}{rgb}{0.121,0.398,0.711} % #1f64b0
\definecolor{red4}{rgb}{0.5,0,0}
\definecolor{blue4}{rgb}{0,0,0.5}
\definecolor{Blue}{rgb}{0,0,0.66}
\definecolor{LightBlue}{rgb}{0.9,0.9,1}
\definecolor{Green}{rgb}{0,0.5,0}
\definecolor{LightGreen}{rgb}{0.9,1,0.9}
\definecolor{Red}{rgb}{0.9,0,0}
\definecolor{LightRed}{rgb}{1,0.9,0.9}
\definecolor{White}{gray}{1}
\definecolor{Black}{gray}{0}
\definecolor{LightGray}{gray}{0.8}
\definecolor{Orange}{rgb}{0.1,0.2,1}
\setbeamerfont{sidebar right}{size=\scriptsize}
\setbeamercolor{sidebar right}{fg=Black}

\renewcommand{\emph}[1]{{\textcolor{solarizedRed}{\itshape #1}}}

\newcommand\cA{\mathcal A}
\newcommand\cB{\mathcal B}
\newcommand\cC{\mathcal C}
\newcommand\cD{\mathcal D}
\newcommand\cE{\mathcal E}
\newcommand\cF{\mathcal F}
\newcommand\cG{\mathcal G}
\newcommand\cH{\mathcal H}
\newcommand\cI{\mathcal I}
\newcommand\cJ{\mathcal J}
\newcommand\cK{\mathcal K}
\newcommand\cL{\mathcal L}
\newcommand\cM{\mathcal M}
\newcommand\cN{\mathcal N}
\newcommand\cO{\mathcal O}
\newcommand\cP{\mathcal P}
\newcommand\cQ{\mathcal Q}
\newcommand\cR{\mathcal R}
\newcommand\cS{\mathcal S}
\newcommand\cT{\mathcal T}
\newcommand\cU{\mathcal U}
\newcommand\cV{\mathcal V}
\newcommand\cW{\mathcal W}
\newcommand\cX{\mathcal X}
\newcommand\cY{\mathcal Y}
\newcommand\cZ{\mathcal Z}

\newcommand\fA{\mathfrak A}
\newcommand\fB{\mathfrak B}
\newcommand\fC{\mathfrak C}
\newcommand\fD{\mathfrak D}
\newcommand\fE{\mathfrak E}
\newcommand\fF{\mathfrak F}
\newcommand\fG{\mathfrak G}
\newcommand\fH{\mathfrak H}
\newcommand\fI{\mathfrak I}
\newcommand\fJ{\mathfrak J}
\newcommand\fK{\mathfrak K}
\newcommand\fL{\mathfrak L}
\newcommand\fM{\mathfrak M}
\newcommand\fN{\mathfrak N}
\newcommand\fO{\mathfrak O}
\newcommand\fP{\mathfrak P}
\newcommand\fQ{\mathfrak Q}
\newcommand\fR{\mathfrak R}
\newcommand\fS{\mathfrak S}
\newcommand\fT{\mathfrak T}
\newcommand\fU{\mathfrak U}
\newcommand\fV{\mathfrak V}
\newcommand\fW{\mathfrak W}
\newcommand\fX{\mathfrak X}
\newcommand\fY{\mathfrak Y}
\newcommand\fZ{\mathfrak Z}

\newcommand\fa{\mathfrak a}
\newcommand\fb{\mathfrak b}
\newcommand\fc{\mathfrak c}
\newcommand\fd{\mathfrak d}
\newcommand\fe{\mathfrak e}
\newcommand\ff{\mathfrak f}
\newcommand\fg{\mathfrak g}
\newcommand\fh{\mathfrak h}
%\newcommand\fi{\mathfrak i}
\newcommand\fj{\mathfrak j}
\newcommand\fk{\mathfrak k}
\newcommand\fl{\mathfrak l}
\newcommand\fm{\mathfrak m}
\newcommand\fn{\mathfrak n}
\newcommand\fo{\mathfrak o}
\newcommand\fp{\mathfrak p}
\newcommand\fq{\mathfrak q}
\newcommand\fr{\mathfrak r}
\newcommand\fs{\mathfrak s}
\newcommand\ft{\mathfrak t}
\newcommand\fu{\mathfrak u}
\newcommand\fv{\mathfrak v}
\newcommand\fw{\mathfrak w}
\newcommand\fx{\mathfrak x}
\newcommand\fy{\mathfrak y}
\newcommand\fz{\mathfrak z}

\newcommand\vA{\vec A}
\newcommand\vB{\vec B}
\newcommand\vC{\vec C}
\newcommand\vD{\vec D}
\newcommand\vE{\vec E}
\newcommand\vF{\vec F}
\newcommand\vG{\vec G}
\newcommand\vH{\vec H}
\newcommand\vI{\vec I}
\newcommand\vJ{\vec J}
\newcommand\vK{\vec K}
\newcommand\vL{\vec L}
\newcommand\vM{\vec M}
\newcommand\vN{\vec N}
\newcommand\vO{\vec O}
\newcommand\vP{\vec P}
\newcommand\vQ{\vec Q}
\newcommand\vR{\vec R}
\newcommand\vS{\vec S}
\newcommand\vT{\vec T}
\newcommand\vU{\vec U}
\newcommand\vV{\vec V}
\newcommand\vW{\vec W}
\newcommand\vX{\vec X}
\newcommand\vY{\vec Y}
\newcommand\vZ{\vec Z}

\newcommand\va{\vec a}
\newcommand\vb{\vec b}
\newcommand\vc{\vec c}
\newcommand\vd{\vec d}
\newcommand\ve{\vec e}
\newcommand\vf{\vec f}
\newcommand\vg{\vec g}
\newcommand\vh{\vec h}
\newcommand\vi{\vec i}
\newcommand\vj{\vec j}
\newcommand\vk{\vec k}
\newcommand\vl{\vec l}
\newcommand\vm{\vec m}
\newcommand\vn{\vec n}
\newcommand\vo{\vec o}
\newcommand\vp{\vec p}
\newcommand\vq{\vec q}
\newcommand\vr{\vec r}
\newcommand\vs{\vec s}
\newcommand\vt{\vec t}
\newcommand\vu{\vec u}
\newcommand\vv{\vec v}
\newcommand\vw{\vec w}
\newcommand\vx{\vec x}
\newcommand\vy{\vec y}
\newcommand\vz{\vec z}

\newcommand\NN{\mathbb N}
\newcommand\ZZ{\mathbb Z}
\newcommand\PP{\mathbb P}
\newcommand\QQ{\mathbb Q}
\newcommand\RR{\mathbb R}
\newcommand\CC{\mathbb C}

\newcommand{\pr}{\mathrm{P}}
\newcommand{\Vol}{\mathrm{vol}}
\newcommand\norm[1]{\left\|{#1}\right\|} 
\newcommand\sign{\mathrm{sign}}
\newcommand{\eps}{\varepsilon}
\newcommand{\abs}[1]{\left|#1\right|}
\newcommand\bc[1]{\left({#1}\right)} 
\newcommand\cbc[1]{\left\{{#1}\right\}} 
\newcommand\bcfr[2]{\bc{\frac{#1}{#2}}} 
\newcommand{\bck}[1]{\left\langle{#1}\right\rangle} 
\newcommand\brk[1]{\left\lbrack{#1}\right\rbrack} 
\newcommand\scal[2]{\bck{{#1},{#2}}} 
\newcommand{\vecone}{\mathbb{1}}
\newcommand{\tensor}{\otimes}
\newcommand{\diag}{\mathrm{diag}}
\newcommand{\ggt}{\mathrm{ggT}}
\newcommand{\kgv}{\mathrm{kgV}}

\newcommand{\Karonski}{Karo\'nski}
\newcommand{\Erdos}{Erd\H{o}s}
\newcommand{\Renyi}{R\'enyi}
\newcommand{\Lovasz}{Lov\'asz}
\newcommand{\Juhasz}{Juh\'asz}
\newcommand{\Bollobas}{Bollob\'as}
\newcommand{\Furedi}{F\"uredi}
\newcommand{\Komlos}{Koml\'os}
\newcommand{\Luczak}{\L uczak}
\newcommand{\Kucera}{Ku\v{c}era}
\newcommand{\Szemeredi}{Szemer\'edi}

\renewcommand{\ae}{\"a}
\renewcommand{\oe}{\"o}
\newcommand{\ue}{\"u}
\newcommand{\Ae}{\"A}
\newcommand{\Oe}{\"O}
\newcommand{\Ue}{\"U}

\title[Linadi]{Modulare Arithmetik}
\author[Amin Coja-Oghlan]{Amin Coja-Oghlan}
\institute[Frankfurt]{Goethe Universit\"at Frankfurt}
\date{}

\begin{document}

\frame[plain]{\titlepage}

\begin{frame}\frametitle{Modulo-Schreibweise}
	\begin{block}{Definition}
		Seien $x,y\in\ZZ$ und $m\in\ZZ\setminus\cbc 0$.
		Wir schreiben
		\begin{align*}
			x\equiv y\mod m&&\mbox{falls}&&m\mid x-y
		\end{align*}
		\emph{Sprich:} ``$x$ is kongruent zu $y$ modulo $m$''
	\end{block}
\end{frame}

\begin{frame}\frametitle{Modulo-Schreibweise}
	\begin{block}{Beispiel}
		\begin{itemize}
			\item $12\equiv -9\mod 3$, denn $3\mid 12-(-9)=21$
			\item $0\equiv 14\mod 7$, denn $7\mid 0-14=-14$
			\item $48\equiv 14\mod -2$, denn $-2|34$
			\item $x\equiv x\mod m$ f\ue r alle $x,m$, weil $m|x-x=0$
			\item $x\equiv x+m\mod m$ f\ue r alle $x,m$, weil $m|x-(x+m)=-m$
			\item $x\equiv y\mod m$ genau dann, wenn $y\equiv x\mod m$
		\end{itemize}
	\end{block}
\end{frame}

\begin{frame}\frametitle{Modulo-Schreibweise}
	\begin{block}{Lemma}
		Seien $x,y,x',y'\in\ZZ$ und $m\in\ZZ\setminus\cbc 0$.
		Wenn
		\begin{align*}
			x\equiv y\mod m&&\mbox{und}&&x'\equiv y'\mod m,&&\mbox{dann}\\
			x+x'\equiv y+y'\mod m&&\mbox{und}&&x\cdot x'\equiv y\cdot y'\mod m.
		\end{align*}
	\end{block}
	\begin{block}{Beweis}
		\begin{itemize}
			\item Angenommen $m|x-y$ und $m|x'-y'$.
			\item Dann gilt $m|(x-y)+(x'-y')=(x+x')-(y+y')$.
			\item Ferner $m|x(x'-y')+y'(x-y)=(xx'-yy')$.
		\end{itemize}	
	\end{block}
\end{frame}

\begin{frame}\frametitle{Modulo-Schreibweise}
	\begin{block}{Beispiel}
		\begin{itemize}
			\item $2\equiv 5\mod 3$ und $1\equiv -5\mod 3$
			\item $3=2+1\equiv 5+(-5)=0\mod 3$ 
			\item $2=2\cdot 1\equiv 5\cdot(-5)=-25\mod 3$ 
		\end{itemize}
	\end{block}
\end{frame}

\begin{frame}\frametitle{Modulo-Schreibweise}
	\begin{block}{Lemma}
		Angenommen $x,y\in\ZZ$, $m,n\in\ZZ\setminus\cbc 0$ und $n\mid m$.	
		Wenn
		\begin{align*}
			x\equiv y\mod m&&\mbox{dann}&&x\equiv y\mod n.
		\end{align*}
	\end{block}
	\begin{block}{Beweis}
		Es gilt	$n\mid m\mid x-y.$
	\end{block}
\end{frame}

\begin{frame}\frametitle{Modulo-Schreibweise}
	\begin{block}{Lemma}
		Angenommen $x,y\in\ZZ$, $m,n\in\ZZ\setminus\cbc 0$ und $\ggt(m,n)=1$.	
		Wenn
		\begin{align*}
			x\equiv y\mod m&&\mbox{und}&&x\equiv y\mod n&&\mbox{dann}&&x\equiv y\mod m\cdot n
		\end{align*}
	\end{block}
	\begin{block}{Beweis}
		\begin{itemize}
			\item Weil $\ggt(m,n)=1$, sind die Primteiler von $m,n$ verschieden.
			\item Weil $m\mid x-y$, gilt $w_p(x-y)\geq w_p(m)$ f\ue r alle $p\in\PP$.
			\item Weil $n\mid x-y$, gilt $w_p(x-y)\geq w_p(n)$ f\ue r alle $p\in\PP$.
			\item Folglich $w_p(x-y)\geq w_p(m)+w_p(n)=w_p(mn)$ f\ue r alle $p\in\PP$.
			\item Also $mn|x-y$.
		\end{itemize}
	\end{block}
\end{frame}

\begin{frame}\frametitle{Der chinesische Restsatz}
	\begin{block}{Satz}
		Angenommen $m,n\in\NN$ erf\ue llen $\ggt(m,n)=1$.	
		Dann gibt es f\ue r je zwei ganze Zahlen $x,y$ eine ganze Zahl $z$, so da\ss\
		\begin{align*}
			z\equiv x\mod m&&\mbox{und}&&z\equiv y\mod n.
		\end{align*}
	\end{block}
	\begin{overprint}
		\onslide<1>
		\begin{block}{Anmerkung}
			\begin{itemize}
				\item wir nennen zwei Zahlen $a,b\in\ZZ$ \emph{teilerfremd} oder \emph{relativ prim}, wenn $\ggt(a,b)=1$.
				\item ohne die Annahme, da\ss\ $m,n$ teilerfremd sind, trifft die Aussage des chinesischen Restsatzes nicht zu.
			\end{itemize}	
		\end{block}
		\onslide<2>
		\begin{block}{Bijektive Abbildungen}
			\begin{itemize}
				\item Seien $A,B$ Mengen und $f:A\to B$ eine Abbildung.
				\item $f$ hei\ss t \emph{injektiv}, falls aus $f(a)=f(b)$ folgt, da\ss\ $a=b$.
				\item $f$ hei\ss t \emph{surjketiv}, falls zu jedem $b\in B$ ein $a\in A$ mit $f(a)=b$ existiert.
				\item $f$ hei\ss t \emph{bijektiv}, falls beides zutrifft.
				\item Wenn $f$ bijektiv ist, gibt es eine Umkehrabbildung $f^{-1}:B\to A$.
			\end{itemize}
		\end{block}
		\onslide<3>
		\begin{block}{Bijektive Abbildungen}
			\begin{itemize}
				\item Seien $A,B$ endliche Mengen mit $|A|=|B|$.
				\item Wenn $f:A\to B$ injektiv ist, dann ist $f$ surjketiv.
				\item Wenn $f:A\to B$ surjketiv ist, dann ist $f$ injektiv.
			\end{itemize}
		\end{block}
		\onslide<4>
		\begin{block}{Beweis}
			\begin{itemize}
				\item Sei $A=\{0,1,2,\ldots,m\cdot n-1\}$ und $B=\{(i,j):0\leq i< m,\, 0\leq j< n\}$.
				\item Es gilt $|A|=|B|$.
				\item Jede Zahl $a\in A$ kann mit Rest durch $m,n$ geteilt werden:
					\begin{align*}
						a=p\cdot m+r,&&a=q\cdot n+s&&0\leq r<m,\ 0\leq s<n
					\end{align*}
				\item Definiere $f:A\to B$ durch $a\mapsto (r,s)$.
			\end{itemize}	
		\end{block}
		\onslide<5>
		\begin{block}{Beweis (Fortsetzung)}
			\begin{itemize}
				\item Die Abbildung $f$ ist injektiv.
				\item Denn angenommen $f(a)=f(a')$; dann gilt
					\begin{align*}
						a&=p\cdot m+r,&a&=q\cdot n+s\\ a'&=p'\cdot m+r,&a'&=q'\cdot n+s.
					\end{align*}
				\item Also $m|a-a'$ und $n|a-a'$.
			\end{itemize}	
		\end{block}
		\onslide<6>
		\begin{block}{Beweis (Fortsetzung)}
			\begin{itemize}
				\item Weil $\ggt(m,n)=1$, folgt $mn\mid a-a'$.
				\item Weil $0\leq a,a'<mn$, schliessen wir $a-a'=0$; also $f$ ist injektiv.
				\item Weil $|A|=|B|$ ist $f$ auch surjektiv.
			\end{itemize}	
		\end{block}
		\onslide<7>
		\begin{block}{Beweis (Fortsetzung)}
			\begin{itemize}
				\item Indem wir $x,y$ durch $m,n$ teilen, finden wir $(r,s)\in B$ mit
					\begin{align*}
						x\equiv r\mod m,&&&&y\equiv s\mod n.
					\end{align*}
				\item Das Urbild $z=f^{-1}(r,s)$ erf\ue llt dann
					\begin{align*}
						z\equiv r\equiv x\mod m,&&&&z\equiv s\equiv y\mod n.
					\end{align*}
			\end{itemize}	
		\end{block}
	\end{overprint}
\end{frame}

\begin{frame}\frametitle{Der chinesische Restsatz}
	\begin{block}{Konstruktion}
		\begin{itemize}
			\item \emph{Gegeben:} teilerfremde $m,n\in\NN$ und $x,y\in\ZZ$.
			\item \emph{Gesucht:} $z\in\ZZ$ mit $z\equiv x\mod m$ und $z\equiv y\mod n$.
			\item Mit dem erweiterten Euklidischen Algorithmus finde $a,b$ mit
				\begin{align*}
					am+bn=1.
				\end{align*}
			\item Dann gilt
				\begin{align*}
					bn&\equiv 1\mod m&bn&\equiv0\mod n\\
					am&\equiv 0\mod m&am&\equiv1\mod n
				\end{align*}
			\item Also erf\ue llt $z=amy+bnx$
				\begin{align*}
					z&\equiv x\mod m,&z&\equiv y\mod n.
				\end{align*}
		\end{itemize}
	\end{block}
\end{frame}

\begin{frame}\frametitle{Der chinesische Restsatz}
	\begin{block}{Beispiel}
		\begin{itemize}
			\item \emph{Gegeben:} $m=7$, $n=12$, $x=2$, $y=4$
			\item Eukldischer Algorithmus:
				\begin{align*}
					12&=1\cdot 7+5&&7=1\cdot 5+2&&7=3\cdot 2+1
				\end{align*}
			\item Mit $a=-5$ und $b=3$ erhalten wir
				\begin{align*}
				am+bn=-5\cdot 7+3\cdot 12=1
				\end{align*}
			\item Also lautet die (oder genauer: eine) L\oe sung
				\begin{align*}
				z=amy+bnx=-5\cdot 7\cdot 4+3\cdot 12\cdot 2=-68.
				\end{align*}
		\end{itemize}
	\end{block}
\end{frame}

\begin{frame}\frametitle{Schnelles Potenzieren}
	\begin{block}{Problemstellung}
		\begin{itemize}
			\item Gegeben $x\in\ZZ$ und $\ell,m\in\NN$, finde $z\in\ZZ$ mit
				\begin{align*}
					x^\ell\equiv z\mod m.
				\end{align*}
			\item \emph{Ziel:} effiziente Berechnung.
			\item Die Zahl der elementaren Rechenoperationen soll vern\ue nftig in der L\ae nge der Dezimaldarstellungen von $x,m,\ell$ skalieren!
			\item Au\ss erdem sollen die Zwischenergebnis nicht zu gro\ss\ werden!
		\end{itemize}
	\end{block}
\end{frame}

\begin{frame}\frametitle{Schnelles Potenzieren}
	\begin{block}{Algorithmus}
		\begin{enumerate}
			\item Bestimme die Darstellung von $\ell$ im Dualsystem:
				\begin{align*}
					\ell&=\sum_{i=0}^k\ell_i2^i,&\ell_i&\in\{0,1\},&\ell_k&\neq0,&k\geq0.
				\end{align*}
			\item Sei $y_0$ der Divisonsrest von $x$ durch $m$.
			\item F\ue r $i=1,\ldots,k$:
			\item $\qquad$sei $y_i$ der Divisionsrest von $y_{i-1}^2$ durch $m$.
			\item Setze $z=1$.
			\item F\ue r $i=0,\ldots,k$:
			\item $\qquad$sei $r$ der Rest von $z\cdot y_i^{\ell_i}$ durch $m$.
			\item $\qquad$setze $z$ auf den Wert $r$.
			\item Gib $z$ aus.
		\end{enumerate}
	\end{block}
\end{frame}

\begin{frame}\frametitle{Schnelles Potenzieren}
	\begin{block}{Anmerkungen}
		\begin{itemize}
			\item \emph{Wichtig:} die Schritte 4,7 bilden den Rest bei Division durch $m$.
			\item Deshalb treten nur mit Zahlen bis zu $(|x|+|m|)^2$ auf.
			\item Die Zahl der Multiplikationen ist durch $k\leq2(1+\log_2\ell)$ beschr\ae nkt.
		\end{itemize}
	\end{block}
\end{frame}

\begin{frame}\frametitle{Schnelles Potenzieren}
	\begin{block}{Beispiel}
		\begin{itemize}
			\item $m=7$, $\ell=29$, $x=2$
			\item $29=2^4+2^3+2^2+2^0$
			\item $k=4$, $\ell_0=\ell_2=\ell_3=\ell_4=1$, $\ell_1=0$
			\item $y_0=2$, $y_1=4$, $y_2=2$, $y_3=4$, $y_4=2$ 
			\item Folge der $z$-Werte:
				\begin{align*}
					1,&&2,&&2,&&4,&&2,&&4
				\end{align*}
			\item \emph{Ergebnis:} $z=4$
		\end{itemize}
	\end{block}
\end{frame}

\begin{frame}\frametitle{Zusammenfassung}
	\begin{itemize}
		\item Die Modulo-Schreibweise
		\item Chinesischer Restsatz
		\item Schnelles Potenzieren
	\end{itemize}
\end{frame}

\end{document}
