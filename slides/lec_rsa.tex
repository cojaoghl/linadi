\documentclass{beamer}
\usepackage{amsmath,graphics}
\usepackage{amssymb}

\usetheme{default}
\usepackage{xcolor}

\definecolor{solarizedBase03}{HTML}{002B36}
\definecolor{solarizedBase02}{HTML}{073642}
\definecolor{solarizedBase01}{HTML}{586e75}
\definecolor{solarizedBase00}{HTML}{657b83}
\definecolor{solarizedBase0}{HTML}{839496}
\definecolor{solarizedBase1}{HTML}{93a1a1}
\definecolor{solarizedBase2}{HTML}{EEE8D5}
\definecolor{solarizedBase3}{HTML}{FDF6E3}
\definecolor{solarizedYellow}{HTML}{B58900}
\definecolor{solarizedOrange}{HTML}{CB4B16}
\definecolor{solarizedRed}{HTML}{DC322F}
\definecolor{solarizedMagenta}{HTML}{D33682}
\definecolor{solarizedViolet}{HTML}{6C71C4}
%\definecolor{solarizedBlue}{HTML}{268BD2}
\definecolor{solarizedBlue}{HTML}{134676}
\definecolor{solarizedCyan}{HTML}{2AA198}
\definecolor{solarizedGreen}{HTML}{859900}
\definecolor{myBlue}{HTML}{162DB0}%{261CA4}
\setbeamercolor*{item}{fg=myBlue}
\setbeamercolor{normal text}{fg=solarizedBase03, bg=solarizedBase3}
\setbeamercolor{alerted text}{fg=myBlue}
\setbeamercolor{example text}{fg=myBlue, bg=solarizedBase3}
\setbeamercolor*{frametitle}{fg=solarizedRed}
\setbeamercolor*{title}{fg=solarizedRed}
\setbeamercolor{block title}{fg=myBlue, bg=solarizedBase3}
\setbeameroption{hide notes}
\setbeamertemplate{note page}[plain]
\beamertemplatenavigationsymbolsempty
\usefonttheme{professionalfonts}
\usefonttheme{serif}

\usepackage{fourier}

\def\vec#1{\mathchoice{\mbox{\boldmath$\displaystyle#1$}}
{\mbox{\boldmath$\textstyle#1$}}
{\mbox{\boldmath$\scriptstyle#1$}}
{\mbox{\boldmath$\scriptscriptstyle#1$}}}
\definecolor{OwnGrey}{rgb}{0.560,0.000,0.000} % #999999
\definecolor{OwnBlue}{rgb}{0.121,0.398,0.711} % #1f64b0
\definecolor{red4}{rgb}{0.5,0,0}
\definecolor{blue4}{rgb}{0,0,0.5}
\definecolor{Blue}{rgb}{0,0,0.66}
\definecolor{LightBlue}{rgb}{0.9,0.9,1}
\definecolor{Green}{rgb}{0,0.5,0}
\definecolor{LightGreen}{rgb}{0.9,1,0.9}
\definecolor{Red}{rgb}{0.9,0,0}
\definecolor{LightRed}{rgb}{1,0.9,0.9}
\definecolor{White}{gray}{1}
\definecolor{Black}{gray}{0}
\definecolor{LightGray}{gray}{0.8}
\definecolor{Orange}{rgb}{0.1,0.2,1}
\setbeamerfont{sidebar right}{size=\scriptsize}
\setbeamercolor{sidebar right}{fg=Black}

\renewcommand{\emph}[1]{{\textcolor{solarizedRed}{\itshape #1}}}

\newcommand\cA{\mathcal A}
\newcommand\cB{\mathcal B}
\newcommand\cC{\mathcal C}
\newcommand\cD{\mathcal D}
\newcommand\cE{\mathcal E}
\newcommand\cF{\mathcal F}
\newcommand\cG{\mathcal G}
\newcommand\cH{\mathcal H}
\newcommand\cI{\mathcal I}
\newcommand\cJ{\mathcal J}
\newcommand\cK{\mathcal K}
\newcommand\cL{\mathcal L}
\newcommand\cM{\mathcal M}
\newcommand\cN{\mathcal N}
\newcommand\cO{\mathcal O}
\newcommand\cP{\mathcal P}
\newcommand\cQ{\mathcal Q}
\newcommand\cR{\mathcal R}
\newcommand\cS{\mathcal S}
\newcommand\cT{\mathcal T}
\newcommand\cU{\mathcal U}
\newcommand\cV{\mathcal V}
\newcommand\cW{\mathcal W}
\newcommand\cX{\mathcal X}
\newcommand\cY{\mathcal Y}
\newcommand\cZ{\mathcal Z}

\newcommand\fA{\mathfrak A}
\newcommand\fB{\mathfrak B}
\newcommand\fC{\mathfrak C}
\newcommand\fD{\mathfrak D}
\newcommand\fE{\mathfrak E}
\newcommand\fF{\mathfrak F}
\newcommand\fG{\mathfrak G}
\newcommand\fH{\mathfrak H}
\newcommand\fI{\mathfrak I}
\newcommand\fJ{\mathfrak J}
\newcommand\fK{\mathfrak K}
\newcommand\fL{\mathfrak L}
\newcommand\fM{\mathfrak M}
\newcommand\fN{\mathfrak N}
\newcommand\fO{\mathfrak O}
\newcommand\fP{\mathfrak P}
\newcommand\fQ{\mathfrak Q}
\newcommand\fR{\mathfrak R}
\newcommand\fS{\mathfrak S}
\newcommand\fT{\mathfrak T}
\newcommand\fU{\mathfrak U}
\newcommand\fV{\mathfrak V}
\newcommand\fW{\mathfrak W}
\newcommand\fX{\mathfrak X}
\newcommand\fY{\mathfrak Y}
\newcommand\fZ{\mathfrak Z}

\newcommand\fa{\mathfrak a}
\newcommand\fb{\mathfrak b}
\newcommand\fc{\mathfrak c}
\newcommand\fd{\mathfrak d}
\newcommand\fe{\mathfrak e}
\newcommand\ff{\mathfrak f}
\newcommand\fg{\mathfrak g}
\newcommand\fh{\mathfrak h}
%\newcommand\fi{\mathfrak i}
\newcommand\fj{\mathfrak j}
\newcommand\fk{\mathfrak k}
\newcommand\fl{\mathfrak l}
\newcommand\fm{\mathfrak m}
\newcommand\fn{\mathfrak n}
\newcommand\fo{\mathfrak o}
\newcommand\fp{\mathfrak p}
\newcommand\fq{\mathfrak q}
\newcommand\fr{\mathfrak r}
\newcommand\fs{\mathfrak s}
\newcommand\ft{\mathfrak t}
\newcommand\fu{\mathfrak u}
\newcommand\fv{\mathfrak v}
\newcommand\fw{\mathfrak w}
\newcommand\fx{\mathfrak x}
\newcommand\fy{\mathfrak y}
\newcommand\fz{\mathfrak z}

\newcommand\vA{\vec A}
\newcommand\vB{\vec B}
\newcommand\vC{\vec C}
\newcommand\vD{\vec D}
\newcommand\vE{\vec E}
\newcommand\vF{\vec F}
\newcommand\vG{\vec G}
\newcommand\vH{\vec H}
\newcommand\vI{\vec I}
\newcommand\vJ{\vec J}
\newcommand\vK{\vec K}
\newcommand\vL{\vec L}
\newcommand\vM{\vec M}
\newcommand\vN{\vec N}
\newcommand\vO{\vec O}
\newcommand\vP{\vec P}
\newcommand\vQ{\vec Q}
\newcommand\vR{\vec R}
\newcommand\vS{\vec S}
\newcommand\vT{\vec T}
\newcommand\vU{\vec U}
\newcommand\vV{\vec V}
\newcommand\vW{\vec W}
\newcommand\vX{\vec X}
\newcommand\vY{\vec Y}
\newcommand\vZ{\vec Z}

\newcommand\va{\vec a}
\newcommand\vb{\vec b}
\newcommand\vc{\vec c}
\newcommand\vd{\vec d}
\newcommand\ve{\vec e}
\newcommand\vf{\vec f}
\newcommand\vg{\vec g}
\newcommand\vh{\vec h}
\newcommand\vi{\vec i}
\newcommand\vj{\vec j}
\newcommand\vk{\vec k}
\newcommand\vl{\vec l}
\newcommand\vm{\vec m}
\newcommand\vn{\vec n}
\newcommand\vo{\vec o}
\newcommand\vp{\vec p}
\newcommand\vq{\vec q}
\newcommand\vr{\vec r}
\newcommand\vs{\vec s}
\newcommand\vt{\vec t}
\newcommand\vu{\vec u}
\newcommand\vv{\vec v}
\newcommand\vw{\vec w}
\newcommand\vx{\vec x}
\newcommand\vy{\vec y}
\newcommand\vz{\vec z}

\renewcommand\AA{\mathbb A}
\newcommand\NN{\mathbb N}
\newcommand\ZZ{\mathbb Z}
\newcommand\PP{\mathbb P}
\newcommand\QQ{\mathbb Q}
\newcommand\RR{\mathbb R}
\renewcommand\SS{\mathbb S}
\newcommand\CC{\mathbb C}

\newcommand{\ord}{\mathrm{ord}}
\newcommand{\id}{\mathrm{id}}
\newcommand{\pr}{\mathrm{P}}
\newcommand{\Vol}{\mathrm{vol}}
\newcommand\norm[1]{\left\|{#1}\right\|} 
\newcommand\sign{\mathrm{sign}}
\newcommand{\eps}{\varepsilon}
\newcommand{\abs}[1]{\left|#1\right|}
\newcommand\bc[1]{\left({#1}\right)} 
\newcommand\cbc[1]{\left\{{#1}\right\}} 
\newcommand\bcfr[2]{\bc{\frac{#1}{#2}}} 
\newcommand{\bck}[1]{\left\langle{#1}\right\rangle} 
\newcommand\brk[1]{\left\lbrack{#1}\right\rbrack} 
\newcommand\scal[2]{\bck{{#1},{#2}}} 
\newcommand{\vecone}{\mathbb{1}}
\newcommand{\tensor}{\otimes}
\newcommand{\diag}{\mathrm{diag}}
\newcommand{\ggt}{\mathrm{ggT}}
\newcommand{\kgv}{\mathrm{kgV}}

\newcommand{\Karonski}{Karo\'nski}
\newcommand{\Erdos}{Erd\H{o}s}
\newcommand{\Renyi}{R\'enyi}
\newcommand{\Lovasz}{Lov\'asz}
\newcommand{\Juhasz}{Juh\'asz}
\newcommand{\Bollobas}{Bollob\'as}
\newcommand{\Furedi}{F\"uredi}
\newcommand{\Komlos}{Koml\'os}
\newcommand{\Luczak}{\L uczak}
\newcommand{\Kucera}{Ku\v{c}era}
\newcommand{\Szemeredi}{Szemer\'edi}

\renewcommand{\ae}{\"a}
\renewcommand{\oe}{\"o}
\newcommand{\ue}{\"u}
\newcommand{\Ae}{\"A}
\newcommand{\Oe}{\"O}
\newcommand{\Ue}{\"U}

\newcommand{\mytitle}{Das RSA-Kryptosystem}

\title[Linadi]{\mytitle}
\author[Amin Coja-Oghlan]{Amin Coja-Oghlan}
\institute[Frankfurt]{JWGUFFM}
\date{}

\begin{document}

\frame[plain]{\titlepage}

\begin{frame}\frametitle{\mytitle}
	\begin{block}{Kryptographie mit \oe ffentlichen Schl\ue sseln}
		\begin{itemize}
			\item Kommunikation ohne geheimen Kanal.
			\item Verschl\ue sselungsverfahren, das einfach anzuwenden ist.
			\item Dazu stellt der Empf\ae nger einen \alert{\oe ffentlichen Schl\ue ssel} bereit.
			\item Die Entschl\ue sselung sollte mit geheimer Zusatzinformation, dem \alert{geheimen Schl\ue ssel}, einfach aufgehoben werden k\oe nnen.
			\item Ohne den geheimen Schl\ue ssel sollte die Entschl\ue sselung extrem aufw\ae ndig sein.
		\end{itemize}
	\end{block}
\end{frame}

\begin{frame}\frametitle{\mytitle}
	\begin{block}{Das Faktorisierungsproblem als Einwegfunktion}
		\begin{itemize}
			\item Derzeit ist kein effizientes Verfahren bekannt, das eine gegebene Zahl $n\in\NN$ faktorisiert.
			\item Hingegen ist es leicht, gro\ss e Primzahlen zu erzeugen und aufzumultiplizieren.
			\item \emph{Idee:} erzeuge gro\ss e Primzahlen $p,q$ und verwende
				\begin{align*}
					n=p\cdot q
				\end{align*}
				als \oe ffentlichen Schl\ue ssel.
			\item Dann sollte es schwer sein, $p,q$ aus $n$ zu rekonstruieren.
			\item \emph{Frage:} effizientes Ver- und Entschl\ue sseln?
		\end{itemize}
	\end{block}
\end{frame}

\begin{frame}\frametitle{\mytitle}
	\begin{block}{Rivest--Shamir--Adleman}
		{\itshape Erzeugen des \oe ffentliche Schl\ue ssels}
		\begin{enumerate}
			\item \alert{Alice} erzeugt zwei zuf\ae llige gro\ss e Primzahlen $p,q$.
			\item Setze $n=pq$.
			\item Au\ss erdem w\ae hlt sie $0<e<n$ mit
				\begin{align*}
					\ggt(e,(p-1)(q-1))=1.
				\end{align*}
			\item Ferner berechnet Alice eine Zahl $0<d<n$, so da\ss\
				\begin{align*}
					ed\equiv1\mod(p-1)(q-1).
				\end{align*}
			\item Der \oe ffentliche Schl\ue ssel lautet $(n,e)$.
			\item Der private Schl\ue ssel ist $d$.
		\end{enumerate}
			(Alice braucht sich $p,q$ nicht mehr zu merken.)
	\end{block}
\end{frame}

\begin{frame}\frametitle{\mytitle}
	\begin{block}{Rivest--Shamir--Adleman}
		{\itshape Effizienz}
		\begin{enumerate}
			\item siehe letzte Vorlesung (Miller-Rabin).
			\item einfache Multiplikation.
			\item w\ae hle z.B.\ eine weitere zuf\ae llige Primzahl $e<(p-1)(q-1)$.
			\item erweiterter Euklidischer Algorithmus.
		\end{enumerate}
	\end{block}
\end{frame}

\begin{frame}\frametitle{\mytitle}
	\begin{block}{Rivest--Shamir--Adleman}
		{\itshape Verschl\ue sselungsverfahren}
		\begin{itemize}
			\item \emph{Bob} m\oe chte Alice eine Nachricht senden.
			\item Er kennt den \oe ffentlichen Schl\ue ssel $(n,e)$.
			\item Die Nachricht $m$ ist eine Zahl $1<m<n$.
			\item Bob verschl\ue sselt die Nachricht, indem er $0\leq M<n$ findet mit
				\begin{align*}
				M\equiv m^e\mod n.
				\end{align*}
			\item Dazu verwendet Bob schnelles Potenzieren.
			\item $M$ ist also die verschl\ue sselte Botschaft.
		\end{itemize}
	\end{block}
\end{frame}

\begin{frame}\frametitle{\mytitle}
	\begin{block}{Rivest--Shamir--Adleman}
		{\itshape Entschl\ue sselungsverfahren}
		\begin{itemize}
			\item \emph{Alice} empf\ae ngt die verschl\ue sselte Botschaft $M$.
			\item Sie entschl\ue sselt, indem sie $0\leq E<n$ mit
				\begin{align*}
				E\equiv M^d\mod n
				\end{align*}
				berechnet.
			\item Dazu verwendet sie schnelles Potenzieren.
			\item Aus dem \alert{Kleinen Fermat} folgt, da\ss\ $E=m$.
		\end{itemize}
	\end{block}
\end{frame}

\begin{frame}\frametitle{\mytitle}
	\begin{block}{Rivest--Shamir--Adleman}
		{\itshape Wie sicher ist das Verfahren?}
		\begin{itemize}
			\item RSA ist eines der Arbeitspferde der Kryptographie.
			\item Wenn ein Angreifer $n$ faktorisiert, kann er den privaten Schl\ue ssel $d$ bestimmen.
			\item Es ist (mir) nicht bekannt, da\ss\ es auch ohne Faktorisierung geht.
			\item (Mir) ist kein effizientes Faktorisierungsverfahren bekannt.
			\item \emph{Ausnahme:} Quantencomputer $\leadsto$ Shor-Algorithmus.
		\end{itemize}
	\end{block}
\end{frame}

\begin{frame}\frametitle{\mytitle}
	\begin{block}{Rivest--Shamir--Adleman}
		{\itshape Praktische Hinweise}
		\begin{itemize}
			\item Wir haben eine theoretische Version von RSA kennengelernt.
			\item \emph{Verwenden Sie das Verfahren so nicht!}
			\item Zu praktischen Fallstricken s.\ z.B.\ {\itshape D.~Knuth: The Art of Computer Programming, Seminumerical Algorithms}.
			\item Auf Unix-Systemen verwenden Sie z.B.\ $$\mbox{\tt ssh-keygen -t rsa -b 4096}$$
				zum Erzeugen eines Schl\ue sselpaares.
		\end{itemize}
	\end{block}
\end{frame}

\begin{frame}\frametitle{\mytitle}
	\begin{block}{Zusammenfassung}
		\begin{itemize}
			\item das RSA-Verfahren benutzt \oe ffentliche Schl\ue ssel.
			\item mit Hilfe der Werkzeuge aus den vergangenen Wochen k\oe nnen wir RSA effizient implementieren.
			\item \alert{Hypothese:} das Faktorisierungsproblem ist nicht effizient l\oe sbar.
		\end{itemize}
	\end{block}
\end{frame}

\end{document}
