\documentclass{beamer}
\usepackage{amsmath,graphics}
\usepackage{amssymb}

\usetheme{default}
\usepackage{xcolor}

\definecolor{solarizedBase03}{HTML}{002B36}
\definecolor{solarizedBase02}{HTML}{073642}
\definecolor{solarizedBase01}{HTML}{586e75}
\definecolor{solarizedBase00}{HTML}{657b83}
\definecolor{solarizedBase0}{HTML}{839496}
\definecolor{solarizedBase1}{HTML}{93a1a1}
\definecolor{solarizedBase2}{HTML}{EEE8D5}
\definecolor{solarizedBase3}{HTML}{FDF6E3}
\definecolor{solarizedYellow}{HTML}{B58900}
\definecolor{solarizedOrange}{HTML}{CB4B16}
\definecolor{solarizedRed}{HTML}{DC322F}
\definecolor{solarizedMagenta}{HTML}{D33682}
\definecolor{solarizedViolet}{HTML}{6C71C4}
%\definecolor{solarizedBlue}{HTML}{268BD2}
\definecolor{solarizedBlue}{HTML}{134676}
\definecolor{solarizedCyan}{HTML}{2AA198}
\definecolor{solarizedGreen}{HTML}{859900}
\definecolor{myBlue}{HTML}{162DB0}%{261CA4}
\setbeamercolor*{item}{fg=myBlue}
\setbeamercolor{normal text}{fg=solarizedBase03, bg=solarizedBase3}
\setbeamercolor{alerted text}{fg=myBlue}
\setbeamercolor{example text}{fg=myBlue, bg=solarizedBase3}
\setbeamercolor*{frametitle}{fg=solarizedRed}
\setbeamercolor*{title}{fg=solarizedRed}
\setbeamercolor{block title}{fg=myBlue, bg=solarizedBase3}
\setbeameroption{hide notes}
\setbeamertemplate{note page}[plain]
\beamertemplatenavigationsymbolsempty
\usefonttheme{professionalfonts}
\usefonttheme{serif}

\usepackage{fourier}

\def\vec#1{\mathchoice{\mbox{\boldmath$\displaystyle#1$}}
{\mbox{\boldmath$\textstyle#1$}}
{\mbox{\boldmath$\scriptstyle#1$}}
{\mbox{\boldmath$\scriptscriptstyle#1$}}}
\definecolor{OwnGrey}{rgb}{0.560,0.000,0.000} % #999999
\definecolor{OwnBlue}{rgb}{0.121,0.398,0.711} % #1f64b0
\definecolor{red4}{rgb}{0.5,0,0}
\definecolor{blue4}{rgb}{0,0,0.5}
\definecolor{Blue}{rgb}{0,0,0.66}
\definecolor{LightBlue}{rgb}{0.9,0.9,1}
\definecolor{Green}{rgb}{0,0.5,0}
\definecolor{LightGreen}{rgb}{0.9,1,0.9}
\definecolor{Red}{rgb}{0.9,0,0}
\definecolor{LightRed}{rgb}{1,0.9,0.9}
\definecolor{White}{gray}{1}
\definecolor{Black}{gray}{0}
\definecolor{LightGray}{gray}{0.8}
\definecolor{Orange}{rgb}{0.1,0.2,1}
\setbeamerfont{sidebar right}{size=\scriptsize}
\setbeamercolor{sidebar right}{fg=Black}

\renewcommand{\emph}[1]{{\textcolor{solarizedRed}{\itshape #1}}}

\newcommand\cA{\mathcal A}
\newcommand\cB{\mathcal B}
\newcommand\cC{\mathcal C}
\newcommand\cD{\mathcal D}
\newcommand\cE{\mathcal E}
\newcommand\cF{\mathcal F}
\newcommand\cG{\mathcal G}
\newcommand\cH{\mathcal H}
\newcommand\cI{\mathcal I}
\newcommand\cJ{\mathcal J}
\newcommand\cK{\mathcal K}
\newcommand\cL{\mathcal L}
\newcommand\cM{\mathcal M}
\newcommand\cN{\mathcal N}
\newcommand\cO{\mathcal O}
\newcommand\cP{\mathcal P}
\newcommand\cQ{\mathcal Q}
\newcommand\cR{\mathcal R}
\newcommand\cS{\mathcal S}
\newcommand\cT{\mathcal T}
\newcommand\cU{\mathcal U}
\newcommand\cV{\mathcal V}
\newcommand\cW{\mathcal W}
\newcommand\cX{\mathcal X}
\newcommand\cY{\mathcal Y}
\newcommand\cZ{\mathcal Z}

\newcommand\fA{\mathfrak A}
\newcommand\fB{\mathfrak B}
\newcommand\fC{\mathfrak C}
\newcommand\fD{\mathfrak D}
\newcommand\fE{\mathfrak E}
\newcommand\fF{\mathfrak F}
\newcommand\fG{\mathfrak G}
\newcommand\fH{\mathfrak H}
\newcommand\fI{\mathfrak I}
\newcommand\fJ{\mathfrak J}
\newcommand\fK{\mathfrak K}
\newcommand\fL{\mathfrak L}
\newcommand\fM{\mathfrak M}
\newcommand\fN{\mathfrak N}
\newcommand\fO{\mathfrak O}
\newcommand\fP{\mathfrak P}
\newcommand\fQ{\mathfrak Q}
\newcommand\fR{\mathfrak R}
\newcommand\fS{\mathfrak S}
\newcommand\fT{\mathfrak T}
\newcommand\fU{\mathfrak U}
\newcommand\fV{\mathfrak V}
\newcommand\fW{\mathfrak W}
\newcommand\fX{\mathfrak X}
\newcommand\fY{\mathfrak Y}
\newcommand\fZ{\mathfrak Z}

\newcommand\fa{\mathfrak a}
\newcommand\fb{\mathfrak b}
\newcommand\fc{\mathfrak c}
\newcommand\fd{\mathfrak d}
\newcommand\fe{\mathfrak e}
\newcommand\ff{\mathfrak f}
\newcommand\fg{\mathfrak g}
\newcommand\fh{\mathfrak h}
%\newcommand\fi{\mathfrak i}
\newcommand\fj{\mathfrak j}
\newcommand\fk{\mathfrak k}
\newcommand\fl{\mathfrak l}
\newcommand\fm{\mathfrak m}
\newcommand\fn{\mathfrak n}
\newcommand\fo{\mathfrak o}
\newcommand\fp{\mathfrak p}
\newcommand\fq{\mathfrak q}
\newcommand\fr{\mathfrak r}
\newcommand\fs{\mathfrak s}
\newcommand\ft{\mathfrak t}
\newcommand\fu{\mathfrak u}
\newcommand\fv{\mathfrak v}
\newcommand\fw{\mathfrak w}
\newcommand\fx{\mathfrak x}
\newcommand\fy{\mathfrak y}
\newcommand\fz{\mathfrak z}

\newcommand\vA{\vec A}
\newcommand\vB{\vec B}
\newcommand\vC{\vec C}
\newcommand\vD{\vec D}
\newcommand\vE{\vec E}
\newcommand\vF{\vec F}
\newcommand\vG{\vec G}
\newcommand\vH{\vec H}
\newcommand\vI{\vec I}
\newcommand\vJ{\vec J}
\newcommand\vK{\vec K}
\newcommand\vL{\vec L}
\newcommand\vM{\vec M}
\newcommand\vN{\vec N}
\newcommand\vO{\vec O}
\newcommand\vP{\vec P}
\newcommand\vQ{\vec Q}
\newcommand\vR{\vec R}
\newcommand\vS{\vec S}
\newcommand\vT{\vec T}
\newcommand\vU{\vec U}
\newcommand\vV{\vec V}
\newcommand\vW{\vec W}
\newcommand\vX{\vec X}
\newcommand\vY{\vec Y}
\newcommand\vZ{\vec Z}

\newcommand\va{\vec a}
\newcommand\vb{\vec b}
\newcommand\vc{\vec c}
\newcommand\vd{\vec d}
\newcommand\ve{\vec e}
\newcommand\vf{\vec f}
\newcommand\vg{\vec g}
\newcommand\vh{\vec h}
\newcommand\vi{\vec i}
\newcommand\vj{\vec j}
\newcommand\vk{\vec k}
\newcommand\vl{\vec l}
\newcommand\vm{\vec m}
\newcommand\vn{\vec n}
\newcommand\vo{\vec o}
\newcommand\vp{\vec p}
\newcommand\vq{\vec q}
\newcommand\vr{\vec r}
\newcommand\vs{\vec s}
\newcommand\vt{\vec t}
\newcommand\vu{\vec u}
\newcommand\vv{\vec v}
\newcommand\vw{\vec w}
\newcommand\vx{\vec x}
\newcommand\vy{\vec y}
\newcommand\vz{\vec z}

\renewcommand\AA{\mathbb A}
\newcommand\NN{\mathbb N}
\newcommand\ZZ{\mathbb Z}
\newcommand\PP{\mathbb P}
\newcommand\QQ{\mathbb Q}
\newcommand\RR{\mathbb R}
\renewcommand\SS{\mathbb S}
\newcommand\CC{\mathbb C}

\newcommand{\ord}{\mathrm{ord}}
\newcommand{\id}{\mathrm{id}}
\newcommand{\pr}{\mathrm{P}}
\newcommand{\Vol}{\mathrm{vol}}
\newcommand\norm[1]{\left\|{#1}\right\|} 
\newcommand\sign{\mathrm{sign}}
\newcommand{\eps}{\varepsilon}
\newcommand{\abs}[1]{\left|#1\right|}
\newcommand\bc[1]{\left({#1}\right)} 
\newcommand\cbc[1]{\left\{{#1}\right\}} 
\newcommand\bcfr[2]{\bc{\frac{#1}{#2}}} 
\newcommand{\bck}[1]{\left\langle{#1}\right\rangle} 
\newcommand\brk[1]{\left\lbrack{#1}\right\rbrack} 
\newcommand\scal[2]{\bck{{#1},{#2}}} 
\newcommand{\vecone}{\mathbb{1}}
\newcommand{\tensor}{\otimes}
\newcommand{\diag}{\mathrm{diag}}
\newcommand{\ggt}{\mathrm{ggT}}
\newcommand{\kgv}{\mathrm{kgV}}

\newcommand{\Karonski}{Karo\'nski}
\newcommand{\Erdos}{Erd\H{o}s}
\newcommand{\Renyi}{R\'enyi}
\newcommand{\Lovasz}{Lov\'asz}
\newcommand{\Juhasz}{Juh\'asz}
\newcommand{\Bollobas}{Bollob\'as}
\newcommand{\Furedi}{F\"uredi}
\newcommand{\Komlos}{Koml\'os}
\newcommand{\Luczak}{\L uczak}
\newcommand{\Kucera}{Ku\v{c}era}
\newcommand{\Szemeredi}{Szemer\'edi}

\renewcommand{\ae}{\"a}
\renewcommand{\oe}{\"o}
\newcommand{\ue}{\"u}
\newcommand{\Ae}{\"A}
\newcommand{\Oe}{\"O}
\newcommand{\Ue}{\"U}

\title[Linadi]{Carmichaelzahlen}
\author[Amin Coja-Oghlan]{Amin Coja-Oghlan}
\institute[Frankfurt]{JWGUFFM}
\date{}

\begin{document}

\frame[plain]{\titlepage}

\begin{frame}\frametitle{Carmichaelzahlen}
	\begin{block}{Definition}
		Eine zusammengesetzte Zahl $n\geq4$ hei\ss t eine \emph{Carmichaelzahl}, wenn f\ue r alle $1<a<n$ mit $\ggt(a,n)=1$ gilt
		\begin{align*}
			a^{n-1}\equiv1\mod n.
		\end{align*}
	\end{block}
	\begin{block}{Anmerkung}
	\begin{itemize}
		\item Eine Carmichaelzahl ist also eine zusammengesetzte Zahl, die keinen F-Zeugen $a$ mit $\ggt(a,n)=1$ hat.
		\item Die kleinsten Carmichaelzahlen sind
			\begin{align*}
			561,\ 1105,\ 1729,\ 2465,\ 2821.
			\end{align*}
	\end{itemize}	
	\end{block}
\end{frame}

\begin{frame}\frametitle{Carmichaelzahlen}
\begin{block}{Satz}
Eine Carmichaelzahl wird von mindestens drei verschiedenen Primzahlen geteilt.
\end{block}
\begin{overprint}
\onslide<1>
\begin{block}{Beweis}
\begin{itemize}
\item Angenommen $n>3$ hat h\oe chstens zwei Primteiler.
\item \emph{Fall 1:} es gibt eine Primzahl $p$ mit $w_p(n)>1$.
\item Wir schreiben $n$ als
	\begin{align*}
		n&=p^km,&k&=w_p(n),&\ggt(p,m)=1.
	\end{align*}
\item Wenn $m=1$ definiere $a=p+1$.
\item Sonst gibt es nach dem Chinesischen Restsatz $a\in\NN$ mit
	\begin{align*}
		a\equiv p+1\mod p^2,&&a\equiv 1\mod m.
	\end{align*}
\end{itemize}
\end{block}
\onslide<2>
\begin{block}{Beweis (Fortsetzung)}
\begin{itemize}
	\item Jedenfalls gilt $\ggt(a,n)=1$.
	\item Also $a+n\ZZ\in\ZZ_n^\times$.
	\item Nach dem binomischen Lehrsatz gilt
		\begin{align*}
			a^{n-1}&\equiv(p+1)^{n-1}\mod p^2\\
				   &\equiv1+(n-1)p+\sum_{i=2}^{n-1}\binom{n-1}ip^i\mod p^2\\
				   &\equiv1+(n-1)p\mod p^2.
		\end{align*}
\end{itemize}
\end{block}
\onslide<3>
\begin{block}{Beweis (Fortsetzung)}
\begin{itemize}
	\item Angenommen $a^{n-1}\equiv1\mod p^2$.
	\item Dann m\ue\ss te
		\begin{align*}
			0\equiv(n-1)p\mod p^2
		\end{align*}
		gelten, also $p^2\mid(n-1)p$ also $p|n-1$.
	\item Das kann nicht sein, weil $p|n$.
	\item Also ist $a$ ein F-Zeuge mit $\ggt(a,n)=1$.
	\item Somit ist $n$ keine Carmichaelzahl.
\end{itemize}
\end{block}
\onslide<4>
\begin{block}{Beweis (Fortsetzung)}
\begin{itemize}
	\item \emph{Fall 2:} $n=pq$ f\ue r zwei Primzahlen $p>q>1$.
	\item Nach dem Satz vom primitiven Element gibt es $1<g<p$ mit
		\begin{align*}
			\ZZ_p^\times=\cbc{g^i+p\ZZ:i=1,\ldots,p-1}.
		\end{align*}
	\item Nach dem Chinesischen Restsatz gibt es $a\in\NN$ mit
		\begin{align*}
			a\equiv g\mod p,&&a\equiv1\mod q.
		\end{align*}
	\item Weil $p\nmid a$, $q\nmid a$ folgt $\ggt(a,n)=1$.
\end{itemize}
\end{block}
\onslide<5>
\begin{block}{Beweis (Fortsetzung)}
\begin{itemize}
	\item Angenommen $a^{n-1}\equiv1\mod n$.
	\item Dann folgt
		\begin{align*}
			1\equiv a^{n-1}\equiv g^{n-1}\mod p.
		\end{align*}
	\item Weil $g$ primitives Element ist, folgt daraus
		\begin{align*}
			p-1\mid n-1=pq-1=(p-1)q+q-1.
		\end{align*}
	\item Also $p-1\mid q-1$, im Widerspruch zu $p>q$.
	\item Somit ist $a$ ein F-Zeuge und $n$ keine Carmichaelzahl.
\end{itemize}
\end{block}
\end{overprint}
\end{frame}

\begin{frame}\frametitle{Carmichaelzahlen}
\begin{block}{Zusammenfassung}
\begin{itemize}
\item Carmichaelzahlen werden von mindestens drei verschiedenen Primzahlen geteilt.
\item Zum Beweis dieser Tatsache haben wir den Chinesischen Restsatz und den Satz vom primitiven Element ben\oe tigt.
\end{itemize}
\end{block}
\end{frame}


\end{document}
