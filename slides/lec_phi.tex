\documentclass{beamer}
\usepackage{amsmath,graphics}
\usepackage{amssymb}

\usetheme{default}
\usepackage{xcolor}

\definecolor{solarizedBase03}{HTML}{002B36}
\definecolor{solarizedBase02}{HTML}{073642}
\definecolor{solarizedBase01}{HTML}{586e75}
\definecolor{solarizedBase00}{HTML}{657b83}
\definecolor{solarizedBase0}{HTML}{839496}
\definecolor{solarizedBase1}{HTML}{93a1a1}
\definecolor{solarizedBase2}{HTML}{EEE8D5}
\definecolor{solarizedBase3}{HTML}{FDF6E3}
\definecolor{solarizedYellow}{HTML}{B58900}
\definecolor{solarizedOrange}{HTML}{CB4B16}
\definecolor{solarizedRed}{HTML}{DC322F}
\definecolor{solarizedMagenta}{HTML}{D33682}
\definecolor{solarizedViolet}{HTML}{6C71C4}
%\definecolor{solarizedBlue}{HTML}{268BD2}
\definecolor{solarizedBlue}{HTML}{134676}
\definecolor{solarizedCyan}{HTML}{2AA198}
\definecolor{solarizedGreen}{HTML}{859900}
\definecolor{myBlue}{HTML}{162DB0}%{261CA4}
\setbeamercolor*{item}{fg=myBlue}
\setbeamercolor{normal text}{fg=solarizedBase03, bg=solarizedBase3}
\setbeamercolor{alerted text}{fg=myBlue}
\setbeamercolor{example text}{fg=myBlue, bg=solarizedBase3}
\setbeamercolor*{frametitle}{fg=solarizedRed}
\setbeamercolor*{title}{fg=solarizedRed}
\setbeamercolor{block title}{fg=myBlue, bg=solarizedBase3}
\setbeameroption{hide notes}
\setbeamertemplate{note page}[plain]
\beamertemplatenavigationsymbolsempty
\usefonttheme{professionalfonts}
\usefonttheme{serif}

\usepackage{fourier}

\def\vec#1{\mathchoice{\mbox{\boldmath$\displaystyle#1$}}
{\mbox{\boldmath$\textstyle#1$}}
{\mbox{\boldmath$\scriptstyle#1$}}
{\mbox{\boldmath$\scriptscriptstyle#1$}}}
\definecolor{OwnGrey}{rgb}{0.560,0.000,0.000} % #999999
\definecolor{OwnBlue}{rgb}{0.121,0.398,0.711} % #1f64b0
\definecolor{red4}{rgb}{0.5,0,0}
\definecolor{blue4}{rgb}{0,0,0.5}
\definecolor{Blue}{rgb}{0,0,0.66}
\definecolor{LightBlue}{rgb}{0.9,0.9,1}
\definecolor{Green}{rgb}{0,0.5,0}
\definecolor{LightGreen}{rgb}{0.9,1,0.9}
\definecolor{Red}{rgb}{0.9,0,0}
\definecolor{LightRed}{rgb}{1,0.9,0.9}
\definecolor{White}{gray}{1}
\definecolor{Black}{gray}{0}
\definecolor{LightGray}{gray}{0.8}
\definecolor{Orange}{rgb}{0.1,0.2,1}
\setbeamerfont{sidebar right}{size=\scriptsize}
\setbeamercolor{sidebar right}{fg=Black}

\renewcommand{\emph}[1]{{\textcolor{solarizedRed}{\itshape #1}}}

\newcommand\cA{\mathcal A}
\newcommand\cB{\mathcal B}
\newcommand\cC{\mathcal C}
\newcommand\cD{\mathcal D}
\newcommand\cE{\mathcal E}
\newcommand\cF{\mathcal F}
\newcommand\cG{\mathcal G}
\newcommand\cH{\mathcal H}
\newcommand\cI{\mathcal I}
\newcommand\cJ{\mathcal J}
\newcommand\cK{\mathcal K}
\newcommand\cL{\mathcal L}
\newcommand\cM{\mathcal M}
\newcommand\cN{\mathcal N}
\newcommand\cO{\mathcal O}
\newcommand\cP{\mathcal P}
\newcommand\cQ{\mathcal Q}
\newcommand\cR{\mathcal R}
\newcommand\cS{\mathcal S}
\newcommand\cT{\mathcal T}
\newcommand\cU{\mathcal U}
\newcommand\cV{\mathcal V}
\newcommand\cW{\mathcal W}
\newcommand\cX{\mathcal X}
\newcommand\cY{\mathcal Y}
\newcommand\cZ{\mathcal Z}

\newcommand\fA{\mathfrak A}
\newcommand\fB{\mathfrak B}
\newcommand\fC{\mathfrak C}
\newcommand\fD{\mathfrak D}
\newcommand\fE{\mathfrak E}
\newcommand\fF{\mathfrak F}
\newcommand\fG{\mathfrak G}
\newcommand\fH{\mathfrak H}
\newcommand\fI{\mathfrak I}
\newcommand\fJ{\mathfrak J}
\newcommand\fK{\mathfrak K}
\newcommand\fL{\mathfrak L}
\newcommand\fM{\mathfrak M}
\newcommand\fN{\mathfrak N}
\newcommand\fO{\mathfrak O}
\newcommand\fP{\mathfrak P}
\newcommand\fQ{\mathfrak Q}
\newcommand\fR{\mathfrak R}
\newcommand\fS{\mathfrak S}
\newcommand\fT{\mathfrak T}
\newcommand\fU{\mathfrak U}
\newcommand\fV{\mathfrak V}
\newcommand\fW{\mathfrak W}
\newcommand\fX{\mathfrak X}
\newcommand\fY{\mathfrak Y}
\newcommand\fZ{\mathfrak Z}

\newcommand\fa{\mathfrak a}
\newcommand\fb{\mathfrak b}
\newcommand\fc{\mathfrak c}
\newcommand\fd{\mathfrak d}
\newcommand\fe{\mathfrak e}
\newcommand\ff{\mathfrak f}
\newcommand\fg{\mathfrak g}
\newcommand\fh{\mathfrak h}
%\newcommand\fi{\mathfrak i}
\newcommand\fj{\mathfrak j}
\newcommand\fk{\mathfrak k}
\newcommand\fl{\mathfrak l}
\newcommand\fm{\mathfrak m}
\newcommand\fn{\mathfrak n}
\newcommand\fo{\mathfrak o}
\newcommand\fp{\mathfrak p}
\newcommand\fq{\mathfrak q}
\newcommand\fr{\mathfrak r}
\newcommand\fs{\mathfrak s}
\newcommand\ft{\mathfrak t}
\newcommand\fu{\mathfrak u}
\newcommand\fv{\mathfrak v}
\newcommand\fw{\mathfrak w}
\newcommand\fx{\mathfrak x}
\newcommand\fy{\mathfrak y}
\newcommand\fz{\mathfrak z}

\newcommand\vA{\vec A}
\newcommand\vB{\vec B}
\newcommand\vC{\vec C}
\newcommand\vD{\vec D}
\newcommand\vE{\vec E}
\newcommand\vF{\vec F}
\newcommand\vG{\vec G}
\newcommand\vH{\vec H}
\newcommand\vI{\vec I}
\newcommand\vJ{\vec J}
\newcommand\vK{\vec K}
\newcommand\vL{\vec L}
\newcommand\vM{\vec M}
\newcommand\vN{\vec N}
\newcommand\vO{\vec O}
\newcommand\vP{\vec P}
\newcommand\vQ{\vec Q}
\newcommand\vR{\vec R}
\newcommand\vS{\vec S}
\newcommand\vT{\vec T}
\newcommand\vU{\vec U}
\newcommand\vV{\vec V}
\newcommand\vW{\vec W}
\newcommand\vX{\vec X}
\newcommand\vY{\vec Y}
\newcommand\vZ{\vec Z}

\newcommand\va{\vec a}
\newcommand\vb{\vec b}
\newcommand\vc{\vec c}
\newcommand\vd{\vec d}
\newcommand\ve{\vec e}
\newcommand\vf{\vec f}
\newcommand\vg{\vec g}
\newcommand\vh{\vec h}
\newcommand\vi{\vec i}
\newcommand\vj{\vec j}
\newcommand\vk{\vec k}
\newcommand\vl{\vec l}
\newcommand\vm{\vec m}
\newcommand\vn{\vec n}
\newcommand\vo{\vec o}
\newcommand\vp{\vec p}
\newcommand\vq{\vec q}
\newcommand\vr{\vec r}
\newcommand\vs{\vec s}
\newcommand\vt{\vec t}
\newcommand\vu{\vec u}
\newcommand\vv{\vec v}
\newcommand\vw{\vec w}
\newcommand\vx{\vec x}
\newcommand\vy{\vec y}
\newcommand\vz{\vec z}

\newcommand\NN{\mathbb N}
\newcommand\ZZ{\mathbb Z}
\newcommand\PP{\mathbb P}
\newcommand\QQ{\mathbb Q}
\newcommand\RR{\mathbb R}
\newcommand\CC{\mathbb C}

\newcommand{\pr}{\mathrm{P}}
\newcommand{\Vol}{\mathrm{vol}}
\newcommand\norm[1]{\left\|{#1}\right\|} 
\newcommand\sign{\mathrm{sign}}
\newcommand{\eps}{\varepsilon}
\newcommand{\abs}[1]{\left|#1\right|}
\newcommand\bc[1]{\left({#1}\right)} 
\newcommand\cbc[1]{\left\{{#1}\right\}} 
\newcommand\bcfr[2]{\bc{\frac{#1}{#2}}} 
\newcommand{\bck}[1]{\left\langle{#1}\right\rangle} 
\newcommand\brk[1]{\left\lbrack{#1}\right\rbrack} 
\newcommand\scal[2]{\bck{{#1},{#2}}} 
\newcommand{\vecone}{\mathbb{1}}
\newcommand{\tensor}{\otimes}
\newcommand{\diag}{\mathrm{diag}}
\newcommand{\ggt}{\mathrm{ggT}}
\newcommand{\kgv}{\mathrm{kgV}}

\newcommand{\Karonski}{Karo\'nski}
\newcommand{\Erdos}{Erd\H{o}s}
\newcommand{\Renyi}{R\'enyi}
\newcommand{\Lovasz}{Lov\'asz}
\newcommand{\Juhasz}{Juh\'asz}
\newcommand{\Bollobas}{Bollob\'as}
\newcommand{\Furedi}{F\"uredi}
\newcommand{\Komlos}{Koml\'os}
\newcommand{\Luczak}{\L uczak}
\newcommand{\Kucera}{Ku\v{c}era}
\newcommand{\Szemeredi}{Szemer\'edi}

\renewcommand{\ae}{\"a}
\renewcommand{\oe}{\"o}
\newcommand{\ue}{\"u}
\newcommand{\Ae}{\"A}
\newcommand{\Oe}{\"O}
\newcommand{\Ue}{\"U}

\title[Linadi]{Die $\phi$-Funktion}
\author[Amin Coja-Oghlan]{Amin Coja-Oghlan}
\institute[Frankfurt]{JWGUFFM}
\date{}

\begin{document}

\frame[plain]{\titlepage}

\begin{frame}\frametitle{Multiplikative Inverse in modularer Arthmetik}
	\begin{block}{Proposition}
		Sei $m\in\NN$ und $x\in\ZZ$.
		Es gibt ein $y\in\ZZ$ mit
		\begin{align*}
			x\cdot y\equiv 1\mod m
		\end{align*}
		genau dann, wenn $\ggt(x,m)=1$.
	\end{block}
	\begin{overprint}
		\onslide<1>
		\begin{block}{Beweis}
			\begin{itemize}
				\item Angenommen $\ggt(x,m)=1$.
				\item Dann gibt es $y,z\in\ZZ$ mit $xy+zm=1$.
				\item Also gilt $xy\equiv 1\mod m$.
			\end{itemize} 
		\end{block}
		\onslide<2>
		\begin{block}{Beweis (Fortsetzung)}
			\begin{itemize}
				\item Angeommen es gibt $y\in\ZZ$ mit $xy\equiv 1\mod m$.
				\item Dann gibt es $q\in\ZZ$ mit $xy=qm+1$.
				\item Also $xy-qm=1$.
				\item Daraus folgt $\ggt(x,m)=1$.
			\end{itemize} 
		\end{block}
		\onslide<3>
		\begin{block}{}
			\begin{itemize}
				\item Wir nennen $y$ das \emph{Inverse} von $x$ modulo $m$.
				\item Das inverse ist modulo $m$ eindeutig bestimmt.
				\item Wenn $\ggt(x,m)=1$, hei\ss t $x$ eine \emph{Einheit} modulo $m$.
			\end{itemize}
		\end{block}
	\end{overprint}
\end{frame}

\begin{frame}\frametitle{Multiplikative Inverse in modularer Arthmetik}
\begin{block}{Beispiel}
\begin{itemize}
	\item \emph{Inverses von $x=4$ modulo $m=11$}
	\item Erweiterter Euklidischer Algorithmus:
		\begin{align*}
			11&=2\cdot 4+3&\Rightarrow&&3=11-2\cdot 4\\
			4&=1\cdot 3+1&\Rightarrow&&1=4-1\cdot 3\\
			 &&\Rightarrow&&1=4-1\cdot\bc{11-2\cdot 4}=3\cdot 4-11
		\end{align*}
	\item Das Inverse ist also $y=3$.
	\item \emph{Probe:} $xy=4\cdot 3=12\equiv 1\mod 11$.
\end{itemize}
\end{block}
\end{frame}

\begin{frame}\frametitle{Die $\phi$-Funktion}
	\begin{block}{Definition}
		F\ue r eine Zahl $m\in\NN$ definieren wir
		\begin{align*}
			\phi(m)&=\abs{\cbc{x\in\{1,2,3,\ldots,m\}:\ggt(x,m)=1}}.
		\end{align*}
		Diese Funktion $\phi:\NN\to\NN$ hei\ss t \emph{Eulersche $\phi$-Funktion} (sprich: ``Phi-Funktion'').
	\end{block}
\end{frame}

\begin{frame}\frametitle{Die $\phi$-Funktion}
	\begin{block}{Beispiel}
		\begin{itemize}
			\item $\phi(1)=1$\hfill  $\cbc{1}$
			\item $\phi(2)=1$\hfill  $\cbc{1}$
			\item $\phi(3)=2$\hfill  $\cbc{1,2}$
			\item $\phi(4)=2$\hfill  $\cbc{1,3}$
			\item $\phi(5)=4$\hfill  $\cbc{1,2,3,4}$
			\item $\phi(6)=2$\hfill  $\cbc{1,5}$
			\item $\phi(7)=6$\hfill  $\cbc{1,2,3,4,5,6}$
			\item $\phi(8)=4$\hfill  $\cbc{1,3,5,7}$
		\end{itemize}
	\end{block}
\end{frame}

\begin{frame}\frametitle{Die $\phi$-Funktion}
	\begin{block}{Lemma}
		F\ue r Primzahlen $p\geq2$ und $k\in\NN$ gilt $\phi(p^k)=p^k-p^{k-1}$.
	\end{block}
	\begin{block}{Beweis}
		\begin{itemize}
			\item F\ue r $1\leq x\leq p^k$ gilt $\ggt(x,p^k)>1$ genau dann, wenn $p|x$.
			\item Jede $p$-te Zahl ist durch $p$ teilbar.
			\item Also gibt es genau $p^{k-1}$ Zahlen $1\leq x\leq p^k$ mit $\ggt(x,p^k)>1$.
		\end{itemize}
	\end{block}
\end{frame}

\begin{frame}\frametitle{Die $\phi$-Funktion}
	\begin{block}{Lemma}
		Wenn $m,n$ teilerfremd sind, gilt $\phi(m\cdot n)=\phi(m)\cdot\phi(n)$.
	\end{block}
	\begin{overprint}
		\onslide<1>
		\begin{block}{Beweis}
			\begin{itemize}
				\item Angenommen $1\leq x\leq m$ und $1\leq y\leq n$ erf\ue llen $\ggt(x,m)=\ggt(y,n)=1$.
				\item Nach dem chinesischen Restsatz gibt es $1\leq z\leq mn$ mit
					\begin{align*}
						z\equiv x\mod m&&\mbox{und}&&z\equiv y\mod n.
					\end{align*}
				\item Weil $\ggt(z,m)=\ggt(x,m)=1$ und $\ggt(z,n)=\ggt(y,n)=1$ folgt $\ggt(z,mn)=1$.
				\item Also $\phi(m\cdot n)\geq\phi(m)\phi(n)$.
			\end{itemize}
		\end{block}
		\onslide<2>
		\begin{block}{Beweis (Fortsetzung)}
			\begin{itemize}
				\item Angenommen $\ggt(z,mn)=1$.
				\item Dann gilt $\ggt(z,m)=\ggt(z,n)=1$.
				\item Nach dem chinesischen Restsatz ist die Abbildung $z\mapsto(x,y)$ mit
					\begin{align*}
						z\equiv x\mod m&&\mbox{und}&&z\equiv y\mod n
					\end{align*}
					bijektiv.
				\item Also gilt $\phi(m\cdot n)\leq\phi(m)\phi(n)$.
			\end{itemize}
		\end{block}
		\onslide<3>
		\begin{block}{Korollar}
			F\ue r jede Zahl $n\in\NN$ gilt
			\begin{align*}
				\phi(n)=\prod_{\substack{p\in\PP\\2\leq p|n}}p^{w_p(n)}-p^{w_p(n)-1}
			\end{align*}
		\end{block}
		{\itshape Wenn wir die Primfaktorisierung von $n$ kennen, ist $\phi(n)$ leicht zu bestimmen.}
	\end{overprint}
\end{frame}

\begin{frame}\frametitle{Die $\phi$-Funktion}
	\begin{block}{Beispiel}
		\begin{itemize}
			\item $n=12=2^2\cdot 3$
			\item $\phi(n)=(2^2-2^1)\cdot(3^1-3^0)=2\cdot2=4$
			\item Einheiten modulo 12: $1,5,7,11$
		\end{itemize}
	\end{block}
\end{frame}

\begin{frame}\frametitle{Zusammenfassung}
	\begin{itemize}
		\item Inverse modulo $m$ existieren genau f\ue r die zu $m$ teilerfremden Zahlen.
		\item Die $\phi$-Funktion z\ae hlt die zu $m$ teilerfremden Zahlen zwischen $1$ und $m$.
		\item Gegeben die Primfaktorisierung von $m$, kann man $\phi(m)$ leicht ausrechnen.
	\end{itemize}
\end{frame}


\end{document}
