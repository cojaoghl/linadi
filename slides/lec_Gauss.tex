\documentclass{beamer}
\usepackage{amsmath,graphics}
\usepackage{amssymb}

\usetheme{default}
\usepackage{xcolor}

\definecolor{solarizedBase03}{HTML}{002B36}
\definecolor{solarizedBase02}{HTML}{073642}
\definecolor{solarizedBase01}{HTML}{586e75}
\definecolor{solarizedBase00}{HTML}{657b83}
\definecolor{solarizedBase0}{HTML}{839496}
\definecolor{solarizedBase1}{HTML}{93a1a1}
\definecolor{solarizedBase2}{HTML}{EEE8D5}
\definecolor{solarizedBase3}{HTML}{FDF6E3}
\definecolor{solarizedYellow}{HTML}{B58900}
\definecolor{solarizedOrange}{HTML}{CB4B16}
\definecolor{solarizedRed}{HTML}{DC322F}
\definecolor{solarizedMagenta}{HTML}{D33682}
\definecolor{solarizedViolet}{HTML}{6C71C4}
%\definecolor{solarizedBlue}{HTML}{268BD2}
\definecolor{solarizedBlue}{HTML}{134676}
\definecolor{solarizedCyan}{HTML}{2AA198}
\definecolor{solarizedGreen}{HTML}{859900}
\definecolor{myBlue}{HTML}{162DB0}%{261CA4}
\setbeamercolor*{item}{fg=myBlue}
\setbeamercolor{normal text}{fg=solarizedBase03, bg=solarizedBase3}
\setbeamercolor{alerted text}{fg=myBlue}
\setbeamercolor{example text}{fg=myBlue, bg=solarizedBase3}
\setbeamercolor*{frametitle}{fg=solarizedRed}
\setbeamercolor*{title}{fg=solarizedRed}
\setbeamercolor{block title}{fg=myBlue, bg=solarizedBase3}
\setbeameroption{hide notes}
\setbeamertemplate{note page}[plain]
\beamertemplatenavigationsymbolsempty
\usefonttheme{professionalfonts}
\usefonttheme{serif}

\usepackage{fourier}

\def\vec#1{\mathchoice{\mbox{\boldmath$\displaystyle#1$}}
{\mbox{\boldmath$\textstyle#1$}}
{\mbox{\boldmath$\scriptstyle#1$}}
{\mbox{\boldmath$\scriptscriptstyle#1$}}}
\definecolor{OwnGrey}{rgb}{0.560,0.000,0.000} % #999999
\definecolor{OwnBlue}{rgb}{0.121,0.398,0.711} % #1f64b0
\definecolor{red4}{rgb}{0.5,0,0}
\definecolor{blue4}{rgb}{0,0,0.5}
\definecolor{Blue}{rgb}{0,0,0.66}
\definecolor{LightBlue}{rgb}{0.9,0.9,1}
\definecolor{Green}{rgb}{0,0.5,0}
\definecolor{LightGreen}{rgb}{0.9,1,0.9}
\definecolor{Red}{rgb}{0.9,0,0}
\definecolor{LightRed}{rgb}{1,0.9,0.9}
\definecolor{White}{gray}{1}
\definecolor{Black}{gray}{0}
\definecolor{LightGray}{gray}{0.8}
\definecolor{Orange}{rgb}{0.1,0.2,1}
\setbeamerfont{sidebar right}{size=\scriptsize}
\setbeamercolor{sidebar right}{fg=Black}

\renewcommand{\emph}[1]{{\textcolor{solarizedRed}{\itshape #1}}}

\newcommand\cA{\mathcal A}
\newcommand\cB{\mathcal B}
\newcommand\cC{\mathcal C}
\newcommand\cD{\mathcal D}
\newcommand\cE{\mathcal E}
\newcommand\cF{\mathcal F}
\newcommand\cG{\mathcal G}
\newcommand\cH{\mathcal H}
\newcommand\cI{\mathcal I}
\newcommand\cJ{\mathcal J}
\newcommand\cK{\mathcal K}
\newcommand\cL{\mathcal L}
\newcommand\cM{\mathcal M}
\newcommand\cN{\mathcal N}
\newcommand\cO{\mathcal O}
\newcommand\cP{\mathcal P}
\newcommand\cQ{\mathcal Q}
\newcommand\cR{\mathcal R}
\newcommand\cS{\mathcal S}
\newcommand\cT{\mathcal T}
\newcommand\cU{\mathcal U}
\newcommand\cV{\mathcal V}
\newcommand\cW{\mathcal W}
\newcommand\cX{\mathcal X}
\newcommand\cY{\mathcal Y}
\newcommand\cZ{\mathcal Z}

\newcommand\fA{\mathfrak A}
\newcommand\fB{\mathfrak B}
\newcommand\fC{\mathfrak C}
\newcommand\fD{\mathfrak D}
\newcommand\fE{\mathfrak E}
\newcommand\fF{\mathfrak F}
\newcommand\fG{\mathfrak G}
\newcommand\fH{\mathfrak H}
\newcommand\fI{\mathfrak I}
\newcommand\fJ{\mathfrak J}
\newcommand\fK{\mathfrak K}
\newcommand\fL{\mathfrak L}
\newcommand\fM{\mathfrak M}
\newcommand\fN{\mathfrak N}
\newcommand\fO{\mathfrak O}
\newcommand\fP{\mathfrak P}
\newcommand\fQ{\mathfrak Q}
\newcommand\fR{\mathfrak R}
\newcommand\fS{\mathfrak S}
\newcommand\fT{\mathfrak T}
\newcommand\fU{\mathfrak U}
\newcommand\fV{\mathfrak V}
\newcommand\fW{\mathfrak W}
\newcommand\fX{\mathfrak X}
\newcommand\fY{\mathfrak Y}
\newcommand\fZ{\mathfrak Z}

\newcommand\fa{\mathfrak a}
\newcommand\fb{\mathfrak b}
\newcommand\fc{\mathfrak c}
\newcommand\fd{\mathfrak d}
\newcommand\fe{\mathfrak e}
\newcommand\ff{\mathfrak f}
\newcommand\fg{\mathfrak g}
\newcommand\fh{\mathfrak h}
%\newcommand\fi{\mathfrak i}
\newcommand\fj{\mathfrak j}
\newcommand\fk{\mathfrak k}
\newcommand\fl{\mathfrak l}
\newcommand\fm{\mathfrak m}
\newcommand\fn{\mathfrak n}
\newcommand\fo{\mathfrak o}
\newcommand\fp{\mathfrak p}
\newcommand\fq{\mathfrak q}
\newcommand\fr{\mathfrak r}
\newcommand\fs{\mathfrak s}
\newcommand\ft{\mathfrak t}
\newcommand\fu{\mathfrak u}
\newcommand\fv{\mathfrak v}
\newcommand\fw{\mathfrak w}
\newcommand\fx{\mathfrak x}
\newcommand\fy{\mathfrak y}
\newcommand\fz{\mathfrak z}

\newcommand\vA{\vec A}
\newcommand\vB{\vec B}
\newcommand\vC{\vec C}
\newcommand\vD{\vec D}
\newcommand\vE{\vec E}
\newcommand\vF{\vec F}
\newcommand\vG{\vec G}
\newcommand\vH{\vec H}
\newcommand\vI{\vec I}
\newcommand\vJ{\vec J}
\newcommand\vK{\vec K}
\newcommand\vL{\vec L}
\newcommand\vM{\vec M}
\newcommand\vN{\vec N}
\newcommand\vO{\vec O}
\newcommand\vP{\vec P}
\newcommand\vQ{\vec Q}
\newcommand\vR{\vec R}
\newcommand\vS{\vec S}
\newcommand\vT{\vec T}
\newcommand\vU{\vec U}
\newcommand\vV{\vec V}
\newcommand\vW{\vec W}
\newcommand\vX{\vec X}
\newcommand\vY{\vec Y}
\newcommand\vZ{\vec Z}

\newcommand\va{\vec a}
\newcommand\vb{\vec b}
\newcommand\vc{\vec c}
\newcommand\vd{\vec d}
\newcommand\ve{\vec e}
\newcommand\vf{\vec f}
\newcommand\vg{\vec g}
\newcommand\vh{\vec h}
\newcommand\vi{\vec i}
\newcommand\vj{\vec j}
\newcommand\vk{\vec k}
\newcommand\vl{\vec l}
\newcommand\vm{\vec m}
\newcommand\vn{\vec n}
\newcommand\vo{\vec o}
\newcommand\vp{\vec p}
\newcommand\vq{\vec q}
\newcommand\vr{\vec r}
\newcommand\vs{\vec s}
\newcommand\vt{\vec t}
\newcommand\vu{\vec u}
\newcommand\vv{\vec v}
\newcommand\vw{\vec w}
\newcommand\vx{\vec x}
\newcommand\vy{\vec y}
\newcommand\vz{\vec z}

\renewcommand\AA{\mathbb A}
\newcommand\NN{\mathbb N}
\newcommand\ZZ{\mathbb Z}
\newcommand\PP{\mathbb P}
\newcommand\QQ{\mathbb Q}
\newcommand\RR{\mathbb R}
\renewcommand\SS{\mathbb S}
\newcommand\CC{\mathbb C}

\newcommand{\ord}{\mathrm{ord}}
\newcommand{\id}{\mathrm{id}}
\newcommand{\pr}{\mathrm{P}}
\newcommand{\Vol}{\mathrm{vol}}
\newcommand\norm[1]{\left\|{#1}\right\|} 
\newcommand\sign{\mathrm{sign}}
\newcommand{\eps}{\varepsilon}
\newcommand{\abs}[1]{\left|#1\right|}
\newcommand\bc[1]{\left({#1}\right)} 
\newcommand\cbc[1]{\left\{{#1}\right\}} 
\newcommand\bcfr[2]{\bc{\frac{#1}{#2}}} 
\newcommand{\bck}[1]{\left\langle{#1}\right\rangle} 
\newcommand\brk[1]{\left\lbrack{#1}\right\rbrack} 
\newcommand\scal[2]{\bck{{#1},{#2}}} 
\newcommand{\vecone}{\mathbb{1}}
\newcommand{\tensor}{\otimes}
\newcommand{\diag}{\mathrm{diag}}
\newcommand{\ggt}{\mathrm{ggT}}
\newcommand{\kgv}{\mathrm{kgV}}
\newcommand{\trans}{\top}

\newcommand{\Karonski}{Karo\'nski}
\newcommand{\Erdos}{Erd\H{o}s}
\newcommand{\Renyi}{R\'enyi}
\newcommand{\Lovasz}{Lov\'asz}
\newcommand{\Juhasz}{Juh\'asz}
\newcommand{\Bollobas}{Bollob\'as}
\newcommand{\Furedi}{F\"uredi}
\newcommand{\Komlos}{Koml\'os}
\newcommand{\Luczak}{\L uczak}
\newcommand{\Kucera}{Ku\v{c}era}
\newcommand{\Szemeredi}{Szemer\'edi}

\renewcommand{\ae}{\"a}
\renewcommand{\oe}{\"o}
\newcommand{\ue}{\"u}
\newcommand{\Ae}{\"A}
\newcommand{\Oe}{\"O}
\newcommand{\Ue}{\"U}

\newcommand{\im}{\mathrm{im}}

\newcommand{\mytitle}{Lineare Gleichungssysteme}

\title[Linadi]{\mytitle}
\author[Amin Coja-Oghlan]{Amin Coja-Oghlan}
\institute[Frankfurt]{JWGUFFM}
\date{}

\begin{document}

\frame[plain]{\titlepage}

\begin{frame}\frametitle{\mytitle}
	\begin{block}{Zeilenumformungen}
	\begin{itemize}
	\item F\ue r eine $m\times n$-Matrix $A=(a_{ij})$ f\ue hren wir einige Operationen ein.
	Das Ergebnis ist jeweils eine $m\times n$-Matrix $B=(b_{ij})$.
	\item \alert{Zeilenvertauschung:} wir erhalten $B$  durch Vertauschen von zwei Zeilen von $A$; d.h.\ es gibt $1\leq h<k\leq m$ mit
	\begin{align*}
		b_{ij}&=\begin{cases}
			a_{ij}&\mbox{ falls $i\neq h$ und $i\neq k$}\\
			a_{hj}&\mbox{ falls $i=k$}\\
			a_{kj}&\mbox{ falls $i=h$}
		\end{cases}
	\end{align*}
	\end{itemize}
	\end{block}
\end{frame}

\begin{frame}\frametitle{\mytitle}
	\begin{block}{Zeilenumformungen}
	\begin{itemize}
	\item \alert{Skalierung:} wir multiplizieren die $h$-te Zeile von $A$ mit $c\in\RR\setminus\cbc 0$; d.h.\ 
	\begin{align*}
		b_{ij}&=\begin{cases}
			a_{ij}&\mbox{ falls $i\neq h$}\\
			c\cdot a_{hj}&\mbox{ falls $i=h$}
		\end{cases}
	\end{align*}
	\end{itemize}
	\end{block}
\end{frame}

\begin{frame}\frametitle{\mytitle}
	\begin{block}{Zeilenumformungen}
	\begin{itemize}
		\item \alert{Pivot:} wir addieren ein Vielfaches einer Zeile zu einer anderen; d.h.\ es gibt verschiedene $h,k\in\{1,\ldots,m\}$ und eine Zahl $c\in\RR$, so da\ss\
	\begin{align*}
		b_{ij}&=\begin{cases}
			a_{ij}&\mbox{ falls $i\neq k$}\\
			c\cdot a_{hj}+a_{kj}&\mbox{ falls $i=k$}
		\end{cases}
	\end{align*}
	\end{itemize}
	\end{block}
\end{frame}

\begin{frame}\frametitle{\mytitle}
	\begin{block}{Definition}
		Zwei $m\times n$-Matrizen $A,B$ hei\ss en \emph{zeilen\ae quivalent}, falls $B$ aus $A$ durch eine oder mehrere Zeilenumformungen entsteht.
	\end{block}
\end{frame}

\begin{frame}\frametitle{\mytitle}
	\begin{block}{Proposition}
		Zwei $m\times n$-Matrizen $A,B$ sind genau dann zeilen\ae quivalent, wenn $$\ker(A)=\ker(B).$$
	\end{block}
\end{frame}

\begin{frame}\frametitle{\mytitle}
	\begin{block}{Beispiel}
	\begin{itemize}
	\item Wir beginnen mit der $3\times 3$-Matrix 
		\begin{align*}
			A&=\begin{pmatrix}1&-1&2\\1&7&0\\3&-4&6\end{pmatrix}
		\end{align*}
		und f\ue hren Zeilenumformungen aus, um die Matrix zu vereinfachen.
	\item \alert{Pivot:} addiere das $-1$-fache der ersten Zeile zur zweiten Zeile:
\begin{align*}
		\begin{pmatrix}
1&-1&2\\0&8&-2\\3&-4&6
		\end{pmatrix}
		\end{align*}
	\end{itemize}
	\end{block}
\end{frame}

\begin{frame}\frametitle{\mytitle}
	\begin{block}{Beispiel}
	\begin{itemize}
	\item \alert{Pivot:} addiere das $-3$-fache der ersten Zeile zur letzten:
		\begin{align*}
		\begin{pmatrix}
			1&-1&2\\0&8&-2\\0&-1&0
		\end{pmatrix}
		\end{align*}
	\item \alert{Zeilenvertauschung:} tausche die zweite und die dritte Zeile:
		\begin{align*}
		\begin{pmatrix}
			1&-1&2\\0&-1&0\\0&8&-2
		\end{pmatrix}
		\end{align*}
	\end{itemize}
	\end{block}
\end{frame}

\begin{frame}\frametitle{\mytitle}
	\begin{block}{Beispiel}
	\begin{itemize}
	\item \alert{Skalierung:} multipliziere die zweite Zeile mit $-1$:
		\begin{align*}
		\begin{pmatrix}
			1&-1&2\\0&1&0\\0&8&-2
		\end{pmatrix}
		\end{align*}
	\item \alert{Pivot:} addiere das $-8$-fache der zweiten Zeile zur dritten Zeile:
		\begin{align*}
		\begin{pmatrix}
			1&-1&2\\0&1&0\\0&0&-2
		\end{pmatrix}
		\end{align*}
	\end{itemize}
	\end{block}
\end{frame}

\begin{frame}\frametitle{\mytitle}
	\begin{block}{Beispiel}
	\begin{itemize}
		\item \alert{Skalierung:} multipliziere die letzte Zeile mit $-1/2$:
\begin{align*}
	B&=\begin{pmatrix} 1&-1&2\\0&1&0\\0&0&1 \end{pmatrix}
		\end{align*}
	\end{itemize}
	\end{block}
\end{frame}

\begin{frame}\frametitle{\mytitle}
	\begin{block}{Beispiel}
	\begin{itemize}
	\item Den Kern der Matrix $B$ kann man direkt ablesen:
		\begin{align*}
			\ker(B)=\cbc 0
		\end{align*}
	\item Denn angenommen $u\in\ker(B)$; dann l\oe st $u$ das lineare Gleichungssystem
		\begin{align*}
			Bu&=0&&\Leftrightarrow&
			  &\begin{pmatrix} 1\cdot u_1-1\cdot u_2+2u_3\\ u_2\\ u_3
			\end{pmatrix}=\begin{pmatrix} 0\\0\\0 \end{pmatrix}\\
			  &&&\Leftrightarrow&& u_3=0\mbox{ und }u_2=0\mbox{ und }u_1-u_2+2u_3=0
		\end{align*}
	\item Weil $A$ und $B$ zeilen\ae quivalent sind, folgt $\ker(A)=\cbc 0$.
	\end{itemize}
	\end{block}
\end{frame}

\begin{frame}\frametitle{\mytitle}
	\begin{block}{Zeilenstufenform}
		Eine $m\times n$-Matrix $A=(a_{ij})$ ist in \emph{Zeilenstufenform}, wenn es ein $k\in\{0,\ldots,m\}$ und Zahlen $h_1,\ldots,h_k\in\{1,\ldots,n\}$ gibt, so da\ss
	\begin{itemize}
		\item $h_1<h_2<\cdots<h_k$,
		\item f\ue r alle $i\in\{1,\ldots,k\}$ gilt $a_{ih_i}\neq0$,
		\item f\ue r alle $i\in\{1,\ldots,k\}$ und alle $1\leq j<h_i$ gilt $a_{ij}=0$,
		\item f\ue r alle $k<i\leq m$ und alle $1\leq j\leq n$ gilt $a_{ij}=0$.
	\end{itemize}
	\emph{Beispiel:}
	\begin{align*}
	\begin{pmatrix}
		-1&2&3&4&0\\0&0&5&-6&-3\\0&0&0&7&0\\0&0&0&0&0\\0&0&0&0&0
	\end{pmatrix}&&
	k=3,\ h_1=1,\ h_2=3,\ h_3=4
	\end{align*}
	\end{block}
\end{frame}

\begin{frame}\frametitle{\mytitle}
	\begin{block}{Das Gau\ss sche Eliminationsverfahren}
	\begin{itemize}
	\item Sei $A$ eine $m\times n$-Matrix.
	\item \emph{Ziel:} bestimme eine zeilen\ae quivalente Matrix, die in Zeilenstufenform ist.
	\item Beginne mit $h=1$ und setze $k=0$.
	\item Wenn es keine Zeile $i>k$ gibt mit $a_{ih}\neq0$, erh\oe he $h$; wenn $h>n$, stoppe. Sonst wiederhole.
	\item Wenn es eine Zeile $i>k$ gibt, so da\ss\ $a_{ih}\neq0$, addiere Vielfache dieser Zeile zu den anderen Zeilen $i'>k$, $i'\neq i$, so da\ss\ in diesen Zeilen der $h$-te Eintrag Null wird.
	\item Erh\oe he anschlie\ss end $h$ und $k$ und vertausche die $i$-te und die $k$-te Zeile.
	\item Wenn $h>n$ oder $k=m$, stoppe. Sonst wiederhole.
	\end{itemize}
	\end{block}
\end{frame}

\begin{frame}\frametitle{\mytitle}
	\begin{block}{Beispiel}
	\begin{itemize}
		\item Sei
			\begin{align*}
				A&=\begin{pmatrix}
					-1&1&1\\2&2&-4\\-3&-3&6
				\end{pmatrix}
			\end{align*}
		\item Wir addieren das $2$-fache der ersten Zeile zur zweiten:
\begin{align*}
				\begin{pmatrix}
					-1&1&1\\0&4&-2\\-3&-3&6
				\end{pmatrix}
			\end{align*}
	\end{itemize}
	\end{block}
\end{frame}

\begin{frame}\frametitle{\mytitle}
	\begin{block}{Beispiel}
	\begin{itemize}
		\item Wir addieren das $-3$-fache der ersten Zeile zur dritten:
\begin{align*}
				\begin{pmatrix}
					-1&1&1\\0&4&-2\\0&-6&3
				\end{pmatrix}
			\end{align*}
	\end{itemize}
	\end{block}
\end{frame}

\begin{frame}\frametitle{\mytitle}
	\begin{block}{Beispiel}
	\begin{itemize}
		\item Wir addieren das $3/2$-fache der zweiten Zeile zur dritten:
\begin{align*}
				\begin{pmatrix}
					-1&1&1\\0&4&-2\\0&0&0
				\end{pmatrix}
			\end{align*}
		\item Die Matrix ist jetzt in Zeilenstufenform.
	\end{itemize}
	\end{block}
\end{frame}

\begin{frame}\frametitle{\mytitle}
	\begin{block}{Beispiel}
	\begin{itemize}
		\item Betrachte die Matrix
			\begin{align*}
				A&=\begin{pmatrix}
					0&1&1&1\\1&1&0&1\\1&0&1&1
				\end{pmatrix}
			\end{align*}
		\item Wir vertauschen zun\ae chst die erste und die zweite Zeile:
			\begin{align*}
			\begin{pmatrix}
					1&1&0&1\\0&1&1&1\\1&0&1&1
				\end{pmatrix}
			\end{align*}
	\end{itemize}
	\end{block}
\end{frame}

\begin{frame}\frametitle{\mytitle}
	\begin{block}{Beispiel}
	\begin{itemize}
		\item Jetzt addiere das $-1$-fache der ersten Zeile zur letzten:
			\begin{align*}
			\begin{pmatrix}
					1&1&0&1\\0&1&1&1\\0&-1&1&0
				\end{pmatrix}
			\end{align*}
		\item Als n\ae chstes addiere die zweite Zeile zur letzten:
			\begin{align*}
			\begin{pmatrix}
					1&1&0&1\\0&1&1&1\\0&0&2&1
				\end{pmatrix}
			\end{align*}
		\item Die Matrix ist jetzt in Zeilenstufenform.
	\end{itemize}
	\end{block}
\end{frame}

\begin{frame}\frametitle{\mytitle}
	\begin{block}{Homogene lineare Gleichungssysteme}
	\begin{itemize}
		\item Mit dem Gau\ss schen Eliminationsverfahren k\oe nnen wir den Kern einer Matrix bestimmen.
		\item Anders formuliert: wir k\oe nnen die L\oe sungen des homogenen linearen Gleichungssystems $Au=0$ oder, ausgeschrieben,
			\begin{align*}
				a_{11}u_1+\cdots+a_{1n}u_n&=0\\
										  &\vdots\\
				a_{m1}u_1+\cdots+a_{mn}u_n&=0
			\end{align*}
			berechnen.
	\end{itemize}
	\end{block}
\end{frame}

\begin{frame}\frametitle{\mytitle}
	\begin{block}{Homogene lineare Gleichungssysteme}
	\begin{itemize}
		\item Dazu bringen wir die Matrix $A$ zun\ae chst in Zeilenstufenform.
		\item Sei die $m\times n$-Matrix $B=(b_{ij})$ das Ergebnis.
		\item Seien ferner $k$, $h_1,\ldots,h_k$ wie in der Definition der Zeilenstufenform.
		\item Wir w\ae hlen Werte $u_i\in\RR$ f\ue r $i\not\in\{h_1,\ldots,h_k\}$ beliebig aus.
		\item Dann bestimmen wir r\ue ckw\ae rts f\ue r $\ell=k,\ldots,1$ die restlichen Eintr\ae ge $u_{h_\ell}$:
			\begin{align*}
				u_{h_\ell}=-\frac{1}{b_{\ell\, h_\ell}}\sum^n_{j=h_{\ell}+1}b_{\ell j}u_j.
			\end{align*}
	\end{itemize}
	\end{block}
\end{frame}

\begin{frame}\frametitle{\mytitle}
	\begin{block}{Beispiel}
	\begin{itemize}
		\item Betrachte die $3\times 4$-Matrix
			\begin{align*}
				A&=\begin{pmatrix}
					0&1&1&1\\1&1&0&1\\1&0&1&1
				\end{pmatrix},
			\end{align*}
		die wir bereits in Zeilenstufenform gebracht haben:
			\begin{align*}
				B&=\begin{pmatrix}
					1&1&0&1\\0&1&1&1\\0&0&2&1
				\end{pmatrix}
			\end{align*}
		\item Dabei ist $k=3$, $h_1=1$, $h_2=2$, $h_3=3$.
	\end{itemize}
	\end{block}
\end{frame}

\begin{frame}\frametitle{\mytitle}
	\vspace{-6mm}
	\hfill$B=\begin{pmatrix} 1&1&0&1\\0&1&1&1\\0&0&2&1 \end{pmatrix}$
	\begin{block}{Beispiel}
	\begin{itemize}
		\item Wir d\ue rfen also den Wert von $u_4$ frei w\ae hlen.
		\item Anschlie\ss end bestimmen wir den Wert von $u_3$:
			\begin{align*}
				u_3=-\frac{1}{2}\sum_{\ell>3}b_{3\,\ell}u_\ell=-\frac{1}{2}\cdot 1\cdot u_4=-u_4/2.
			\end{align*}
		\item Hernach bestimmen wir $u_2$:
			\begin{align*}
				u_2&=-\frac11\sum_{\ell>2}b_{2,\,\ell} u_\ell
				=-1\cdot u_3-1\cdot u_4=-u_4/2.
			\end{align*}
		\item Zuletzt kommt $u_1$:
			\begin{align*}
				u_1&=-\frac{1}{1}\sum_{\ell>1}b_{1,\,\ell}u_\ell=-1\cdot u_2-0\cdot u_3-1\cdot u_4=-u_4/2.
			\end{align*}
	\end{itemize}
	\end{block}
\end{frame}

\begin{frame}\frametitle{\mytitle}
	\begin{block}{Beispiel}
	\begin{itemize}
		\item Wir erhalten also
			\begin{align*}
				\ker(A)&=\ker(B)=\cbc{\begin{pmatrix} -u_4/2\\-u_4/2\\-u_4/2\\u_4 \end{pmatrix}:u_4\in\RR}
			\end{align*}
		\item Ein konkretes Beispiel f\ue r einen Vektor im Kern ist
			\begin{align*}
				u&=\begin{pmatrix} -1\\-1\\-1\\2 \end{pmatrix}
			\end{align*}
	\end{itemize}
	\end{block}
\end{frame}

\begin{frame}\frametitle{\mytitle}
	\begin{block}{Inhomogene lineare Gleichungssysteme}
		\begin{itemize}
			\item Wie l\oe sen wir Gleichungssysteme der Form
				\begin{align*}
					Au&=y&&(A\in\RR^{m\times n},\,u\in\RR^n,\,y\in\RR^m)?
				\end{align*}
			\item Ausgeschrieben:
				\begin{align*}
					a_{11}u_1+\cdots+a_{1n}u_n&=y_1\\
											  &\vdots\\
					a_{m1}u_1+\cdots+a_{mn}u_n&=y_m
				\end{align*}
			\item Wie stellen wir fest, ob \ue berhaupt eine L\oe sung existiert?
			\item Falls ja, wie finden wir eine oder alle L\oe sungen?
		\end{itemize}
	\end{block}
\end{frame}

\begin{frame}\frametitle{\mytitle}
	\begin{block}{Rechenschema}
		\begin{itemize}
			\item Wir f\ue gen den Vektor $y$ als Spalte hinzu:
				\begin{align*}
					A'&=(A\quad y)
				\end{align*}
			\item $A'$ hat Gr\oe\ss e $m\times(n+1)$.
			\item Dann wenden wir das Gau\ss verfahren an, um die ersten $n$ Spalten von $A'$ in Zeilenstufenform zu bringen.
			\item Sei
				\begin{align*}
					B'&=(B\quad z)&&(B\in\RR^{m\times n},\,z\in\RR^m)
				\end{align*}
				das Ergebnis.
		\end{itemize}
	\end{block}
\end{frame}

\begin{frame}\frametitle{\mytitle}
	\begin{block}{Rechenschema}
		\begin{itemize}
			\item Seien $k$, $h_1,\ldots,h_\ell$ wie in der Definition der Zeilenstufenform.
			\item Falls es ein $k<i\leq m$ mit $z_i\neq0$ gibt, hat das Gleichungssystem keine L\oe sung.
			\item Sonst k\oe nnen wir alle L\oe sungen bestimmen, indem wir beliebige Werte f\ue r $u_i$ mit $i\in\{1,\ldots,n\}\setminus\{h_1,\ldots,h_k\}$ w\ae hlen und dann r\ue ckw\ae rts f\ue r $\ell=k,\ldots,1$ die Formel
				\begin{align*}
					u_{h_\ell}=\frac{z_\ell-\sum^n_{j=h_{\ell}+1}b_{\ell j}u_j}{b_{\ell\, h_\ell}}.
				\end{align*}
				anwenden.
		\end{itemize}
	\end{block}
\end{frame}

\begin{frame}\frametitle{\mytitle}
	\begin{block}{Beispiel}
	\begin{itemize}
		\item Betrachte 
			\begin{align*}
				A&=\begin{pmatrix}
					0&1&1&1\\1&1&0&1\\1&0&1&1
					\end{pmatrix},&&&y&=\begin{pmatrix} 2\\4\\8 \end{pmatrix}
			\end{align*}
		\item Wir erweitern die Matrix um den Vektor $y$; es ist \ue blich, dabei eine Trennlinie einzuf\ue gen:
			\begin{align*}
				A'&=\bc{\begin{array}{cccc|c} 0&1&1&1&2\\1&1&0&1&4\\1&0&1&1&8\end{array}}
			\end{align*}
	\end{itemize}
	\end{block}
\end{frame}

\begin{frame}\frametitle{\mytitle}
	\begin{block}{Beispiel}
		\begin{itemize}
			\item Wir bringen die Matrix $A'$ jetzt in Zeilenstufenform. 
			\item Vertausche die erste und die zweite Zeile:
				\begin{align*}
					\bc{\begin{array}{cccc|c}1&1&0&1&4\\0&1&1&1&2\\1&0&1&1&8\end{array}}
				\end{align*}
			\item Subtrahiere die erste Zeile von der letzten Zeile:
				\begin{align*}
					\bc{\begin{array}{cccc|c}1&1&0&1&4\\0&1&1&1&2\\0&-1&1&0&4\end{array}}
				\end{align*}
		\end{itemize}
	\end{block}
\end{frame}

\begin{frame}\frametitle{\mytitle}
	\begin{block}{Beispiel}
		\begin{itemize}
			\item Addiere die zweite Zeile zur letzten Zeile:
				\begin{align*}
					\bc{\begin{array}{cccc|c}1&1&0&1&4\\0&1&1&1&2\\0&0&2&1&6\end{array}}
				\end{align*}
			\item Die Matrix ist nun in Zeilenstufenform mit
				\begin{align*}
					k&=3&h_1&=1&h_2&=2&h_3&=3
				\end{align*}
			\item Wir d\ue rfen also $u_4$ frei w\ae hlen.
			\item F\ue r die \ue brigen Variablen erhalten wir
				\begin{align*}
					u_3&=\frac{6-u_4}{2}=3-\frac{u_4}{2}&
					u_2&=2-u_3-u_4=2-\frac{u_4}2\\
					u_1&=4-u_2-u_4=5-\frac{u_4}{2}
				\end{align*}
		\end{itemize}
	\end{block}
\end{frame}

\begin{frame}\frametitle{\mytitle}
	\begin{block}{Beispiel}
		\begin{itemize}
			\item Die L\oe sungsmenge des linearen Gleichungssystems lautet also
				\begin{align*}
					\cbc{u\in\RR^4:Au=y}&=\cbc{\begin{pmatrix}
					5-u_4/2\\2-u_4/2\\3-u_4/2\\u_4
			\end{pmatrix}:u_4\in\RR}
				\end{align*}
			\item Ein konkretes Beispiel f\ue r eine L\oe sung ist
				\begin{align*}
					u&=\begin{pmatrix} 5\\2\\3\\0 \end{pmatrix}
				\end{align*}
		\end{itemize}
	\end{block}
\end{frame}


\begin{frame}\frametitle{\mytitle}
	\begin{block}{Beispiel}
		\begin{itemize}
			\item Sei
				\begin{align*}
					A&=\begin{pmatrix} 1&1&1\\2&2&2\\3&3&4\end{pmatrix}&&&y&=\begin{pmatrix} 1\\-1\\0 \end{pmatrix}
				\end{align*}
			\item Die erweiterte Matrix lautet
				\begin{align*}
					A'&=\bc{\begin{array}{ccc|c} 1&1&1&1\\2&2&2&-1\\3&3&4&0\end{array}}
				\end{align*}
		\end{itemize}
	\end{block}
\end{frame}

\begin{frame}\frametitle{\mytitle}
	\begin{block}{Beispiel}
		\begin{itemize}
			\item Wir bringen die Matrix in Zeilenstufenform.
			\item Subtrahiere die erste Zeil zweimal von der zweiten und dreimal von der dritten:
				\begin{align*}
				\bc{\begin{array}{ccc|c} 1&1&1&1\\0&0&0&-3\\0&0&1&-3\end{array}}
				\end{align*}
			\item Vertausche die zweite und die dritte Zeile:
\begin{align*}
	B'&=\bc{\begin{array}{ccc|c} 1&1&1&1\\0&0&1&-3\\0&0&0&-3\end{array}}
				\end{align*}
		\end{itemize}
	\end{block}
\end{frame}

\begin{frame}\frametitle{\mytitle}
	\hfill$B'=\bc{\begin{array}{ccc|c} 1&1&1&1\\0&0&1&-3\\0&0&0&-3\end{array}}$
	\begin{block}{Beispiel}
		\begin{itemize}
			\item Die Matrix ist jetzt in Zeilenstufenform mit
				\begin{align*}
					k&=2&h_1&=1&h_2&=3.
				\end{align*}
			\item Wir haben
				\begin{align*}
					B&=\begin{pmatrix}1&1&1\\0&0&1\\0&0&0\end{pmatrix}&
					z&=\begin{pmatrix} 1\\-3\\-3 \end{pmatrix}
				\end{align*}
			\item Weil $z_3=-3\neq0$, hat das Gleichungssystem keine L\oe sung. 
		\end{itemize}
	\end{block}
\end{frame}

\begin{frame}\frametitle{\mytitle}
	\begin{block}{Satz}
		Sei $A$ eine $m\times n$-Matrix und $y\in\RR^m$.
		\begin{itemize}
			\item Das lineare Gleichungssystem $Au=z$ hat entweder gar keine, eine einzige oder unendlich viele L\oe sungen.
			\item Wenn $x\in\RR^n$ eine L\oe sung des Gleichungssystems ist, dann ist
				\begin{align*}
					x+\ker(A)=\cbc{x+v:v\in\ker(A)}
				\end{align*}
				die L\oe sungsmenge.
		\end{itemize}
	\end{block}
	\emph{Wenn das lineare Gleichungssystem eine L\oe sung hat, ist die L\oe sungsmenge also immer eine Translation des Kerns.}
\end{frame}

\begin{frame}\frametitle{\mytitle}
	\begin{block}{Zusammenfassung}
	\begin{itemize}
	\item Mit dem Gau\ss verfahren k\oe nnen Matrizen in Zeilenstufenform gebracht werden.
	\item Wir k\oe nnen dann den Kern der Matrix ablesen.
	\item Inhomogene lineare Gleichungssysteme k\oe nnen ebenfalls mit dem Gau\ss verfahren gel\oe st werden.
	\end{itemize}
	\end{block}
\end{frame}

\end{document}
