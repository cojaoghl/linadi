\documentclass{beamer}
\usepackage{amsmath,graphics}
\usepackage{amssymb}

\usetheme{default}
\usepackage{xcolor}

\definecolor{solarizedBase03}{HTML}{002B36}
\definecolor{solarizedBase02}{HTML}{073642}
\definecolor{solarizedBase01}{HTML}{586e75}
\definecolor{solarizedBase00}{HTML}{657b83}
\definecolor{solarizedBase0}{HTML}{839496}
\definecolor{solarizedBase1}{HTML}{93a1a1}
\definecolor{solarizedBase2}{HTML}{EEE8D5}
\definecolor{solarizedBase3}{HTML}{FDF6E3}
\definecolor{solarizedYellow}{HTML}{B58900}
\definecolor{solarizedOrange}{HTML}{CB4B16}
\definecolor{solarizedRed}{HTML}{DC322F}
\definecolor{solarizedMagenta}{HTML}{D33682}
\definecolor{solarizedViolet}{HTML}{6C71C4}
%\definecolor{solarizedBlue}{HTML}{268BD2}
\definecolor{solarizedBlue}{HTML}{134676}
\definecolor{solarizedCyan}{HTML}{2AA198}
\definecolor{solarizedGreen}{HTML}{859900}
\definecolor{myBlue}{HTML}{162DB0}%{261CA4}
\setbeamercolor*{item}{fg=myBlue}
\setbeamercolor{normal text}{fg=solarizedBase03, bg=solarizedBase3}
\setbeamercolor{alerted text}{fg=myBlue}
\setbeamercolor{example text}{fg=myBlue, bg=solarizedBase3}
\setbeamercolor*{frametitle}{fg=solarizedRed}
\setbeamercolor*{title}{fg=solarizedRed}
\setbeamercolor{block title}{fg=myBlue, bg=solarizedBase3}
\setbeameroption{hide notes}
\setbeamertemplate{note page}[plain]
\beamertemplatenavigationsymbolsempty
\usefonttheme{professionalfonts}
\usefonttheme{serif}

\usepackage{fourier}

\def\vec#1{\mathchoice{\mbox{\boldmath$\displaystyle#1$}}
{\mbox{\boldmath$\textstyle#1$}}
{\mbox{\boldmath$\scriptstyle#1$}}
{\mbox{\boldmath$\scriptscriptstyle#1$}}}
\definecolor{OwnGrey}{rgb}{0.560,0.000,0.000} % #999999
\definecolor{OwnBlue}{rgb}{0.121,0.398,0.711} % #1f64b0
\definecolor{red4}{rgb}{0.5,0,0}
\definecolor{blue4}{rgb}{0,0,0.5}
\definecolor{Blue}{rgb}{0,0,0.66}
\definecolor{LightBlue}{rgb}{0.9,0.9,1}
\definecolor{Green}{rgb}{0,0.5,0}
\definecolor{LightGreen}{rgb}{0.9,1,0.9}
\definecolor{Red}{rgb}{0.9,0,0}
\definecolor{LightRed}{rgb}{1,0.9,0.9}
\definecolor{White}{gray}{1}
\definecolor{Black}{gray}{0}
\definecolor{LightGray}{gray}{0.8}
\definecolor{Orange}{rgb}{0.1,0.2,1}
\setbeamerfont{sidebar right}{size=\scriptsize}
\setbeamercolor{sidebar right}{fg=Black}

\renewcommand{\emph}[1]{{\textcolor{solarizedRed}{\itshape #1}}}

\newcommand\cA{\mathcal A}
\newcommand\cB{\mathcal B}
\newcommand\cC{\mathcal C}
\newcommand\cD{\mathcal D}
\newcommand\cE{\mathcal E}
\newcommand\cF{\mathcal F}
\newcommand\cG{\mathcal G}
\newcommand\cH{\mathcal H}
\newcommand\cI{\mathcal I}
\newcommand\cJ{\mathcal J}
\newcommand\cK{\mathcal K}
\newcommand\cL{\mathcal L}
\newcommand\cM{\mathcal M}
\newcommand\cN{\mathcal N}
\newcommand\cO{\mathcal O}
\newcommand\cP{\mathcal P}
\newcommand\cQ{\mathcal Q}
\newcommand\cR{\mathcal R}
\newcommand\cS{\mathcal S}
\newcommand\cT{\mathcal T}
\newcommand\cU{\mathcal U}
\newcommand\cV{\mathcal V}
\newcommand\cW{\mathcal W}
\newcommand\cX{\mathcal X}
\newcommand\cY{\mathcal Y}
\newcommand\cZ{\mathcal Z}

\newcommand\fA{\mathfrak A}
\newcommand\fB{\mathfrak B}
\newcommand\fC{\mathfrak C}
\newcommand\fD{\mathfrak D}
\newcommand\fE{\mathfrak E}
\newcommand\fF{\mathfrak F}
\newcommand\fG{\mathfrak G}
\newcommand\fH{\mathfrak H}
\newcommand\fI{\mathfrak I}
\newcommand\fJ{\mathfrak J}
\newcommand\fK{\mathfrak K}
\newcommand\fL{\mathfrak L}
\newcommand\fM{\mathfrak M}
\newcommand\fN{\mathfrak N}
\newcommand\fO{\mathfrak O}
\newcommand\fP{\mathfrak P}
\newcommand\fQ{\mathfrak Q}
\newcommand\fR{\mathfrak R}
\newcommand\fS{\mathfrak S}
\newcommand\fT{\mathfrak T}
\newcommand\fU{\mathfrak U}
\newcommand\fV{\mathfrak V}
\newcommand\fW{\mathfrak W}
\newcommand\fX{\mathfrak X}
\newcommand\fY{\mathfrak Y}
\newcommand\fZ{\mathfrak Z}

\newcommand\fa{\mathfrak a}
\newcommand\fb{\mathfrak b}
\newcommand\fc{\mathfrak c}
\newcommand\fd{\mathfrak d}
\newcommand\fe{\mathfrak e}
\newcommand\ff{\mathfrak f}
\newcommand\fg{\mathfrak g}
\newcommand\fh{\mathfrak h}
%\newcommand\fi{\mathfrak i}
\newcommand\fj{\mathfrak j}
\newcommand\fk{\mathfrak k}
\newcommand\fl{\mathfrak l}
\newcommand\fm{\mathfrak m}
\newcommand\fn{\mathfrak n}
\newcommand\fo{\mathfrak o}
\newcommand\fp{\mathfrak p}
\newcommand\fq{\mathfrak q}
\newcommand\fr{\mathfrak r}
\newcommand\fs{\mathfrak s}
\newcommand\ft{\mathfrak t}
\newcommand\fu{\mathfrak u}
\newcommand\fv{\mathfrak v}
\newcommand\fw{\mathfrak w}
\newcommand\fx{\mathfrak x}
\newcommand\fy{\mathfrak y}
\newcommand\fz{\mathfrak z}

\newcommand\vA{\vec A}
\newcommand\vB{\vec B}
\newcommand\vC{\vec C}
\newcommand\vD{\vec D}
\newcommand\vE{\vec E}
\newcommand\vF{\vec F}
\newcommand\vG{\vec G}
\newcommand\vH{\vec H}
\newcommand\vI{\vec I}
\newcommand\vJ{\vec J}
\newcommand\vK{\vec K}
\newcommand\vL{\vec L}
\newcommand\vM{\vec M}
\newcommand\vN{\vec N}
\newcommand\vO{\vec O}
\newcommand\vP{\vec P}
\newcommand\vQ{\vec Q}
\newcommand\vR{\vec R}
\newcommand\vS{\vec S}
\newcommand\vT{\vec T}
\newcommand\vU{\vec U}
\newcommand\vV{\vec V}
\newcommand\vW{\vec W}
\newcommand\vX{\vec X}
\newcommand\vY{\vec Y}
\newcommand\vZ{\vec Z}

\newcommand\va{\vec a}
\newcommand\vb{\vec b}
\newcommand\vc{\vec c}
\newcommand\vd{\vec d}
\newcommand\ve{\vec e}
\newcommand\vf{\vec f}
\newcommand\vg{\vec g}
\newcommand\vh{\vec h}
\newcommand\vi{\vec i}
\newcommand\vj{\vec j}
\newcommand\vk{\vec k}
\newcommand\vl{\vec l}
\newcommand\vm{\vec m}
\newcommand\vn{\vec n}
\newcommand\vo{\vec o}
\newcommand\vp{\vec p}
\newcommand\vq{\vec q}
\newcommand\vr{\vec r}
\newcommand\vs{\vec s}
\newcommand\vt{\vec t}
\newcommand\vu{\vec u}
\newcommand\vv{\vec v}
\newcommand\vw{\vec w}
\newcommand\vx{\vec x}
\newcommand\vy{\vec y}
\newcommand\vz{\vec z}

\renewcommand\AA{\mathbb A}
\newcommand\NN{\mathbb N}
\newcommand\ZZ{\mathbb Z}
\newcommand\PP{\mathbb P}
\newcommand\QQ{\mathbb Q}
\newcommand\RR{\mathbb R}
\renewcommand\SS{\mathbb S}
\newcommand\CC{\mathbb C}

\newcommand{\ord}{\mathrm{ord}}
\newcommand{\id}{\mathrm{id}}
\newcommand{\pr}{\mathrm{P}}
\newcommand{\Vol}{\mathrm{vol}}
\newcommand\norm[1]{\left\|{#1}\right\|} 
\newcommand\sign{\mathrm{sign}}
\newcommand{\eps}{\varepsilon}
\newcommand{\abs}[1]{\left|#1\right|}
\newcommand\bc[1]{\left({#1}\right)} 
\newcommand\cbc[1]{\left\{{#1}\right\}} 
\newcommand\bcfr[2]{\bc{\frac{#1}{#2}}} 
\newcommand{\bck}[1]{\left\langle{#1}\right\rangle} 
\newcommand\brk[1]{\left\lbrack{#1}\right\rbrack} 
\newcommand\scal[2]{\bck{{#1},{#2}}} 
\newcommand{\vecone}{\mathbb{1}}
\newcommand{\tensor}{\otimes}
\newcommand{\diag}{\mathrm{diag}}
\newcommand{\ggt}{\mathrm{ggT}}
\newcommand{\kgv}{\mathrm{kgV}}
\newcommand{\trans}{\top}

\newcommand{\Karonski}{Karo\'nski}
\newcommand{\Erdos}{Erd\H{o}s}
\newcommand{\Renyi}{R\'enyi}
\newcommand{\Lovasz}{Lov\'asz}
\newcommand{\Juhasz}{Juh\'asz}
\newcommand{\Bollobas}{Bollob\'as}
\newcommand{\Furedi}{F\"uredi}
\newcommand{\Komlos}{Koml\'os}
\newcommand{\Luczak}{\L uczak}
\newcommand{\Kucera}{Ku\v{c}era}
\newcommand{\Szemeredi}{Szemer\'edi}

\renewcommand{\ae}{\"a}
\renewcommand{\oe}{\"o}
\newcommand{\ue}{\"u}
\newcommand{\Ae}{\"A}
\newcommand{\Oe}{\"O}
\newcommand{\Ue}{\"U}

\newcommand{\im}{\mathrm{im}}
\newcommand{\rrk}{\mathrm{zrg}}
\newcommand{\crk}{\mathrm{srg}}
\newcommand{\rk}{\mathrm{rg}}
\newcommand{\GL}{\mathrm{GL}}

\newcommand{\mytitle}{Die Determinante}

\title[Linadi]{\mytitle}
\author[Amin Coja-Oghlan]{Amin Coja-Oghlan}
\institute[Frankfurt]{JWGUFFM}
\date{}

\begin{document}

\frame[plain]{\titlepage}

\begin{frame}\frametitle{\mytitle}
	\begin{block}{Definition}
		\begin{itemize}
			\item Sei $A=(a_{ij})$ eine $n\times n$-Matrix.
			\item \alert{Erinnerung:} $\SS_n$ ist die symmetrische Gruppe und $\sign(\sigma)\in\{\pm1\}$ ist das Signum der Permutation $\sigma$.
			\item Die \emph{Determinante} von $A$ ist definiert als
				\begin{align*}
					\det(A)&=\sum_{\sigma\in\SS_n}\sign(\sigma)\prod_{i=1}^n a_{i\,\sigma_i}\in\RR.
				\end{align*}
			\item Man schreibt auch $|A|=\det(A)$.
		\end{itemize}
	\end{block}
\end{frame}

\begin{frame}\frametitle{\mytitle}
	\begin{block}{Beispiel}
		\begin{itemize}
			\item Im Fall $n=2$ gibt es nur zwei Permutationen: die Identit\ae t $1\mapsto 1$, $2\mapsto 2$ und die Transposition $1\mapsto 2$, $2\mapsto 1$.
			\item Die Identit\ae t hat Signum $1$ und die Transposition hat Signum $-1$.
			\item Die Formel f\ue r die Determinante einer $2\times 2$-Matrix lautet daher
				\begin{align*}
					\det\begin{pmatrix} a_{11}&a_{12}\\a_{21}&a_{22} \end{pmatrix}=a_{11}a_{22}-a_{12}a_{21}.
				\end{align*}
			\item \emph{Zahlenbeispiel:}
				\begin{align*}
					\det\begin{pmatrix}1&2\\3&4\end{pmatrix}&=1\cdot 4-2\cdot 3=-2.
				\end{align*}
		\end{itemize}
	\end{block}
\end{frame}

\begin{frame}\frametitle{\mytitle}
	\begin{block}{Rechenregeln f\ue r die Determinante}
		\begin{itemize}
			\item Es gilt $\det(\id_n)=1$.
			\item F\ue r $n\times n$-Matrizen $A,B$ gilt $\det(A\cdot B)=\det(A)\cdot\det(B)$.
			\item F\ue r $n\times n$-Matrizen $A$ gilt $\det(A^\trans)=\det(A)$.
			\item F\ue r eine reelle Zahl $c$ und eine $n\times n$-Matrix $A$ gilt
				\begin{align*}
					\det(c\cdot A)&=c^n\det(A).
				\end{align*}
		\end{itemize}
	\end{block}
\end{frame}

\begin{frame}\frametitle{\mytitle}
	\begin{block}{Rechenregeln f\ue r die Determinante}
		Sei $A$ eine $n\times n$-Matrix.
		\begin{itemize}
			\item Wenn $A$ eine Zeile besitzt, die nur aus Nullen besteht, dann gilt
				$$\det(A)=0.$$
			\item Wenn $A$ zwei identische Zeilen besitzt, so gilt
				$$\det(A)=0.$$
			\item Wenn $A$ zwei identische Spalten besitzt, so gilt
				$$\det(A)=0.$$
		\end{itemize}
	\end{block}
\end{frame}

\begin{frame}\frametitle{\mytitle}
	\begin{block}{Rechenregeln f\ue r die Determinante}
		Sei $A$ eine $n\times n$-Matrix.
		\begin{itemize}
			\item Wenn $B$ aus $A$ durch eine \alert{Zeilen- oder Spaltenpivotoperation} entsteht, dann gilt
				\begin{align*}
					\det(B)=\det(A).
				\end{align*}
			\item Wenn $B$ aus $A$ durch \alert{Vertauschen} von zwei Zeilen oder Spalten entsteht, dann gilt
				\begin{align*}
					\det(B)=-\det(A).
				\end{align*}
			\item Wenn $B$ aus $A$ durch \alert{Skalieren} einer Zeile oder Spalte mit $c\in\RR$ entsteht, dann gilt
				\begin{align*}
					\det(B)=c\cdot\det(A).
				\end{align*}
		\end{itemize}
	\end{block}
\end{frame}

\begin{frame}\frametitle{\mytitle}
	\begin{block}{Rechenregeln f\ue r die Determinante}
		Sei $A$ eine $n\times n$-Matrix.
		\begin{itemize}
			\item Wenn $A=(a_{ij})$ in Zeilenstufenform ist, dann gilt
				\begin{align*}
					\det(A)&=\prod_{i=1}^na_{ii}.
				\end{align*}
			\item Wenn $A=(a_{ij})$ in Spaltenstufenform ist, dann gilt
				\begin{align*}
					\det(A)&=\prod_{i=1}^na_{ii}.
				\end{align*}
		\end{itemize}
	\end{block}
\end{frame}

\begin{frame}\frametitle{\mytitle}
	\begin{block}{Rechenschema f\ue r die Determinante}
		\begin{itemize}
			\item Sei $A$ eine $n\times n$-Matrix.
			\item \emph{Ziel:} $\det(A)$ berechnen.
			\item Wir f\ue hren Zeilen- und Spaltenumformungen durch, um $A$ in Zeilenstufenform zu bringen.
			\item Jedes Mal, wenn wir zwei Zeilen oder Spalten vertauschen, \ae ndert sich dabei das Vorzeichen der Determinante.
			\item Wenn wir eine Zeile oder Spalte mit $c\in\RR$ skalieren, dann \ae ndert sich die Determinante um denselben Faktor $c$.
			\item Wir m\ue ssen uns also genau merken, welche Skalierungen  und wieviele Vertauschungen wir durchgef\ue hrt haben!
		\end{itemize}
	\end{block}
\end{frame}

\begin{frame}\frametitle{\mytitle}
	\begin{block}{Beispiel}
		\begin{itemize}
			\item Sei
\begin{align*}
				A&=\begin{pmatrix}
					-4&4&4\\1&-2&1\\1&1&-3
				\end{pmatrix}
				\end{align*}
			\item Wir addieren die erste Spalte zur zweiten Spalte; dabei \ae ndert sich die Determinante nicht:
\begin{align*}
	\det(A)&=\det\begin{pmatrix}
					-4&0&0\\1&-1&2\\1&2&-2
				\end{pmatrix}
				\end{align*}
		\end{itemize}
	\end{block}
\end{frame}

\begin{frame}\frametitle{\mytitle}
	\begin{block}{Beispiel}
		\begin{itemize}
			\item Jetzt vertauschen wir die erste und die letzte Zeile; dabei \ae ndert sich das Vorzeichen der Determinante:
				\begin{align*}
					\det(A)&=-\det\begin{pmatrix} 1&2&-2\\1&-1&2\\-4&0&0 \end{pmatrix}
				\end{align*}
			\item Als n\ae chstes tauschen wir die erste und die letzte Spalte; wiederum \ae ndert sich das Vorzeichen:
				\begin{align*}
					\det(A)&=\det\begin{pmatrix} -2&2&1\\2&-1&1\\0&0&-4 \end{pmatrix}
				\end{align*}
		\end{itemize}
	\end{block}
\end{frame}

\begin{frame}\frametitle{\mytitle}
	\begin{block}{Beispiel}
		\begin{itemize}
			\item Wir multiplizieren nun die letzte Zeile mit $-\frac{1}{4}$; zum Ausgleich dividieren wir durch $-\frac{1}{4}$:
				\begin{align*}
					\det(A)&=-4\cdot\det\begin{pmatrix} -2&2&1\\2&-1&1\\0&0&1 \end{pmatrix}
				\end{align*}
			\item Als n\ae chstes addieren wir die erste Zeile zur zweiten Zeile; die Determinante bleibt gleich:
\begin{align*}
					\det(A)&=-4\cdot\det\begin{pmatrix} -2&2&1\\0&1&2\\0&0&1 \end{pmatrix}
				\end{align*}
		\end{itemize}
	\end{block}
\end{frame}

\begin{frame}\frametitle{\mytitle}
	\begin{block}{Beispiel}
		\begin{itemize}
			\item Die letzte Matrix ist in Zeilenstufenform, so da\ss\ wir ihre Determinante leicht ausrechnen k\oe nnen:
\begin{align*}
	\det\begin{pmatrix} -2&2&1\\0&1&2\\0&0&1 \end{pmatrix}&=(-2)\cdot1\cdot 1=-2.
				\end{align*}
			\item Wir erhalten also
\begin{align*}
	\det(A)&=-4\cdot\det\begin{pmatrix} -2&2&1\\0&1&2\\0&0&1 \end{pmatrix}=(-4)\cdot(-2)=8
				\end{align*}
		\end{itemize}
	\end{block}
\end{frame}

\begin{frame}\frametitle{\mytitle}
	\begin{block}{Beispiel}
		\begin{itemize}
			\item Wir berechnen die Determinante der Matrix	
				\begin{align*}
					A&=\begin{pmatrix}
						0&0&0&7\\0&0&-1&0\\0&-1&1&0\\1&0&0&0
					\end{pmatrix}
				\end{align*}
			\item Wir vertauschen die erste und die letzte Zeile; dabei \ae ndert sich das Vorzeichen der Determinante:
\begin{align*}
	\det(A)&=-\det\begin{pmatrix}
						1&0&0&0\\0&0&-1&0\\0&-1&1&0\\0&0&0&7
					\end{pmatrix}
				\end{align*}
		\end{itemize}
	\end{block}
\end{frame}

\begin{frame}\frametitle{\mytitle}
	\begin{block}{Beispiel}
		\begin{itemize}
			\item Als n\ae chstes vertausche die zweite und dritte Zeile; dabei \ae ndert sich das Vorzeichen der Determinante nochmal:
\begin{align*}
	\det(A)&=\det\begin{pmatrix}
						1&0&0&0\\0&-1&1&0\\0&0&-1&0\\0&0&0&7
					\end{pmatrix}
				\end{align*}
			\item Die Matrix ist jetzt in Zeilenstufenform und wir berechnen
\begin{align*}
	\det(A)&=\det\begin{pmatrix}
						1&0&0&0\\0&-1&1&0\\0&0&-1&0\\0&0&0&7
					\end{pmatrix}=1\cdot(-1)\cdot(-1)\cdot 7=7.
				\end{align*}
		\end{itemize}
	\end{block}
\end{frame}



\begin{frame}\frametitle{\mytitle}
	\begin{block}{Entwicklung der Determinante nach einer Zeile}
		\begin{itemize}
	\item Normalerweise ist die Umformung in Zeilenstufenform der effizienteste Weg, die Determinante zu berechnen.
	\item Falls aber eine Zeile der Matrix sehr wenige von Null verschiedene Eintr\ae ge hat, kann die Entwicklung nach einer Zeile g\ue nstig sein.
	\item Sei dazu $A=(a_{ij})$ eine $n\times n$-Matrix und $i\in\{1,\ldots,n\}$.
	\item Sei ferner $A_{ij}$ die Matrix, die aus $A$ durch Weglassen der $i$-ten Zeile sowie der $j$-ten Spalte entsteht.
	\item Die Formel f\ue r die \emph{Entwicklung nach der $i$-ten Zeile} lautet
		\begin{align*}
			\det(A)&=\sum_{j=1}^n(-1)^{i+j}a_{ij}\det(A_{ij}).
		\end{align*}
		\end{itemize}
	\end{block}
\end{frame}

\begin{frame}\frametitle{\mytitle}
	\begin{block}{Beispiel}
		\begin{itemize}
			\item Wir berechnen die Determinante von
				\begin{align*}
					A&=\begin{pmatrix}
						0&0&0&7\\0&0&-1&0\\0&-1&1&0\\1&0&0&0
					\end{pmatrix}
				\end{align*}
				durch Entwicklung nach der dritten Zeile.
		\end{itemize}
	\end{block}
\end{frame}

\begin{frame}\frametitle{\mytitle}
	\begin{block}{Beispiel}
		\begin{itemize}
			\item Wir erhalten
				\begin{align*}
					\det(A)&=\sum_{j=1}^4(-1)^{3+j}a_{3j}\det(A_{3j})\\
						   &=0\cdot\abs{\begin{array}{ccc}0&0&7\\0&-1&0\\0&0&0\end{array}}-(-1)\cdot\abs{\begin{array}{ccc}0&0&7\\0&-1&0\\1&0&0\end{array}}\\
						&+1\cdot\abs{\begin{array}{ccc}0&0&7\\0&0&0\\1&0&0\end{array}}
						-0\cdot\abs{\begin{array}{ccc}0&0&0\\0&0&-1\\1&0&0\end{array}}\\
													   &=\abs{\begin{array}{ccc}0&0&7\\0&-1&0\\1&0&0\end{array}}+\abs{\begin{array}{ccc}0&0&7\\0&0&0\\1&0&0\end{array}}\\
				\end{align*}
		\end{itemize}
	\end{block}
\end{frame}

\begin{frame}\frametitle{\mytitle}
	\begin{block}{Beispiel}
		\begin{itemize}
			\item Es gilt
				\begin{align*}
					\abs{\begin{array}{ccc}0&0&7\\0&-1&0\\1&0&0\end{array}}&=7&
					\abs{\begin{array}{ccc}0&0&7\\0&0&0\\1&0&0\end{array}}&=0
				\end{align*}
			\item Also erhalten wir 
				\begin{align*}
					\det(A)&=\abs{\begin{array}{ccc}0&0&7\\0&-1&0\\1&0&0\end{array}}+\abs{\begin{array}{ccc}0&0&7\\0&0&0\\1&0&0\end{array}}=7.
				\end{align*}
		\end{itemize}
	\end{block}
\end{frame}

\begin{frame}\frametitle{\mytitle}
	\begin{block}{Proposition}
		Eine $n\times n$-Matrix	$A$ ist genau dann invertierbar, wenn $\det(A)\neq0$.
		In diesem Fall gilt
		\begin{align*}
			\det(A^{-1})=\frac{1}{\det(A)}.
		\end{align*}
	\end{block}
\end{frame}

\begin{frame}\frametitle{\mytitle}
	\begin{block}{Die Cramersche Regel}
		\begin{itemize}
			\item Angenommen $A$ ist eine invertierbare $n\times n$-Matrix.
			\item Sei $y\in\RR^n$.
			\item Dann gibt es genau einen Vektor $x\in\RR^n$ mit $Ax=y$
			\item Seine Eintr\ae ge lauten
				\begin{align*}
					x_i&=\det\begin{pmatrix}
						a_{11}&\cdots&a_{1\,i-1}&y_1&a_{1\,i+1}&\cdots&a_{1n}\\
						\vdots&\ddots&\vdots\\
						a_{n1}&\cdots&a_{n\,i-1}&y_n&a_{n\,i+1}&\cdots&a_{nn}
					\end{pmatrix}/\det(A).
				\end{align*}
			\item \emph{Die Cramersche Regel ist normalerweise kein praktisches Rechenschema zum L\oe sen linearer Gleichungssysteme.}
		\end{itemize}
	\end{block}
\end{frame}

\begin{frame}\frametitle{\mytitle}
	\begin{block}{Beispiel}
		\begin{itemize}
			\item Wir l\oe sen das Gleichungssystem
				\begin{align*}
				\begin{pmatrix} 1&2\\3&4 \end{pmatrix}\cdot\binom{x_1}{x_2}=\binom56
				\end{align*}
				mit der Cramerschen Regel.
			\item Dazu berechnen wir
				\begin{align*}
					\det\begin{pmatrix} 1&2\\3&4 \end{pmatrix}&=1\cdot4-2\cdot 3=-2&
					\det\begin{pmatrix} 5&2\\6&4 \end{pmatrix}&=5\cdot4-2\cdot 6=8\\
					\det\begin{pmatrix} 1&5\\3&6 \end{pmatrix}&=1\cdot 6-3\cdot 5=-9
				\end{align*}
		\end{itemize}
	\end{block}
\end{frame}

\begin{frame}\frametitle{\mytitle}
	\begin{block}{Beispiel}
		\begin{itemize}
			\item Wir erhalten die L\oe sung
				\begin{align*}
					x_1&=\frac{\det\begin{pmatrix} 5&2\\6&4 \end{pmatrix}}{\det\begin{pmatrix} 1&2\\3&4 \end{pmatrix}}=\frac{8}{-2}=-4&
					x_2&=\frac{\det\begin{pmatrix} 1&5\\3&6 \end{pmatrix}}{\det\begin{pmatrix} 1&2\\3&4 \end{pmatrix}}=\frac{-9}{-2}=\frac{9}{2}
				\end{align*}
		\end{itemize}
	\end{block}
\end{frame}

\begin{frame}\frametitle{\mytitle}
	\begin{block}{Zusammenfassung}
		\begin{itemize}
			\item Die Determinante kann mittels Zeilen- und Spaltenumformungen berechnet werden.
			\item Die Cramersche Regel bietet eine Formel f\ue r die L\oe sung linearer Gleichungssysteme in Begriffen der Determinante.
		\end{itemize}
	\end{block}
\end{frame}


\end{document}
