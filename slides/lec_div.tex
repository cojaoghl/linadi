\documentclass{beamer}
\usepackage{amsmath,graphics}
\usepackage{amssymb}

\usetheme{default}
\usepackage{xcolor}

\definecolor{solarizedBase03}{HTML}{002B36}
\definecolor{solarizedBase02}{HTML}{073642}
\definecolor{solarizedBase01}{HTML}{586e75}
\definecolor{solarizedBase00}{HTML}{657b83}
\definecolor{solarizedBase0}{HTML}{839496}
\definecolor{solarizedBase1}{HTML}{93a1a1}
\definecolor{solarizedBase2}{HTML}{EEE8D5}
\definecolor{solarizedBase3}{HTML}{FDF6E3}
\definecolor{solarizedYellow}{HTML}{B58900}
\definecolor{solarizedOrange}{HTML}{CB4B16}
\definecolor{solarizedRed}{HTML}{DC322F}
\definecolor{solarizedMagenta}{HTML}{D33682}
\definecolor{solarizedViolet}{HTML}{6C71C4}
%\definecolor{solarizedBlue}{HTML}{268BD2}
\definecolor{solarizedBlue}{HTML}{134676}
\definecolor{solarizedCyan}{HTML}{2AA198}
\definecolor{solarizedGreen}{HTML}{859900}
\definecolor{myBlue}{HTML}{162DB0}%{261CA4}
\setbeamercolor*{item}{fg=myBlue}
\setbeamercolor{normal text}{fg=solarizedBase03, bg=solarizedBase3}
\setbeamercolor{alerted text}{fg=myBlue}
\setbeamercolor{example text}{fg=myBlue, bg=solarizedBase3}
\setbeamercolor*{frametitle}{fg=solarizedRed}
\setbeamercolor*{title}{fg=solarizedRed}
\setbeamercolor{block title}{fg=myBlue, bg=solarizedBase3}
\setbeameroption{hide notes}
\setbeamertemplate{note page}[plain]
\beamertemplatenavigationsymbolsempty
\usefonttheme{professionalfonts}
\usefonttheme{serif}

\usepackage{fourier}

\def\vec#1{\mathchoice{\mbox{\boldmath$\displaystyle#1$}}
{\mbox{\boldmath$\textstyle#1$}}
{\mbox{\boldmath$\scriptstyle#1$}}
{\mbox{\boldmath$\scriptscriptstyle#1$}}}

\definecolor{OwnGrey}{rgb}{0.560,0.000,0.000} % #999999
\definecolor{OwnBlue}{rgb}{0.121,0.398,0.711} % #1f64b0
\definecolor{red4}{rgb}{0.5,0,0}
\definecolor{blue4}{rgb}{0,0,0.5}
\definecolor{Blue}{rgb}{0,0,0.66}
\definecolor{LightBlue}{rgb}{0.9,0.9,1}
\definecolor{Green}{rgb}{0,0.5,0}
\definecolor{LightGreen}{rgb}{0.9,1,0.9}
\definecolor{Red}{rgb}{0.9,0,0}
\definecolor{LightRed}{rgb}{1,0.9,0.9}
\definecolor{White}{gray}{1}
\definecolor{Black}{gray}{0}
\definecolor{LightGray}{gray}{0.8}
\definecolor{Orange}{rgb}{0.1,0.2,1}
\setbeamerfont{sidebar right}{size=\scriptsize}
\setbeamercolor{sidebar right}{fg=Black}

\renewcommand{\emph}[1]{{\textcolor{solarizedRed}{\itshape #1}}}

\newcommand\cA{\mathcal A}
\newcommand\cB{\mathcal B}
\newcommand\cC{\mathcal C}
\newcommand\cD{\mathcal D}
\newcommand\cE{\mathcal E}
\newcommand\cF{\mathcal F}
\newcommand\cG{\mathcal G}
\newcommand\cH{\mathcal H}
\newcommand\cI{\mathcal I}
\newcommand\cJ{\mathcal J}
\newcommand\cK{\mathcal K}
\newcommand\cL{\mathcal L}
\newcommand\cM{\mathcal M}
\newcommand\cN{\mathcal N}
\newcommand\cO{\mathcal O}
\newcommand\cP{\mathcal P}
\newcommand\cQ{\mathcal Q}
\newcommand\cR{\mathcal R}
\newcommand\cS{\mathcal S}
\newcommand\cT{\mathcal T}
\newcommand\cU{\mathcal U}
\newcommand\cV{\mathcal V}
\newcommand\cW{\mathcal W}
\newcommand\cX{\mathcal X}
\newcommand\cY{\mathcal Y}
\newcommand\cZ{\mathcal Z}

\newcommand\fA{\mathfrak A}
\newcommand\fB{\mathfrak B}
\newcommand\fC{\mathfrak C}
\newcommand\fD{\mathfrak D}
\newcommand\fE{\mathfrak E}
\newcommand\fF{\mathfrak F}
\newcommand\fG{\mathfrak G}
\newcommand\fH{\mathfrak H}
\newcommand\fI{\mathfrak I}
\newcommand\fJ{\mathfrak J}
\newcommand\fK{\mathfrak K}
\newcommand\fL{\mathfrak L}
\newcommand\fM{\mathfrak M}
\newcommand\fN{\mathfrak N}
\newcommand\fO{\mathfrak O}
\newcommand\fP{\mathfrak P}
\newcommand\fQ{\mathfrak Q}
\newcommand\fR{\mathfrak R}
\newcommand\fS{\mathfrak S}
\newcommand\fT{\mathfrak T}
\newcommand\fU{\mathfrak U}
\newcommand\fV{\mathfrak V}
\newcommand\fW{\mathfrak W}
\newcommand\fX{\mathfrak X}
\newcommand\fY{\mathfrak Y}
\newcommand\fZ{\mathfrak Z}

\newcommand\fa{\mathfrak a}
\newcommand\fb{\mathfrak b}
\newcommand\fc{\mathfrak c}
\newcommand\fd{\mathfrak d}
\newcommand\fe{\mathfrak e}
\newcommand\ff{\mathfrak f}
\newcommand\fg{\mathfrak g}
\newcommand\fh{\mathfrak h}
%\newcommand\fi{\mathfrak i}
\newcommand\fj{\mathfrak j}
\newcommand\fk{\mathfrak k}
\newcommand\fl{\mathfrak l}
\newcommand\fm{\mathfrak m}
\newcommand\fn{\mathfrak n}
\newcommand\fo{\mathfrak o}
\newcommand\fp{\mathfrak p}
\newcommand\fq{\mathfrak q}
\newcommand\fr{\mathfrak r}
\newcommand\fs{\mathfrak s}
\newcommand\ft{\mathfrak t}
\newcommand\fu{\mathfrak u}
\newcommand\fv{\mathfrak v}
\newcommand\fw{\mathfrak w}
\newcommand\fx{\mathfrak x}
\newcommand\fy{\mathfrak y}
\newcommand\fz{\mathfrak z}

\newcommand\vA{\vec A}
\newcommand\vB{\vec B}
\newcommand\vC{\vec C}
\newcommand\vD{\vec D}
\newcommand\vE{\vec E}
\newcommand\vF{\vec F}
\newcommand\vG{\vec G}
\newcommand\vH{\vec H}
\newcommand\vI{\vec I}
\newcommand\vJ{\vec J}
\newcommand\vK{\vec K}
\newcommand\vL{\vec L}
\newcommand\vM{\vec M}
\newcommand\vN{\vec N}
\newcommand\vO{\vec O}
\newcommand\vP{\vec P}
\newcommand\vQ{\vec Q}
\newcommand\vR{\vec R}
\newcommand\vS{\vec S}
\newcommand\vT{\vec T}
\newcommand\vU{\vec U}
\newcommand\vV{\vec V}
\newcommand\vW{\vec W}
\newcommand\vX{\vec X}
\newcommand\vY{\vec Y}
\newcommand\vZ{\vec Z}

\newcommand\va{\vec a}
\newcommand\vb{\vec b}
\newcommand\vc{\vec c}
\newcommand\vd{\vec d}
\newcommand\ve{\vec e}
\newcommand\vf{\vec f}
\newcommand\vg{\vec g}
\newcommand\vh{\vec h}
\newcommand\vi{\vec i}
\newcommand\vj{\vec j}
\newcommand\vk{\vec k}
\newcommand\vl{\vec l}
\newcommand\vm{\vec m}
\newcommand\vn{\vec n}
\newcommand\vo{\vec o}
\newcommand\vp{\vec p}
\newcommand\vq{\vec q}
\newcommand\vr{\vec r}
\newcommand\vs{\vec s}
\newcommand\vt{\vec t}
\newcommand\vu{\vec u}
\newcommand\vv{\vec v}
\newcommand\vw{\vec w}
\newcommand\vx{\vec x}
\newcommand\vy{\vec y}
\newcommand\vz{\vec z}

\newcommand\NN{\mathbb N}
\newcommand\ZZ{\mathbb Z}
\newcommand\QQ{\mathbb Q}
\newcommand\RR{\mathbb R}
\newcommand\CC{\mathbb C}

\newcommand{\pr}{\mathrm{P}}
\newcommand{\Vol}{\mathrm{vol}}
\newcommand\norm[1]{\left\|{#1}\right\|} 
\newcommand\sign{\mathrm{sign}}
\newcommand{\eps}{\varepsilon}
\newcommand{\abs}[1]{\left|#1\right|}
\newcommand\bc[1]{\left({#1}\right)} 
\newcommand\cbc[1]{\left\{{#1}\right\}} 
\newcommand\bcfr[2]{\bc{\frac{#1}{#2}}} 
\newcommand{\bck}[1]{\left\langle{#1}\right\rangle} 
\newcommand\brk[1]{\left\lbrack{#1}\right\rbrack} 
\newcommand\scal[2]{\bck{{#1},{#2}}} 
\newcommand{\vecone}{\mathbb{1}}
\newcommand{\tensor}{\otimes}
\newcommand{\diag}{\mathrm{diag}}
\newcommand{\ggt}{\mathrm{ggT}}
\newcommand{\kgv}{\mathrm{kgV}}

\newcommand{\Karonski}{Karo\'nski}
\newcommand{\Erdos}{Erd\H{o}s}
\newcommand{\Renyi}{R\'enyi}
\newcommand{\Lovasz}{Lov\'asz}
\newcommand{\Juhasz}{Juh\'asz}
\newcommand{\Bollobas}{Bollob\'as}
\newcommand{\Furedi}{F\"uredi}
\newcommand{\Komlos}{Koml\'os}
\newcommand{\Luczak}{\L uczak}
\newcommand{\Kucera}{Ku\v{c}era}
\newcommand{\Szemeredi}{Szemer\'edi}

\renewcommand{\ae}{\"a}
\renewcommand{\oe}{\"o}
\newcommand{\ue}{\"u}
\newcommand{\Ae}{\"A}
\newcommand{\Oe}{\"O}
\newcommand{\Ue}{\"U}

\title[Linadi]{Teilbarkeit}
\author[Amin Coja-Oghlan]{Amin Coja-Oghlan}
\institute[Frankfurt]{JWGUFFM}
\date{}

\begin{document}

\frame[plain]{\titlepage}

\begin{frame}\frametitle{Die Division}
	\begin{block}{Teilen mit Rest}
		\begin{itemize}
			\item Seien $y,z$ ganze Zahlen und $z\neq0$.
			\item Dann gibt es eindeutige ganze Zahlen $q$ und $r$, so da\ss
				\begin{align*}
					y&=q\cdot z+r\qquad\mbox{und}\qquad 0\leq r<|z|.
				\end{align*}
			\item Wir nennen $q$ den \alert{Quotienten} und $r$ den \alert{Divisionsrest}.
			\item \emph{Erinnerung:} der \alert{Betrag} von $z$ ist definiert als
				\begin{align*}
					|z|=\begin{cases}z&\mbox{ falls }z\geq0,\\-z&\mbox{ falls }z<0. \end{cases}
				\end{align*}
		\end{itemize}
	\end{block}
\end{frame}

\begin{frame}\frametitle{Die Division}
	\begin{block}{Beispiel}
		\begin{itemize}
			\item $y=17$, $z=3$:
			\item	$17=5\cdot 3+2$
			\item $q=5$, $r=2$
		\end{itemize}
	\end{block}
\end{frame}

\begin{frame}\frametitle{Die Division}
	\begin{block}{Beispiel}
		\begin{itemize}
			\item $y=-17$, $z=3$:
			\item	$-17=-6\cdot 3+1$
			\item $q=-6$, $r=1$
		\end{itemize}
	\end{block}
\end{frame}

\begin{frame}\frametitle{Die Division}
	\begin{block}{Beispiel}
		\begin{itemize}
			\item $y=17$, $z=-3$:
			\item	$17=(-5)\cdot(-3)+2$
			\item $q=-5$, $r=2$
		\end{itemize}
	\end{block}
\end{frame}

\begin{frame}\frametitle{Das Divisionsverfahren}
\begin{block}{}
\begin{itemize}
\item Angenommen $y,z\in\NN$.
	\item Tabelliere zun\ae chst die neun Vielfachen
		\begin{align*}
		1\cdot z,\quad 2\cdot z,\quad\ldots,\quad 9\cdot z
		\end{align*}
	\item Falls $y< 10\cdot z$, bestimme das maximale $q\in\{0,1,\ldots,9\}$ mit $$y\geq q\cdot z.$$
	\item Der Divisionsrest ist dann $r=y-q\cdot z$.
\end{itemize}
\end{block}
\end{frame}

\begin{frame}\frametitle{Das Divisionsverfahren}
	\begin{block}{}
		\begin{itemize}
			\item Nehmen wir nun $y\geq 10\cdot z$ an.
			\item Sei $y=\sum_{i=0}^\ell y_i10^i$ mit $y_\ell>0$ die Dezimaldarstellung.
			\item Finde das gr\oe\ss te $k>0$, so da\ss\ 
				\begin{align*}
					\sum_{i=k}^\ell y_i10^{i-k}\geq z\enspace;
				\end{align*}
				dann ist $\sum_{i=k}^\ell y_i10^{i-k}< 10z$.
			\item Wie auf der vorherigen Folie finde $q',r'$ mit
				\begin{align*}
					\sum_{i=k}^\ell y_i10^{i-k}=q'\cdot z+r',\quad 0\leq r'<z.
				\end{align*}
		\end{itemize}
	\end{block}
\end{frame}

\begin{frame}\frametitle{Das Divisionsverfahren}
	\begin{block}{}
		\begin{itemize}
			\item Wir erhalten also
				\begin{align*}
					y=10^k\cdot q'\cdot z+10^kr'+\sum_{i=0}^{k-1}y_i10^i.
				\end{align*}
			\item Wende jetzt das Divisionsverfahren rekursiv an auf
				\begin{align*}
					y'=10^kr'+\sum_{i=0}^{k-1}y_i10^i\enspace;
				\end{align*}
				wir finden $q''$ und $r<z$ mit
				\begin{align*}
					y'=q''\cdot z+r.
				\end{align*}
			\item Somit erhalten wir
				\begin{align*}
					y=(10^k\cdot q'+q'')z+r.
				\end{align*}
		\end{itemize}
	\end{block}
\end{frame}

\begin{frame}\frametitle{Das Divisionsverfahren}
	\begin{block}{Beispiel}
		\begin{tabular}{ccccccccccc}
			2&5&7&:&8&=&3&2&\mbox{ Rest }&1\\
			2&4\\
			 &1&7\\
			 &1&6\\
			 &&1
		\end{tabular}
	\end{block}
\end{frame}

\begin{frame}\frametitle{Elementare Arithmetik}
	\begin{itemize}
		\item Das Divisionsverfahren ben\oe tigt eine Anzahl von elementaren Operationen, die {\itshape im wesentlichen} linear ist in der L\ae nge der Dezimaldarstellungen der beiden Zahlen.
		\item \Ae hnliches gilt f\ue r die Addition und Subtraktion.
		\item Lediglich die Multiplikation ben\oe tigt eine quadratische Zahl elementarer Operationen.
		\item \emph{Hinweis:} es gibt bessere Multiplikationsverfahren, die im wesentlichen nur lineare Zeit ben\oe tigen.
	\end{itemize}
\end{frame}

\begin{frame}\frametitle{Der Teilbarkeitsbegriff}
\begin{block}{Definition}
\begin{itemize}
\item Eine ganze Zahl $x\neq0$ \emph{teilt} eine ganze Zahl $y$, falls es eine Zahl $q\in\ZZ$ gibt, so da\ss
	\begin{align*}
	y=q\cdot x.
	\end{align*}
\item Wir nennen dann $x$ einen \emph{Teiler} von $y$.
\item \emph{Schreibweise:} $x\mid y$ (``$x$ teilt $y$'').
\item Falls $x$ \emph{nicht} $y$ teilt, schreibt man $x\nmid y$.
\end{itemize}
\end{block}	
\end{frame}

\begin{frame}\frametitle{Der Teilbarkeitsbegriff}
	\begin{block}{Beispiele}
		\begin{itemize}
			\item $3\mid9$, $-3\mid9$, $3\mid-9$, $-3\mid-9$
			\item $9\nmid 3$, $-3\nmid 1$, $-9\nmid -3$
			\item Die Zahlen $\pm1$ teilen jede ganze Zahl.
			\item Die Zahl $0$ wird von jeder ganzen Zahl $x\neq0$ geteilt.
		\end{itemize}
	\end{block}	
\end{frame}

\begin{frame}\frametitle{Der Teilbarkeitsbegriff}
	\begin{block}{Rechenregeln}
		\begin{itemize}
			\item Teilbarkeit ist \alert{transitiv}:
				\begin{align*}
					x|y\mbox{ und }y|z\quad\Rightarrow\quad x|z.
				\end{align*}
			\item Wenn $x|y$ und $y|x$, dann gilt $x=\pm y$.
			\item Wenn $x,y>0$ und $x|y$, dann $y\geq x$.
			\item Wenn $x|y$ und $x|z$, dann $x|y+z$ und $x|y-z$.
		\end{itemize}
	\end{block}	
\end{frame}

\begin{frame}\frametitle{Der gr\oe\ss te gemeinsame Teiler}
	\begin{block}{Definition}
		\begin{itemize}
			\item Der \emph{gr\oe\ss te gemeinsame Teiler} von $x,y\in\NN$ ist definiert als
				\begin{align*}
					\ggt(x,y)=\max\cbc{z\in\NN:z\mid x\mbox{ and }z\mid y}.
				\end{align*}
			\item Das \emph{kleinste gemeinsame Vielfache} von $x,y\in\NN$ ist definiert als
				\begin{align*}
					\kgv(x,y)=\min\cbc{z\in\NN:x\mid z\mbox{ und }y\mid z}.
				\end{align*}
		\end{itemize}
	\end{block}
\end{frame}

\begin{frame}\frametitle{Der gr\oe\ss te gemeinsame Teiler}
	\begin{block}{Bemerkung}
		F\ue r alle $x,y\in\NN$ gilt	
		\begin{align*}
			x\cdot y=\ggt(x,y)\cdot\kgv(x,y).
		\end{align*}
	\end{block}
\end{frame}

\begin{frame}\frametitle{Der gr\oe\ss te gemeinsame Teiler}
	\begin{block}{Der Euklidische Algorithmus}
		\emph{Ziel:} Berechnung von $\ggt(x,y)$ f\ue r $x,y\in\NN$.
	\begin{enumerate}
		\item Falls $x<y$, vertausche $x$ und $y$.
		\item F\ue hre Divison mit Rest aus, um $q\in\NN_0$ und $z\in\{0,1,\ldots,y-1\}$ zu finden, so da\ss
			\begin{align*}
			x=q\cdot y+z.
			\end{align*}
		\item Falls $z=0$, gib $y$ als L\oe sung aus und stoppe.
		\item Sonst bestimme rekursiv $\ggt(y,z)$.
	\end{enumerate}
	\end{block}
\end{frame}

\begin{frame}\frametitle{Der gr\oe\ss te gemeinsame Teiler}
	\begin{block}{Satz}
		Der Eulidische Algorithmus gibt $\ggt(x,y)$ aus.
	\end{block}
	\begin{block}{Beweis}
		\begin{itemize}
			\item Wir nehmen $x\geq y$ an und f\ue hren Induktion nach $x+y$.
			\item Falls $x+y=2$ ist nichts zu zeigen.
			\item Falls $z=0$ gilt $y\mid x$ und somit $y=\ggt(x,y)$.
			\item Falls $z>0$ gilt $y\nmid x$ und $\ggt(y,z)\mid x$, also $\ggt(y,z)\leq\ggt(x,y)$.
			\item Au\ss erdem gilt $\ggt(x,y)\mid z$, also $\ggt(x,y)\leq\ggt(y,z)$.
			\item Folglich erhalten wir $\ggt(x,y)=\ggt(y,z)$.
		\end{itemize}
	\end{block}
\end{frame}

\begin{frame}\frametitle{Der gr\oe\ss te gemeinsame Teiler}
	\begin{block}{Beispiel}
	\begin{itemize}
	\item $x=154$, $y=30$.
	\item $154=5\cdot 30+4$.
	\item $30=7\cdot 4+2$.
	\item $4=2\cdot2+0$.
	\item $\Rightarrow$ $\ggt(154,30)=2$.
	\end{itemize}
	\end{block}
\end{frame}

\begin{frame}\frametitle{Der gr\oe\ss te gemeinsame Teiler}
	\begin{block}{Satz}
		Angenommen $x,y\leq10^\ell$ f\ue r eine Zahl $\ell$.
		Dann f\ue hrt der Euklidische Algorithmus h\oe chstens $8(\ell+1)$ Divisionen durch.
	\end{block}
	\begin{block}{Beweis}
		\begin{itemize}
			\item Wir betrachten eine Division
				\begin{align*}
					x=q\cdot y+z&&(z<y\leq x),
				\end{align*}
				die der Algorithmus durchf\ue hrt.
			\item Falls $y\leq x/2$, erhalten wir $z<x/2$.
			\item Falls $y>x/2$, folgt aus $q\geq1$, da\ss\ $z=x-qy\leq x/2$.
			\item Foglich halbiert sich die gr\oe ssere Zahl alle zwei Schritte.
		\end{itemize}
	\end{block}
\end{frame}

\begin{frame}\frametitle{Der gr\oe\ss te gemeinsame Teiler}
	\begin{block}{Satz}
		Angenommen $0<x,y\leq10^\ell$ f\ue r eine Zahl $\ell$.
		Dann f\ue hrt der Euklidische Algorithmus h\oe chstens $8(\ell+1)$ Divisionen durch.
	\end{block}
	\begin{block}{}
		\begin{itemize}
			\item Der $\ggt(x,y)$ kann effizient berechnet werden.
			\item Das $\kgv(x,y)$ kann effizient berechnet werden.
		\end{itemize}
	\end{block}
\end{frame}

\begin{frame}\frametitle{Der gr\oe\ss te gemeinsame Teiler}
	\begin{block}{Satz}
		Seien $x,y,r\in\NN$.
		Es gibt $a,b\in\ZZ$ mit $a\cdot x+b\cdot y=r$ genau dann, wenn $\ggt(x,y)\mid r.$
	\end{block}
	\begin{block}{Beweis}
		\begin{itemize}
			\item Induktion nach $x+y$; im Fall $x=y=1$ ist nichts zu zeigen.
			\item Falls $r=ax+by$, so gilt offenbar $\ggt(x,y)\mid r$.
			\item Nehmen wir nun an, da\ss\ $r=\ggt(x,y)$ und $x\geq y$.
			\item Falls $y\mid x$, w\ae hle $a=0,b=1$.
			\item Sonst finde $q\geq0$, $0<z<y$ mit $x=qy+z$.
			\item Es gilt $r=\ggt(x,y)=\ggt(y,z)$ und $y+z<x+y$.
			\item Nach Induktion gibt es $c,d\in\ZZ$ mit $r=cy+dz$.
			\item Wir erhalten $r=cy+d(x-qy)=(c-q)y+dx$.
		\end{itemize}
	\end{block}
\end{frame}

\begin{frame}\frametitle{Der gr\oe\ss te gemeinsame Teiler}
	\begin{block}{Satz}
		Seien $x,y,r\in\NN$.
		Es gibt $a,b\in\ZZ$ mit $a\cdot x+b\cdot y=r$ genau dann, wenn $\ggt(x,y)\mid r.$
	\end{block}
	\begin{block}{}
		\begin{itemize}
			\item Der Beweis liefert ein effizientes Verfahren zur Berechnung von $a,b$.
			\item Dieses Verfahren ist als \emph{erweiterter Euklidischer Algorithmus} bekannt.
		\end{itemize}
	\end{block}
\end{frame}

\begin{frame}\frametitle{Der gr\oe\ss te gemeinsame Teiler}
	\begin{block}{Beispiel}
		\begin{itemize}
			\item $x=154,t=30,z=\ggt(x,y)=2$.
				\begin{align*}
					154&=5\cdot 30+4&\Rightarrow&&4&=154-5\cdot 30\\
					30&=7\cdot 4+2&\Rightarrow&&2&=30-7\cdot 4\\
					4&=2\cdot 2+0
				\end{align*}
			\item Also erhalten wir
				\begin{align*}
					2=30-7\cdot 4=30-7\cdot(154-5\cdot30)=36\cdot30-7\cdot 154.
				\end{align*}
		\end{itemize}
	\end{block}
\end{frame}

\begin{frame}\frametitle{Zusammenfassung}
\begin{itemize}
	\item Divisionen mit Rest k\oe nnen effizient durchgef\ue hrt werden (`Schulmethode').
	\item Begriffe Teilbarkeit, ggT, kgV
	\item Der Euklidische Algorithmus
\end{itemize}
\end{frame}


\end{document}
