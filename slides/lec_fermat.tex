\documentclass{beamer}
\usepackage{amsmath,graphics}
\usepackage{amssymb}

\usetheme{default}
\usepackage{xcolor}

\definecolor{solarizedBase03}{HTML}{002B36}
\definecolor{solarizedBase02}{HTML}{073642}
\definecolor{solarizedBase01}{HTML}{586e75}
\definecolor{solarizedBase00}{HTML}{657b83}
\definecolor{solarizedBase0}{HTML}{839496}
\definecolor{solarizedBase1}{HTML}{93a1a1}
\definecolor{solarizedBase2}{HTML}{EEE8D5}
\definecolor{solarizedBase3}{HTML}{FDF6E3}
\definecolor{solarizedYellow}{HTML}{B58900}
\definecolor{solarizedOrange}{HTML}{CB4B16}
\definecolor{solarizedRed}{HTML}{DC322F}
\definecolor{solarizedMagenta}{HTML}{D33682}
\definecolor{solarizedViolet}{HTML}{6C71C4}
%\definecolor{solarizedBlue}{HTML}{268BD2}
\definecolor{solarizedBlue}{HTML}{134676}
\definecolor{solarizedCyan}{HTML}{2AA198}
\definecolor{solarizedGreen}{HTML}{859900}
\definecolor{myBlue}{HTML}{162DB0}%{261CA4}
\setbeamercolor*{item}{fg=myBlue}
\setbeamercolor{normal text}{fg=solarizedBase03, bg=solarizedBase3}
\setbeamercolor{alerted text}{fg=myBlue}
\setbeamercolor{example text}{fg=myBlue, bg=solarizedBase3}
\setbeamercolor*{frametitle}{fg=solarizedRed}
\setbeamercolor*{title}{fg=solarizedRed}
\setbeamercolor{block title}{fg=myBlue, bg=solarizedBase3}
\setbeameroption{hide notes}
\setbeamertemplate{note page}[plain]
\beamertemplatenavigationsymbolsempty
\usefonttheme{professionalfonts}
\usefonttheme{serif}

\usepackage{fourier}

\def\vec#1{\mathchoice{\mbox{\boldmath$\displaystyle#1$}}
{\mbox{\boldmath$\textstyle#1$}}
{\mbox{\boldmath$\scriptstyle#1$}}
{\mbox{\boldmath$\scriptscriptstyle#1$}}}
\definecolor{OwnGrey}{rgb}{0.560,0.000,0.000} % #999999
\definecolor{OwnBlue}{rgb}{0.121,0.398,0.711} % #1f64b0
\definecolor{red4}{rgb}{0.5,0,0}
\definecolor{blue4}{rgb}{0,0,0.5}
\definecolor{Blue}{rgb}{0,0,0.66}
\definecolor{LightBlue}{rgb}{0.9,0.9,1}
\definecolor{Green}{rgb}{0,0.5,0}
\definecolor{LightGreen}{rgb}{0.9,1,0.9}
\definecolor{Red}{rgb}{0.9,0,0}
\definecolor{LightRed}{rgb}{1,0.9,0.9}
\definecolor{White}{gray}{1}
\definecolor{Black}{gray}{0}
\definecolor{LightGray}{gray}{0.8}
\definecolor{Orange}{rgb}{0.1,0.2,1}
\setbeamerfont{sidebar right}{size=\scriptsize}
\setbeamercolor{sidebar right}{fg=Black}

\renewcommand{\emph}[1]{{\textcolor{solarizedRed}{\itshape #1}}}

\newcommand\cA{\mathcal A}
\newcommand\cB{\mathcal B}
\newcommand\cC{\mathcal C}
\newcommand\cD{\mathcal D}
\newcommand\cE{\mathcal E}
\newcommand\cF{\mathcal F}
\newcommand\cG{\mathcal G}
\newcommand\cH{\mathcal H}
\newcommand\cI{\mathcal I}
\newcommand\cJ{\mathcal J}
\newcommand\cK{\mathcal K}
\newcommand\cL{\mathcal L}
\newcommand\cM{\mathcal M}
\newcommand\cN{\mathcal N}
\newcommand\cO{\mathcal O}
\newcommand\cP{\mathcal P}
\newcommand\cQ{\mathcal Q}
\newcommand\cR{\mathcal R}
\newcommand\cS{\mathcal S}
\newcommand\cT{\mathcal T}
\newcommand\cU{\mathcal U}
\newcommand\cV{\mathcal V}
\newcommand\cW{\mathcal W}
\newcommand\cX{\mathcal X}
\newcommand\cY{\mathcal Y}
\newcommand\cZ{\mathcal Z}

\newcommand\fA{\mathfrak A}
\newcommand\fB{\mathfrak B}
\newcommand\fC{\mathfrak C}
\newcommand\fD{\mathfrak D}
\newcommand\fE{\mathfrak E}
\newcommand\fF{\mathfrak F}
\newcommand\fG{\mathfrak G}
\newcommand\fH{\mathfrak H}
\newcommand\fI{\mathfrak I}
\newcommand\fJ{\mathfrak J}
\newcommand\fK{\mathfrak K}
\newcommand\fL{\mathfrak L}
\newcommand\fM{\mathfrak M}
\newcommand\fN{\mathfrak N}
\newcommand\fO{\mathfrak O}
\newcommand\fP{\mathfrak P}
\newcommand\fQ{\mathfrak Q}
\newcommand\fR{\mathfrak R}
\newcommand\fS{\mathfrak S}
\newcommand\fT{\mathfrak T}
\newcommand\fU{\mathfrak U}
\newcommand\fV{\mathfrak V}
\newcommand\fW{\mathfrak W}
\newcommand\fX{\mathfrak X}
\newcommand\fY{\mathfrak Y}
\newcommand\fZ{\mathfrak Z}

\newcommand\fa{\mathfrak a}
\newcommand\fb{\mathfrak b}
\newcommand\fc{\mathfrak c}
\newcommand\fd{\mathfrak d}
\newcommand\fe{\mathfrak e}
\newcommand\ff{\mathfrak f}
\newcommand\fg{\mathfrak g}
\newcommand\fh{\mathfrak h}
%\newcommand\fi{\mathfrak i}
\newcommand\fj{\mathfrak j}
\newcommand\fk{\mathfrak k}
\newcommand\fl{\mathfrak l}
\newcommand\fm{\mathfrak m}
\newcommand\fn{\mathfrak n}
\newcommand\fo{\mathfrak o}
\newcommand\fp{\mathfrak p}
\newcommand\fq{\mathfrak q}
\newcommand\fr{\mathfrak r}
\newcommand\fs{\mathfrak s}
\newcommand\ft{\mathfrak t}
\newcommand\fu{\mathfrak u}
\newcommand\fv{\mathfrak v}
\newcommand\fw{\mathfrak w}
\newcommand\fx{\mathfrak x}
\newcommand\fy{\mathfrak y}
\newcommand\fz{\mathfrak z}

\newcommand\vA{\vec A}
\newcommand\vB{\vec B}
\newcommand\vC{\vec C}
\newcommand\vD{\vec D}
\newcommand\vE{\vec E}
\newcommand\vF{\vec F}
\newcommand\vG{\vec G}
\newcommand\vH{\vec H}
\newcommand\vI{\vec I}
\newcommand\vJ{\vec J}
\newcommand\vK{\vec K}
\newcommand\vL{\vec L}
\newcommand\vM{\vec M}
\newcommand\vN{\vec N}
\newcommand\vO{\vec O}
\newcommand\vP{\vec P}
\newcommand\vQ{\vec Q}
\newcommand\vR{\vec R}
\newcommand\vS{\vec S}
\newcommand\vT{\vec T}
\newcommand\vU{\vec U}
\newcommand\vV{\vec V}
\newcommand\vW{\vec W}
\newcommand\vX{\vec X}
\newcommand\vY{\vec Y}
\newcommand\vZ{\vec Z}

\newcommand\va{\vec a}
\newcommand\vb{\vec b}
\newcommand\vc{\vec c}
\newcommand\vd{\vec d}
\newcommand\ve{\vec e}
\newcommand\vf{\vec f}
\newcommand\vg{\vec g}
\newcommand\vh{\vec h}
\newcommand\vi{\vec i}
\newcommand\vj{\vec j}
\newcommand\vk{\vec k}
\newcommand\vl{\vec l}
\newcommand\vm{\vec m}
\newcommand\vn{\vec n}
\newcommand\vo{\vec o}
\newcommand\vp{\vec p}
\newcommand\vq{\vec q}
\newcommand\vr{\vec r}
\newcommand\vs{\vec s}
\newcommand\vt{\vec t}
\newcommand\vu{\vec u}
\newcommand\vv{\vec v}
\newcommand\vw{\vec w}
\newcommand\vx{\vec x}
\newcommand\vy{\vec y}
\newcommand\vz{\vec z}

\renewcommand\AA{\mathbb A}
\newcommand\NN{\mathbb N}
\newcommand\ZZ{\mathbb Z}
\newcommand\PP{\mathbb P}
\newcommand\QQ{\mathbb Q}
\newcommand\RR{\mathbb R}
\renewcommand\SS{\mathbb S}
\newcommand\CC{\mathbb C}

\newcommand{\ord}{\mathrm{ord}}
\newcommand{\id}{\mathrm{id}}
\newcommand{\pr}{\mathrm{P}}
\newcommand{\Vol}{\mathrm{vol}}
\newcommand\norm[1]{\left\|{#1}\right\|} 
\newcommand\sign{\mathrm{sign}}
\newcommand{\eps}{\varepsilon}
\newcommand{\abs}[1]{\left|#1\right|}
\newcommand\bc[1]{\left({#1}\right)} 
\newcommand\cbc[1]{\left\{{#1}\right\}} 
\newcommand\bcfr[2]{\bc{\frac{#1}{#2}}} 
\newcommand{\bck}[1]{\left\langle{#1}\right\rangle} 
\newcommand\brk[1]{\left\lbrack{#1}\right\rbrack} 
\newcommand\scal[2]{\bck{{#1},{#2}}} 
\newcommand{\vecone}{\mathbb{1}}
\newcommand{\tensor}{\otimes}
\newcommand{\diag}{\mathrm{diag}}
\newcommand{\ggt}{\mathrm{ggT}}
\newcommand{\kgv}{\mathrm{kgV}}

\newcommand{\Karonski}{Karo\'nski}
\newcommand{\Erdos}{Erd\H{o}s}
\newcommand{\Renyi}{R\'enyi}
\newcommand{\Lovasz}{Lov\'asz}
\newcommand{\Juhasz}{Juh\'asz}
\newcommand{\Bollobas}{Bollob\'as}
\newcommand{\Furedi}{F\"uredi}
\newcommand{\Komlos}{Koml\'os}
\newcommand{\Luczak}{\L uczak}
\newcommand{\Kucera}{Ku\v{c}era}
\newcommand{\Szemeredi}{Szemer\'edi}

\renewcommand{\ae}{\"a}
\renewcommand{\oe}{\"o}
\newcommand{\ue}{\"u}
\newcommand{\Ae}{\"A}
\newcommand{\Oe}{\"O}
\newcommand{\Ue}{\"U}

\title[Linadi]{Der Fermat-Test}
\author[Amin Coja-Oghlan]{Amin Coja-Oghlan}
\institute[Frankfurt]{JWGUFFM}
\date{}

\begin{document}

\frame[plain]{\titlepage}

\begin{frame}\frametitle{Der Fermat-Test}
	\begin{block}{Primzahltests}
		\begin{itemize}
			\item \emph{Gegeben:} eine Zahl $n\in\NN$.
			\item \emph{Frage:} ist $n$ eine Primzahl?
		\end{itemize}
	\end{block}
\end{frame}

\begin{frame}\frametitle{Der Fermat-Test}
	\begin{block}{Die Brechstange}
		\begin{itemize}
			\item F\ue r jede ganze Zahl $2\leq z\leq\sqrt n$
			\item $\qquad$dividiere $n$ mit Rest durch $z$:
				\begin{align*}
					n=q\cdot z+r&&0\leq r<z.
				\end{align*}
			\item $\qquad$Falls $r=0$, gib ``Nein'' aus und halte.
			\item Gib ``Ja'' aus.
		\end{itemize}
	\end{block}
\end{frame}

\begin{frame}\frametitle{Der Fermat-Test}
	\begin{block}{Laufzeit}
		\begin{itemize}
			\item Es werden bis zu $\sqrt n$ Divisionen ausgef\ue hrt.
			\item In der Kryptographie liegt $n$ in der Gr\oe\ss enordnung $10^{1000}$.
			\item Also werden etwa $10^{500}$ Divisionen gebraucht.
			\item Bei $10^{10}$ Divisionen pro Sekunde dauert das
				\begin{align*}
					10^{482}&&\mbox{Jahre.}
				\end{align*}
			\item \itshape Leider gl\ue ht die Sonne schon in ca.~$10^{10}$ Jahren aus\dots
		\end{itemize}
	\end{block}
\end{frame}

\begin{frame}\frametitle{Der Fermat-Test}
	\begin{block}{Exponentielle Laufzeit}
		\begin{itemize}
			\item Zur Darstellung der Zahl $n$ brauchen wir etwa
				\begin{align*}
					\ell=\log_{10}n
				\end{align*}
				Ziffern.
			\item Wenn $n$ Gr\oe\ss enordnung $10^{1000}$ hat, ist die Darstellungsl\ae nge also etwa $\ell\approx1000$.
			\item Die Laufzeit des Brechstange-Algorithmus ist \alert{exponentiell} in der Darstellungsl\ae nge:
				\begin{align*}
					\sqrt n=\sqrt{10}\,^{\ell}.
				\end{align*}
		\end{itemize}
	\end{block}
\end{frame}

\begin{frame}\frametitle{Der Fermat-Test}
	\begin{block}{Polynomielle Laufzeit}
		\begin{itemize}
			\item F\ue r ein \emph{effizientes} Verfahren fordern wir eine polynomielle Laufzeit, also z.B.
				\begin{align*}
					\ell,\quad\ell^2,\quad\ell^3&&\mbox{elementare Rechenschritte.}
				\end{align*}
			\item Mehr dazu lernen Sie in der \alert{Komplexit\ae tstheorie}.
			\item Der Fermat-Test ist ein erster Schritt auf dem Weg zu einem effizienten Primzahltest.
		\end{itemize}
	\end{block}
\end{frame}

\begin{frame}\frametitle{Der Fermat-Test}
	\begin{block}{Lemma}
		Sei $n\geq2$.
		Wenn f\ue r $1\leq a<n$ ein $r\in\NN$ mit 
		\begin{align*}
			a^r\equiv1\mod n
		\end{align*}
		existiert, dann gilt $a+n\ZZ\in\ZZ_n^\times$.
	\end{block}
	\begin{block}{Beweis}
		Es gilt
		\begin{align*}
			(a+n\ZZ)\cdot(a+n\ZZ)^{r-1}=a^r+n\ZZ=1+n\ZZ.
		\end{align*}
		Also ist $(a+n\ZZ)^{r-1}$ das Inverse zu $a+n\ZZ$.
	\end{block}
\end{frame}

\begin{frame}\frametitle{Der Fermat-Test}
	\begin{block}{Lemma}
		Sei $n\geq2$.
		Wenn 
		\begin{align*}
			a^{n-1}\equiv 1\mod n&&\mbox{f\ue r alle }1\leq a<n,
		\end{align*}
		dann ist $n$ eine Primzahl.
	\end{block}
	\begin{block}{Beweis}
		\begin{itemize}
			\item In diesem Fall gilt $|\ZZ_n^\times|=n-1$.
			\item Also $\ZZ_n^\times=\ZZ_n\setminus\cbc0$.
			\item Folglich ist $\ZZ_n$ ein K\oe rper.
			\item Also ist $n$ eine Primzahl.
		\end{itemize}
	\end{block}
\end{frame}

\begin{frame}\frametitle{Der Fermat-Test}
	\begin{block}{Definition}
		Eine Zahl $1\leq a<n$ hei\ss t \emph{F-Zeuge} von $n\geq2$, falls
		\begin{align*}
			a^{n-1}\not\equiv1\mod n.
		\end{align*}
	\end{block}
	\begin{block}{Korollar}
		Eine Zahl $n\geq2$ hat genau dann einen F-Zeugen, wenn $n$ zusammengestzt ist.
	\end{block}
\end{frame}

\begin{frame}\frametitle{Der Fermat-Test}
	\begin{block}{Algorithmus {\tt Fermat}$(n)$}
		\begin{enumerate}
			\item W\ae hle eine Zahl $a\in\{2,3,\ldots,n-2\}$ zuf\ae llig.
			\item Falls
				\begin{align*}
					a^{n-1}\equiv1\mod n
			\end{align*}
			gib ``Primzahl'' aus.
		\item Sonst gib ``Keine Primzahl'' aus.
		\end{enumerate}
	\end{block}
\end{frame}

\begin{frame}\frametitle{Der Fermat-Test}
	\begin{block}{Anmerkungen}
		\begin{itemize}
			\item Der Fermat-Test ist ein \emph{Monte-Carlo-Algorithmus}.
			\item D.h.\ er verwendet Randomisierung.
			\item Der Algorithmus kann ``eine M\ue nze werfen'' und ob die Antwort richtig ist, h\ae ngt vom Ergebnis der M\ue nzw\ue rfe ab.
			\item Also stehen wir vor der Aufgabe, die Erfolgswahrscheinlichkeit des Algorithmus zu analysieren.
			\item \itshape Leider werden wir feststellen, da\ss\ der Fermattest dabei schlecht abschneidet\dots
		\end{itemize}
	\end{block}
\end{frame}

\begin{frame}\frametitle{Der Fermat-Test}
	\begin{block}{Anmerkungen}
		\begin{itemize}
			\item Der Algorithmus ist \emph{effizient}.
			\item Schritt~1 f\ue hrt etwa $\log_2n$ M\ue nzw\ue rfe durch.
			\item In Schritt~2 verwenden wir das Verfahren zum schnellen modularen Potenzieren.
		\end{itemize}
	\end{block}
\end{frame}

\begin{frame}\frametitle{Der Fermat-Test}
	\begin{block}{Proposition}
		Wenn der Fermattest ``Keine Primzahl'' ausgibt, ist $n$ zusammengesetzt.
	\end{block}
	\begin{block}{Anmerkung}
	\begin{itemize}
	\item Wenn der Fermattest ``Primzahl'' ausgibt, \alert{k\oe nnte} $n$ trotzdem zusammengesetzt sein.	
	\item Wir m\ue ssen also die Fehlerwahrscheinlichkeit absch\ae tzen.
	\end{itemize}
	\end{block}
\end{frame}

\begin{frame}\frametitle{Der Fermat-Test}
	\begin{block}{Satz}
		Wenn $n>3$ zusammengesetzt ist und einen F-Zeugen $a$ mit $\ggt(a,n)=1$ besitzt, dann gibt der Fermattest mit Wahrscheinlichkeit mindestens $\frac{1}{2}$ ``keine Primzahl'' aus.
	\end{block}
	\begin{overprint}
		\onslide<1>
		\begin{block}{Beweis}
			\begin{itemize}
				\item Sei $ L=\cbc{a+n\ZZ:1\leq a<n\mbox{ ist kein F-Zeuge und }\ggt(a,n)=1}.  $
				\item Dann ist $L$ eine Untergruppe von $\ZZ_n^\times$.
				\item Au\ss erdem gilt $L\neq\ZZ_n^\times$.
				\item Aus dem Satz von Lagrange folgt also $|L|\leq|\ZZ_n^\times|/2$.
				\item Die Wahrscheinlichkeit, da\ss\ Schritt~1 ein $a$ mit
					\begin{align*}
						\ggt(a,n)>1&&\mbox{oder}&&a+n\ZZ\in L
					\end{align*}
					trifft, ist also mindestens $1/2$.
			\end{itemize}
		\end{block}
	\end{overprint}
\end{frame}

\begin{frame}\frametitle{Der Fermat-Test}
	\begin{block}{Die Erfolgswahrscheinlichkeit}
			\begin{itemize}
				\item Angenommen $n$ ist zusammengesetzt und hat einen F-Zeugen $a$ mit $\ggt(a,n)=1$.
				\item Der Fermattest gibt mit Wahrscheinlichkeit $\geq\frac{1}{2}$ ``keine Primzahl'' aus.
				\item Wenn wir den Fermattest 100 mal mit unabh\ae ngigen M\ue nzw\ue rfen wiederholen, ist die Wahrscheinlichkeit, jedesmal f\ae lschlich die Antwort ``Primzahl'' zu erhalten, also
					\begin{align*}
						\leq 2^{-100}\approx 10^{-30}.
					\end{align*}
				\item Wenn hingegen $n$ prim ist, gibt der Fermattest \alert{immer} Primzahl aus (``einseitiger Fehler'').
			\end{itemize}
		\end{block}
\end{frame}

\begin{frame}{Der Fermat-Test}
	\begin{block}{Carmichaelzahlen}
		\begin{itemize}
			\item Jede zusammengestzte Zahl hat einen F-Zeugen.
			\item Aber nicht jede zusammengesetzte Zahl $n$ hat einen F-Zeugen $a$ mit $\ggt(a,n)=1$.
			\item Solche Zahlen nennt man Carmichaelzahlen.
			\item Sie werden vom Fermattest m\oe glicherweise mit sehr hoher Wahrscheinlichkeit f\ae lschlich f\ue r Primzahlen gehalten.
			\item Der Fermattest ist also \emph{kein zuverl\ae ssiger Primzahltest!}
		\end{itemize}
	\end{block}
\end{frame}

\begin{frame}{Zusammenfassung}
		\begin{itemize}
			\item Der naive Primzahltest ``Teiler durchprobieren'' ist hoffnungslos ineffizient.
			\item Der Fermattest kann feststellen, da\ss\ eine Zahl zusammengesetzt ist, ohne einen Teiler der Zahl zu bestimmen.
			\item Der Fermattest ist ein Monte-Carlo-Algorithmus.
			\item Leider bei\ss t er sich aber an Carmichaelzahlen die Z\ae hne aus.
		\end{itemize}
\end{frame}


\end{document}
