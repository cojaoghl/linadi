\documentclass{beamer}
\usepackage{amsmath,graphics}
\usepackage{amssymb}

\usetheme{default}
\usepackage{xcolor}

\definecolor{solarizedBase03}{HTML}{002B36}
\definecolor{solarizedBase02}{HTML}{073642}
\definecolor{solarizedBase01}{HTML}{586e75}
\definecolor{solarizedBase00}{HTML}{657b83}
\definecolor{solarizedBase0}{HTML}{839496}
\definecolor{solarizedBase1}{HTML}{93a1a1}
\definecolor{solarizedBase2}{HTML}{EEE8D5}
\definecolor{solarizedBase3}{HTML}{FDF6E3}
\definecolor{solarizedYellow}{HTML}{B58900}
\definecolor{solarizedOrange}{HTML}{CB4B16}
\definecolor{solarizedRed}{HTML}{DC322F}
\definecolor{solarizedMagenta}{HTML}{D33682}
\definecolor{solarizedViolet}{HTML}{6C71C4}
%\definecolor{solarizedBlue}{HTML}{268BD2}
\definecolor{solarizedBlue}{HTML}{134676}
\definecolor{solarizedCyan}{HTML}{2AA198}
\definecolor{solarizedGreen}{HTML}{859900}
\definecolor{myBlue}{HTML}{162DB0}%{261CA4}
\setbeamercolor*{item}{fg=myBlue}
\setbeamercolor{normal text}{fg=solarizedBase03, bg=solarizedBase3}
\setbeamercolor{alerted text}{fg=myBlue}
\setbeamercolor{example text}{fg=myBlue, bg=solarizedBase3}
\setbeamercolor*{frametitle}{fg=solarizedRed}
\setbeamercolor*{title}{fg=solarizedRed}
\setbeamercolor{block title}{fg=myBlue, bg=solarizedBase3}
\setbeameroption{hide notes}
\setbeamertemplate{note page}[plain]
\beamertemplatenavigationsymbolsempty
\usefonttheme{professionalfonts}
\usefonttheme{serif}

\usepackage{fourier}

\def\vec#1{\mathchoice{\mbox{\boldmath$\displaystyle#1$}}
{\mbox{\boldmath$\textstyle#1$}}
{\mbox{\boldmath$\scriptstyle#1$}}
{\mbox{\boldmath$\scriptscriptstyle#1$}}}
\definecolor{OwnGrey}{rgb}{0.560,0.000,0.000} % #999999
\definecolor{OwnBlue}{rgb}{0.121,0.398,0.711} % #1f64b0
\definecolor{red4}{rgb}{0.5,0,0}
\definecolor{blue4}{rgb}{0,0,0.5}
\definecolor{Blue}{rgb}{0,0,0.66}
\definecolor{LightBlue}{rgb}{0.9,0.9,1}
\definecolor{Green}{rgb}{0,0.5,0}
\definecolor{LightGreen}{rgb}{0.9,1,0.9}
\definecolor{Red}{rgb}{0.9,0,0}
\definecolor{LightRed}{rgb}{1,0.9,0.9}
\definecolor{White}{gray}{1}
\definecolor{Black}{gray}{0}
\definecolor{LightGray}{gray}{0.8}
\definecolor{Orange}{rgb}{0.1,0.2,1}
\setbeamerfont{sidebar right}{size=\scriptsize}
\setbeamercolor{sidebar right}{fg=Black}

\renewcommand{\emph}[1]{{\textcolor{solarizedRed}{\itshape #1}}}

\newcommand\cA{\mathcal A}
\newcommand\cB{\mathcal B}
\newcommand\cC{\mathcal C}
\newcommand\cD{\mathcal D}
\newcommand\cE{\mathcal E}
\newcommand\cF{\mathcal F}
\newcommand\cG{\mathcal G}
\newcommand\cH{\mathcal H}
\newcommand\cI{\mathcal I}
\newcommand\cJ{\mathcal J}
\newcommand\cK{\mathcal K}
\newcommand\cL{\mathcal L}
\newcommand\cM{\mathcal M}
\newcommand\cN{\mathcal N}
\newcommand\cO{\mathcal O}
\newcommand\cP{\mathcal P}
\newcommand\cQ{\mathcal Q}
\newcommand\cR{\mathcal R}
\newcommand\cS{\mathcal S}
\newcommand\cT{\mathcal T}
\newcommand\cU{\mathcal U}
\newcommand\cV{\mathcal V}
\newcommand\cW{\mathcal W}
\newcommand\cX{\mathcal X}
\newcommand\cY{\mathcal Y}
\newcommand\cZ{\mathcal Z}

\newcommand\fA{\mathfrak A}
\newcommand\fB{\mathfrak B}
\newcommand\fC{\mathfrak C}
\newcommand\fD{\mathfrak D}
\newcommand\fE{\mathfrak E}
\newcommand\fF{\mathfrak F}
\newcommand\fG{\mathfrak G}
\newcommand\fH{\mathfrak H}
\newcommand\fI{\mathfrak I}
\newcommand\fJ{\mathfrak J}
\newcommand\fK{\mathfrak K}
\newcommand\fL{\mathfrak L}
\newcommand\fM{\mathfrak M}
\newcommand\fN{\mathfrak N}
\newcommand\fO{\mathfrak O}
\newcommand\fP{\mathfrak P}
\newcommand\fQ{\mathfrak Q}
\newcommand\fR{\mathfrak R}
\newcommand\fS{\mathfrak S}
\newcommand\fT{\mathfrak T}
\newcommand\fU{\mathfrak U}
\newcommand\fV{\mathfrak V}
\newcommand\fW{\mathfrak W}
\newcommand\fX{\mathfrak X}
\newcommand\fY{\mathfrak Y}
\newcommand\fZ{\mathfrak Z}

\newcommand\fa{\mathfrak a}
\newcommand\fb{\mathfrak b}
\newcommand\fc{\mathfrak c}
\newcommand\fd{\mathfrak d}
\newcommand\fe{\mathfrak e}
\newcommand\ff{\mathfrak f}
\newcommand\fg{\mathfrak g}
\newcommand\fh{\mathfrak h}
%\newcommand\fi{\mathfrak i}
\newcommand\fj{\mathfrak j}
\newcommand\fk{\mathfrak k}
\newcommand\fl{\mathfrak l}
\newcommand\fm{\mathfrak m}
\newcommand\fn{\mathfrak n}
\newcommand\fo{\mathfrak o}
\newcommand\fp{\mathfrak p}
\newcommand\fq{\mathfrak q}
\newcommand\fr{\mathfrak r}
\newcommand\fs{\mathfrak s}
\newcommand\ft{\mathfrak t}
\newcommand\fu{\mathfrak u}
\newcommand\fv{\mathfrak v}
\newcommand\fw{\mathfrak w}
\newcommand\fx{\mathfrak x}
\newcommand\fy{\mathfrak y}
\newcommand\fz{\mathfrak z}

\newcommand\vA{\vec A}
\newcommand\vB{\vec B}
\newcommand\vC{\vec C}
\newcommand\vD{\vec D}
\newcommand\vE{\vec E}
\newcommand\vF{\vec F}
\newcommand\vG{\vec G}
\newcommand\vH{\vec H}
\newcommand\vI{\vec I}
\newcommand\vJ{\vec J}
\newcommand\vK{\vec K}
\newcommand\vL{\vec L}
\newcommand\vM{\vec M}
\newcommand\vN{\vec N}
\newcommand\vO{\vec O}
\newcommand\vP{\vec P}
\newcommand\vQ{\vec Q}
\newcommand\vR{\vec R}
\newcommand\vS{\vec S}
\newcommand\vT{\vec T}
\newcommand\vU{\vec U}
\newcommand\vV{\vec V}
\newcommand\vW{\vec W}
\newcommand\vX{\vec X}
\newcommand\vY{\vec Y}
\newcommand\vZ{\vec Z}

\newcommand\va{\vec a}
\newcommand\vb{\vec b}
\newcommand\vc{\vec c}
\newcommand\vd{\vec d}
\newcommand\ve{\vec e}
\newcommand\vf{\vec f}
\newcommand\vg{\vec g}
\newcommand\vh{\vec h}
\newcommand\vi{\vec i}
\newcommand\vj{\vec j}
\newcommand\vk{\vec k}
\newcommand\vl{\vec l}
\newcommand\vm{\vec m}
\newcommand\vn{\vec n}
\newcommand\vo{\vec o}
\newcommand\vp{\vec p}
\newcommand\vq{\vec q}
\newcommand\vr{\vec r}
\newcommand\vs{\vec s}
\newcommand\vt{\vec t}
\newcommand\vu{\vec u}
\newcommand\vv{\vec v}
\newcommand\vw{\vec w}
\newcommand\vx{\vec x}
\newcommand\vy{\vec y}
\newcommand\vz{\vec z}

\renewcommand\AA{\mathbb A}
\newcommand\NN{\mathbb N}
\newcommand\ZZ{\mathbb Z}
\newcommand\PP{\mathbb P}
\newcommand\QQ{\mathbb Q}
\newcommand\RR{\mathbb R}
\renewcommand\SS{\mathbb S}
\newcommand\CC{\mathbb C}

\newcommand{\ord}{\mathrm{ord}}
\newcommand{\id}{\mathrm{id}}
\newcommand{\pr}{\mathrm{P}}
\newcommand{\Vol}{\mathrm{vol}}
\newcommand\norm[1]{\left\|{#1}\right\|} 
\newcommand\sign{\mathrm{sign}}
\newcommand{\eps}{\varepsilon}
\newcommand{\abs}[1]{\left|#1\right|}
\newcommand\bc[1]{\left({#1}\right)} 
\newcommand\cbc[1]{\left\{{#1}\right\}} 
\newcommand\bcfr[2]{\bc{\frac{#1}{#2}}} 
\newcommand{\bck}[1]{\left\langle{#1}\right\rangle} 
\newcommand\brk[1]{\left\lbrack{#1}\right\rbrack} 
\newcommand\scal[2]{\bck{{#1},{#2}}} 
\newcommand{\vecone}{\mathbb{1}}
\newcommand{\tensor}{\otimes}
\newcommand{\diag}{\mathrm{diag}}
\newcommand{\ggt}{\mathrm{ggT}}
\newcommand{\kgv}{\mathrm{kgV}}
\newcommand{\trans}{\top}

\newcommand{\Karonski}{Karo\'nski}
\newcommand{\Erdos}{Erd\H{o}s}
\newcommand{\Renyi}{R\'enyi}
\newcommand{\Lovasz}{Lov\'asz}
\newcommand{\Juhasz}{Juh\'asz}
\newcommand{\Bollobas}{Bollob\'as}
\newcommand{\Furedi}{F\"uredi}
\newcommand{\Komlos}{Koml\'os}
\newcommand{\Luczak}{\L uczak}
\newcommand{\Kucera}{Ku\v{c}era}
\newcommand{\Szemeredi}{Szemer\'edi}

\renewcommand{\ae}{\"a}
\renewcommand{\oe}{\"o}
\newcommand{\ue}{\"u}
\newcommand{\Ae}{\"A}
\newcommand{\Oe}{\"O}
\newcommand{\Ue}{\"U}

\newcommand{\im}{\mathrm{im}}
\newcommand{\rrk}{\mathrm{zrg}}
\newcommand{\crk}{\mathrm{srg}}
\newcommand{\rk}{\mathrm{rg}}
\newcommand{\GL}{\mathrm{GL}}
\newcommand{\SL}{\mathrm{SL}}
\newcommand{\SO}{\mathrm{SO}}
\newcommand{\nul}{\mathrm{nul}}
\newcommand{\eig}{\mathrm{eig}}

\newcommand{\mytitle}{Singul\ae rwerte}

\title[Linadi]{\mytitle}
\author[Amin Coja-Oghlan]{Amin Coja-Oghlan}
\institute[Frankfurt]{JWGUFFM}
\date{}

\begin{document}

\frame[plain]{\titlepage}

\begin{frame}\frametitle{\mytitle}
	\begin{block}{Vorbemerkung}
		\begin{itemize}
			\item Wenn $A,B$ Matrizen der Gr\oe\ss en $m\times n$ und $n\times p$ sind, so ist
				\begin{align*}
					(A\cdot B)^\trans=B^\trans\cdot A^\trans
				\end{align*}
			\item Die Matrix $A^\trans A$ has Gr\oe\ss e $n\times n$.
			\item Die Matrix $A^\trans A$ ist symmetrisch.
		\end{itemize}
	\end{block}
\end{frame}

\begin{frame}\frametitle{\mytitle}
	\begin{block}{Definition}
		Sei $A$ eine symmetrische $n\times n$-Matrix.
		\begin{itemize}
			\item $A$ ist \emph{positiv semidefinit}, falls
				\begin{align*}
					x^\trans Ax\geq0&&\mbox{f\ue r alle }x\in\RR^n.
				\end{align*}
			\item $A$ ist \emph{positiv definit}, falls
				\begin{align*}
					x^\trans Ax>0&&\mbox{f\ue r alle }x\in\RR^n\setminus\cbc 0.
				\end{align*}
			\item $A$ ist \emph{negativ semidefinit}, falls
				\begin{align*}
					x^\trans Ax\leq0&&\mbox{f\ue r alle }x\in\RR^n.
				\end{align*}
			\item $A$ ist \emph{negativ definit}, falls
				\begin{align*}
					x^\trans Ax<0&&\mbox{f\ue r alle }x\in\RR^n\setminus\cbc 0.
				\end{align*}
		\end{itemize}
	\end{block}
\end{frame}

\begin{frame}\frametitle{\mytitle}
	\begin{block}{Notation}
		Seien $A,B$ symmetrische $n\times n$-Matrizen.
		\begin{itemize}
			\item Wir schreiben $A\preceq B$, falls $B-A$ positiv semidefinit ist.
			\item Wir schreiben $A\prec B$, falls $B-A$ positiv definit ist.
			\item Insbesondere bedeutet $0\preceq A$, da\ss\ $A$ positiv semidefinit ist.
			\item Analog bedeutet $0\prec A$, da\ss\ $A$ positiv definit ist.
		\end{itemize}
	\end{block}
\end{frame}

\begin{frame}\frametitle{\mytitle}
	\begin{block}{Proposition}
		Seien $A,B$ symmetrische $n\times n$-Matrizen und sei $c>0$.
		\begin{itemize}
			\item $A$ ist genau dann positiv semidefinit, wenn es eine Matrix $C$ gibt, so da\ss\ $A=C^\trans C$.
			\item $A$ ist genau dann positiv semidefinit, wenn es eine symmetrische $n\times n$-Matrix $C$ gibt, so da\ss\ $A=C^2$.
			\item $A$ ist genau dann positiv semidefinit, wenn alle Eigenwerte von $A$ gr\oe\ss er oder gleich Null sind.
			\item $A$ ist genau dann positiv definit, wenn alle Eigenwerte von $A$ positiv sind.
			\item Wenn $A,B$ positiv semidefinit sind, dann auch $A+B$.
			\item Wenn $A$ positiv semidefinit ist, dann auch $cA$.
			\item Wenn $A,B$ positiv definit sind, dann auch $A+B$.
			\item Wenn $A$ positiv definit ist, dann auch $cA$.
		\end{itemize}
	\end{block}
\end{frame}

\begin{frame}\frametitle{\mytitle}
	\begin{block}{Rechenschema}
	\begin{itemize}
		\item Angenommen $A$ ist positiv semidefinit und wir wollen eine symmetrische Matrix $C$ mit $A=C^2$ finden.
		\item Wir diagonalisieren $A$ und erhalten somit eine orthogonale Matrix $U$ mit
			\begin{align*}
				A&=U\diag(\lambda_1,\ldots,\lambda_n)U^\trans.
			\end{align*}
		\item Wir definieren jetzt
			\begin{align*}
				C=\sqrt A=U\diag(\sqrt{\lambda_1},\ldots,\sqrt{\lambda_n})U^\trans.
			\end{align*}
	\end{itemize}
	\end{block}
\end{frame}

\begin{frame}\frametitle{\mytitle}
	\begin{block}{Allgemeine Matrixexponenten}
	\begin{itemize}
		\item Sei $A$ eine positiv semidefinite $n\times n$-Matrix und $c>0$.
		\item Sei 
\begin{align*}
				A&=U\diag(\lambda_1,\ldots,\lambda_n)U^\trans.
			\end{align*}
			die Diagonalisierung von $A$.
		\item Wir definieren $A^c$ als die Matrix
\begin{align*}
				A^c&=U\diag(\lambda_1^c,\ldots,\lambda_n^c)U^\trans.
			\end{align*}
	\end{itemize}
	\end{block}
\end{frame}

\begin{frame}\frametitle{\mytitle}
	\begin{block}{Satz}
		Zu jeder $m\times n$-Matrix $A$ gibt es eine orthogonale $m\times m$-Matrix $U$, eine orthogonale $n\times n$-Matrix $V$ und eine $m\times n$-Matrix $D=(d_{ij})$ mit
		\begin{align*}
			d_{ij}=0\mbox{ falls }i\neq j\mbox{ und }d_{11}\geq\cdots\geq d_{\min\{n,m\}\,\min\{m,n\}}\geq0
		\end{align*}
		so da\ss\ 
	\begin{align*}
		A=UDV^\trans.
	\end{align*}
	\end{block}
	{\itshape Die Diagonaleintr\ae ge $d_{ii}$ von $D$ hei\ss en die \emph{Singul\ae rwerte} von $A$.}
\end{frame}

\begin{frame}\frametitle{\mytitle}
	\begin{block}{Rechenschema}
	\begin{itemize}
	\item Wir diagonalisieren die beiden positiv semidefiniten Matrizen $A^\trans A$ und $AA^\trans$:
		\begin{align*}
			AA^\trans&=U \diag(\lambda_1',\ldots,\lambda_m') U^\trans\\
			A^\trans A&=V \diag(\lambda_1'',\ldots,\lambda_n'') V^\trans
		\end{align*}
	\item Dabei ist $U$ eine orthogonale $m\times m$-Matrix.
	\item $V$ ist eine orthogonale $n\times n$-Matrix.
	\item Wir ordnen die Spalten von $U,V$ so, da\ss\ 
		\begin{align*}
		\lambda_1'\geq\cdots\geq\lambda_m'\\
		\lambda_1''\geq\cdots\geq\lambda_m''
		\end{align*}
	\end{itemize}
	\end{block}
\end{frame}

\begin{frame}\frametitle{\mytitle}
	\begin{block}{Rechenschema}
	\begin{itemize}
	\item Die Singul\ae rwerte von $A$ sind dann
		\begin{align*}
			\sqrt{\lambda_1'},\ldots,\sqrt{\lambda_m'}&&\mbox{wenn }m\leq n\\
			\sqrt{\lambda_1''},\ldots,\sqrt{\lambda_n''}&&\mbox{wenn }m> n
		\end{align*}
	\end{itemize}
	\end{block}
\end{frame}

\begin{frame}\frametitle{\mytitle}
	\begin{block}{Beispiel}
	\begin{itemize}
		\item Sei $ A=\begin{pmatrix} 1&1&0\\0&1&1 \end{pmatrix}. $
		\item Wir berechnen
			\begin{align*}
				AA^\trans&=\begin{pmatrix}2&1\\1&2\end{pmatrix}&
				A^\trans A&=\begin{pmatrix}1&1&0\\1&2&1\\0&1&1\end{pmatrix}
			\end{align*}
	\end{itemize}
	\end{block}
\end{frame}

\begin{frame}\frametitle{\mytitle}
	\begin{block}{Beispiel}
	\begin{itemize}
		\item Wir diagonalisieren $AA^\trans$:
			\begin{align*}
				AA^\trans&=U\begin{pmatrix} 3&0\\0&1 \end{pmatrix}U^\trans&\mbox{ mit }&&U=\begin{pmatrix}
					1/\sqrt 2&1/\sqrt 2\\
					1/\sqrt 2&-1/\sqrt 2
				\end{pmatrix}
			\end{align*}
		\item Wir diagonalisieren $A^\trans A$:
			\begin{align*}
				A^\trans A&=V\begin{pmatrix}3&0&0\\0&1&0\\0&0&0\end{pmatrix}V^\trans&\mbox{ mit }&&
					V=\begin{pmatrix}
						1/\sqrt6&1/\sqrt 2&1/\sqrt3\\
						\sqrt{2/3}&0&-1/\sqrt3\\
						1/\sqrt6&-1/\sqrt2&1/\sqrt3
				\end{pmatrix}
			\end{align*}
	\end{itemize}
	\end{block}
\end{frame}

\begin{frame}\frametitle{\mytitle}
	\begin{block}{Beispiel}
	\begin{itemize}
		\item Die Singul\ae rwerte von $A$ sind also $\sqrt 3,1$.
		\item Die Singul\ae rwertzerlegung lautet
			\begin{align*}
				A&=U\begin{pmatrix} \sqrt 3&0&0\\0&1&0 \end{pmatrix}V^\trans.
			\end{align*}
	\end{itemize}
	\end{block}
\end{frame}

\begin{frame}\frametitle{\mytitle}
	\begin{block}{Zusammenfassung}
	\begin{itemize}
		\item Die Singul\ae rwertzerlegung liefert auch f\ue r nicht symmetrische Matrizen eine relativ einfache Darstellung.
		\item Anders als die Diagonalisierung sind allerdings zwei Orthonormalbasen erforderlich.
		\item Die Singul\ae rwertzerlegung kann auf zwei Diagonalisierungsoperationen zur\ue ckgef\ue hrt werden.
	\end{itemize}
	\end{block}
\end{frame}
\end{document}

