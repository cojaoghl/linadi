\documentclass{beamer}
\usepackage{amsmath,graphics}
\usepackage{amssymb}

\usetheme{default}
\usepackage{xcolor}

\definecolor{solarizedBase03}{HTML}{002B36}
\definecolor{solarizedBase02}{HTML}{073642}
\definecolor{solarizedBase01}{HTML}{586e75}
\definecolor{solarizedBase00}{HTML}{657b83}
\definecolor{solarizedBase0}{HTML}{839496}
\definecolor{solarizedBase1}{HTML}{93a1a1}
\definecolor{solarizedBase2}{HTML}{EEE8D5}
\definecolor{solarizedBase3}{HTML}{FDF6E3}
\definecolor{solarizedYellow}{HTML}{B58900}
\definecolor{solarizedOrange}{HTML}{CB4B16}
\definecolor{solarizedRed}{HTML}{DC322F}
\definecolor{solarizedMagenta}{HTML}{D33682}
\definecolor{solarizedViolet}{HTML}{6C71C4}
%\definecolor{solarizedBlue}{HTML}{268BD2}
\definecolor{solarizedBlue}{HTML}{134676}
\definecolor{solarizedCyan}{HTML}{2AA198}
\definecolor{solarizedGreen}{HTML}{859900}
\definecolor{myBlue}{HTML}{162DB0}%{261CA4}
\setbeamercolor*{item}{fg=myBlue}
\setbeamercolor{normal text}{fg=solarizedBase03, bg=solarizedBase3}
\setbeamercolor{alerted text}{fg=myBlue}
\setbeamercolor{example text}{fg=myBlue, bg=solarizedBase3}
\setbeamercolor*{frametitle}{fg=solarizedRed}
\setbeamercolor*{title}{fg=solarizedRed}
\setbeamercolor{block title}{fg=myBlue, bg=solarizedBase3}
\setbeameroption{hide notes}
\setbeamertemplate{note page}[plain]
\beamertemplatenavigationsymbolsempty
\usefonttheme{professionalfonts}
\usefonttheme{serif}

\usepackage{fourier}

\def\vec#1{\mathchoice{\mbox{\boldmath$\displaystyle#1$}}
{\mbox{\boldmath$\textstyle#1$}}
{\mbox{\boldmath$\scriptstyle#1$}}
{\mbox{\boldmath$\scriptscriptstyle#1$}}}
\definecolor{OwnGrey}{rgb}{0.560,0.000,0.000} % #999999
\definecolor{OwnBlue}{rgb}{0.121,0.398,0.711} % #1f64b0
\definecolor{red4}{rgb}{0.5,0,0}
\definecolor{blue4}{rgb}{0,0,0.5}
\definecolor{Blue}{rgb}{0,0,0.66}
\definecolor{LightBlue}{rgb}{0.9,0.9,1}
\definecolor{Green}{rgb}{0,0.5,0}
\definecolor{LightGreen}{rgb}{0.9,1,0.9}
\definecolor{Red}{rgb}{0.9,0,0}
\definecolor{LightRed}{rgb}{1,0.9,0.9}
\definecolor{White}{gray}{1}
\definecolor{Black}{gray}{0}
\definecolor{LightGray}{gray}{0.8}
\definecolor{Orange}{rgb}{0.1,0.2,1}
\setbeamerfont{sidebar right}{size=\scriptsize}
\setbeamercolor{sidebar right}{fg=Black}

\renewcommand{\emph}[1]{{\textcolor{solarizedRed}{\itshape #1}}}

\newcommand\cA{\mathcal A}
\newcommand\cB{\mathcal B}
\newcommand\cC{\mathcal C}
\newcommand\cD{\mathcal D}
\newcommand\cE{\mathcal E}
\newcommand\cF{\mathcal F}
\newcommand\cG{\mathcal G}
\newcommand\cH{\mathcal H}
\newcommand\cI{\mathcal I}
\newcommand\cJ{\mathcal J}
\newcommand\cK{\mathcal K}
\newcommand\cL{\mathcal L}
\newcommand\cM{\mathcal M}
\newcommand\cN{\mathcal N}
\newcommand\cO{\mathcal O}
\newcommand\cP{\mathcal P}
\newcommand\cQ{\mathcal Q}
\newcommand\cR{\mathcal R}
\newcommand\cS{\mathcal S}
\newcommand\cT{\mathcal T}
\newcommand\cU{\mathcal U}
\newcommand\cV{\mathcal V}
\newcommand\cW{\mathcal W}
\newcommand\cX{\mathcal X}
\newcommand\cY{\mathcal Y}
\newcommand\cZ{\mathcal Z}

\newcommand\fA{\mathfrak A}
\newcommand\fB{\mathfrak B}
\newcommand\fC{\mathfrak C}
\newcommand\fD{\mathfrak D}
\newcommand\fE{\mathfrak E}
\newcommand\fF{\mathfrak F}
\newcommand\fG{\mathfrak G}
\newcommand\fH{\mathfrak H}
\newcommand\fI{\mathfrak I}
\newcommand\fJ{\mathfrak J}
\newcommand\fK{\mathfrak K}
\newcommand\fL{\mathfrak L}
\newcommand\fM{\mathfrak M}
\newcommand\fN{\mathfrak N}
\newcommand\fO{\mathfrak O}
\newcommand\fP{\mathfrak P}
\newcommand\fQ{\mathfrak Q}
\newcommand\fR{\mathfrak R}
\newcommand\fS{\mathfrak S}
\newcommand\fT{\mathfrak T}
\newcommand\fU{\mathfrak U}
\newcommand\fV{\mathfrak V}
\newcommand\fW{\mathfrak W}
\newcommand\fX{\mathfrak X}
\newcommand\fY{\mathfrak Y}
\newcommand\fZ{\mathfrak Z}

\newcommand\fa{\mathfrak a}
\newcommand\fb{\mathfrak b}
\newcommand\fc{\mathfrak c}
\newcommand\fd{\mathfrak d}
\newcommand\fe{\mathfrak e}
\newcommand\ff{\mathfrak f}
\newcommand\fg{\mathfrak g}
\newcommand\fh{\mathfrak h}
%\newcommand\fi{\mathfrak i}
\newcommand\fj{\mathfrak j}
\newcommand\fk{\mathfrak k}
\newcommand\fl{\mathfrak l}
\newcommand\fm{\mathfrak m}
\newcommand\fn{\mathfrak n}
\newcommand\fo{\mathfrak o}
\newcommand\fp{\mathfrak p}
\newcommand\fq{\mathfrak q}
\newcommand\fr{\mathfrak r}
\newcommand\fs{\mathfrak s}
\newcommand\ft{\mathfrak t}
\newcommand\fu{\mathfrak u}
\newcommand\fv{\mathfrak v}
\newcommand\fw{\mathfrak w}
\newcommand\fx{\mathfrak x}
\newcommand\fy{\mathfrak y}
\newcommand\fz{\mathfrak z}

\newcommand\vA{\vec A}
\newcommand\vB{\vec B}
\newcommand\vC{\vec C}
\newcommand\vD{\vec D}
\newcommand\vE{\vec E}
\newcommand\vF{\vec F}
\newcommand\vG{\vec G}
\newcommand\vH{\vec H}
\newcommand\vI{\vec I}
\newcommand\vJ{\vec J}
\newcommand\vK{\vec K}
\newcommand\vL{\vec L}
\newcommand\vM{\vec M}
\newcommand\vN{\vec N}
\newcommand\vO{\vec O}
\newcommand\vP{\vec P}
\newcommand\vQ{\vec Q}
\newcommand\vR{\vec R}
\newcommand\vS{\vec S}
\newcommand\vT{\vec T}
\newcommand\vU{\vec U}
\newcommand\vV{\vec V}
\newcommand\vW{\vec W}
\newcommand\vX{\vec X}
\newcommand\vY{\vec Y}
\newcommand\vZ{\vec Z}

\newcommand\va{\vec a}
\newcommand\vb{\vec b}
\newcommand\vc{\vec c}
\newcommand\vd{\vec d}
\newcommand\ve{\vec e}
\newcommand\vf{\vec f}
\newcommand\vg{\vec g}
\newcommand\vh{\vec h}
\newcommand\vi{\vec i}
\newcommand\vj{\vec j}
\newcommand\vk{\vec k}
\newcommand\vl{\vec l}
\newcommand\vm{\vec m}
\newcommand\vn{\vec n}
\newcommand\vo{\vec o}
\newcommand\vp{\vec p}
\newcommand\vq{\vec q}
\newcommand\vr{\vec r}
\newcommand\vs{\vec s}
\newcommand\vt{\vec t}
\newcommand\vu{\vec u}
\newcommand\vv{\vec v}
\newcommand\vw{\vec w}
\newcommand\vx{\vec x}
\newcommand\vy{\vec y}
\newcommand\vz{\vec z}

\renewcommand\AA{\mathbb A}
\newcommand\NN{\mathbb N}
\newcommand\ZZ{\mathbb Z}
\newcommand\PP{\mathbb P}
\newcommand\QQ{\mathbb Q}
\newcommand\RR{\mathbb R}
\renewcommand\SS{\mathbb S}
\newcommand\CC{\mathbb C}

\newcommand{\ord}{\mathrm{ord}}
\newcommand{\id}{\mathrm{id}}
\newcommand{\pr}{\mathrm{P}}
\newcommand{\Vol}{\mathrm{vol}}
\newcommand\norm[1]{\left\|{#1}\right\|} 
\newcommand\sign{\mathrm{sign}}
\newcommand{\eps}{\varepsilon}
\newcommand{\abs}[1]{\left|#1\right|}
\newcommand\bc[1]{\left({#1}\right)} 
\newcommand\cbc[1]{\left\{{#1}\right\}} 
\newcommand\bcfr[2]{\bc{\frac{#1}{#2}}} 
\newcommand{\bck}[1]{\left\langle{#1}\right\rangle} 
\newcommand\brk[1]{\left\lbrack{#1}\right\rbrack} 
\newcommand\scal[2]{\bck{{#1},{#2}}} 
\newcommand{\vecone}{\mathbb{1}}
\newcommand{\tensor}{\otimes}
\newcommand{\diag}{\mathrm{diag}}
\newcommand{\ggt}{\mathrm{ggT}}
\newcommand{\kgv}{\mathrm{kgV}}

\newcommand{\Karonski}{Karo\'nski}
\newcommand{\Erdos}{Erd\H{o}s}
\newcommand{\Renyi}{R\'enyi}
\newcommand{\Lovasz}{Lov\'asz}
\newcommand{\Juhasz}{Juh\'asz}
\newcommand{\Bollobas}{Bollob\'as}
\newcommand{\Furedi}{F\"uredi}
\newcommand{\Komlos}{Koml\'os}
\newcommand{\Luczak}{\L uczak}
\newcommand{\Kucera}{Ku\v{c}era}
\newcommand{\Szemeredi}{Szemer\'edi}

\renewcommand{\ae}{\"a}
\renewcommand{\oe}{\"o}
\newcommand{\ue}{\"u}
\newcommand{\Ae}{\"A}
\newcommand{\Oe}{\"O}
\newcommand{\Ue}{\"U}

\title[Linadi]{Der Miller-Rabin-Test}
\author[Amin Coja-Oghlan]{Amin Coja-Oghlan}
\institute[Frankfurt]{JWGUFFM}
\date{}

\begin{document}

\frame[plain]{\titlepage}

\begin{frame}\frametitle{Der Miller-Rabin-Test}
	\begin{block}{Erinnerung}
		\begin{itemize}
			\item Der Fermat-Test ist effizient aber unzuverl\ae ssig.
			\item Wenn die Eingabe eine Primzahl ist, gibt der Fermat-Test stets ``Primzahl'' aus.
			\item Aber wenn die Eingabe eine Carmichaelzahl ist, k\oe nnte der Fermat-Test auch mit sehr hoher Wahrscheinlichkeit ``Primzahl'' antworten.
			\item \emph{Ziel:} ein effizienter Primzahltest, der auch mit Carmichaelzahlen zurechtkommt.
		\end{itemize}
	\end{block}
\end{frame}

\begin{frame}\frametitle{Der Miller-Rabin-Test}
	\begin{block}{Definition}
		\begin{itemize}
			\item Sei $n\geq3$ ungerade.
			\item Sei $k=w_2(n-1)$, so da\ss\ $n-1=2^ku$ f\ue r eine ungerade Zahl $u$.
			\item Eine Zahl $1<a<n$ hei\ss t \emph{A-Zeuge} f\ue r $n$, falls
				\begin{align*}
					a^{u}\not\equiv1\mod n&&\mbox{und}\\
					a^{u2^i}\not\equiv-1\mod n&&\mbox{f\ue r alle $0\leq i<k$.}
				\end{align*}
			\item Andernfalls nennen wir $a$ einen \emph{A-L\ue gner}.
		\end{itemize}
	\end{block}
	\begin{block}{Anmerkung}
Jeder F-Zeuge ist ein A-Zeuge.	
	\end{block}
\end{frame}



\begin{frame}\frametitle{Der Miller-Rabin-Test}
	\begin{block}{Algorithmus {\tt MillerRabin}($n$)}
		\begin{enumerate}
			\item Wenn $n=2$ gib ``Primzahl'' aus; andernfalls, wenn $n$ gerade ist, gib ``zusammengesetzt'' aus.
			\item W\ae hle $1<a<n$ zuf\ae llig.
			\item Wenn $a$ ein A-Zeuge ist, gib ``zusammengesetzt'' aus; sonst gib ``Primzahl'' aus.
		\end{enumerate}
	\end{block}
\end{frame}

\begin{frame}\frametitle{Der Miller-Rabin-Test}
	\begin{block}{Laufzeit}
		\begin{itemize}
			\item Man kann effizient herausfinden, ob $a$ ein A-Zeuge ist.
			\item Es gibt insgesamt nur $k\leq\log_2n$ M\oe glichkeiten f\ue r $i$.
			\item Jede einzelne Bedingung 
				\begin{align*}
					a^{u}&\not\equiv1\mod n\\
					a^{u2^i}&\not\equiv-1\mod n
				\end{align*}
				kann mit schnellem Potenzieren nachgepr\ue ft werden.
		\end{itemize}
	\end{block}
\end{frame}

\begin{frame}\frametitle{Der Miller-Rabin-Test}
	\begin{block}{Lemma}
		Wenn eine ungerade Zahl $n\geq3$ einen A-Zeugen $a$ besitzt, ist $n$ zusammengesetzt.
	\end{block}
	\begin{overprint}
		\onslide<1>
		\begin{block}{Beweis}
			\begin{itemize}
				\item Sei $b_i=a^{u2^i}$ f\ue r $0\leq i\leq k$.
				\item Wenn $b_k\not\equiv1\mod1$, ist $a$ ein F-Zeuge; also ist $n$ zusammengesetzt.
				\item Sonst sei $1\leq i\leq k$ minimal mit $b_i\equiv1\mod n$.
				\item Weil $a$ ein A-Zeuge ist, gilt 
					\begin{align*}
						b_{i-1}&\not\equiv1\mod n\quad\mbox{und}\quad b_{i-1}&\not\equiv-1\mod n&&\mbox{aber}\\
						b_{i-1}^2&\equiv b_i\equiv1\mod n.
					\end{align*}
			\end{itemize}	
		\end{block}
		\onslide<2>
		\begin{block}{Beweis (Fortsetzung)}
			\begin{itemize}
				\item Also gilt $n\mid b_{i-1}^2-1=(b_{i-1}-1)(b_{i-1}+1)$.
				\item Angenommen $n$ w\ae re eine Primzahl.
				\item Dann schlie\ss en wir
					\begin{align*}
						n\mid b_{i-1}-1&&\mbox{oder}&&n\mid b_{i-1}+1.
					\end{align*}
				\item Das bedeutet aber
					\begin{align*}
						b_{i-1}\equiv-1\mod n&&\mbox{oder}&&b_{i-1}\equiv1\mod n,
					\end{align*}
					Widerspruch.
			\end{itemize}	
		\end{block}
		\onslide<3>
		\begin{block}{Anmerkung}
			\begin{itemize}
				\item Aus dem Lemma folgt, da\ss\ $n$ sicher zusammengesetzt ist, wenn der Miller-Rabin-Test ``zusammengesetzt'' ausgibt.	
				\item Auf eine Primzahl $n$ angewandt gibt Miller-Rabin also stets ``Primzahl'' aus.
			\end{itemize}
		\end{block}
	\end{overprint}
\end{frame}

\begin{frame}\frametitle{Der Miller-Rabin-Test}
	\begin{block}{Satz}
		Wenn $n>3$ zusammengesetzt ist, gibt der Miller-Rabin-Test mit Wahrscheinlichkeit mindestens $\frac{1}{2}$ ``zusammengesetzt'' aus.
	\end{block}
	\begin{overprint}
		\onslide<1>
		\begin{block}{Beweis}
			\begin{itemize}
				\item Wir d\ue rfen annehmen, da\ss\ $n$ eine Carmichaelzahl ist.
				\item Definiere
					\begin{align*}
						i_0=\max\cbc{i\geq0:\mbox{es gibt einen A-L\ue gner $a$ mit }a^{u2^i}\equiv-1\mod n}.
					\end{align*}
				\item Sei $a_0$ entsprechend ein A-L\ue gner mit
					\begin{align*}
						a_0^{u2^{i_0}}\equiv-1\mod n.
					\end{align*}
			\end{itemize}
		\end{block}
		\onslide<2>
		\begin{block}{Beweis (Fortsetzung)}
			\begin{itemize}
				\item Weil $n$ eine Carmichaelzahl ist, gilt
					\begin{align*}
						a_0^{n-1}=a_0^{u2^k}\equiv1\mod n.
					\end{align*}
				\item Insbesondere sehen wir, da\ss\ $i_0<k$.
				\item Definiere nun
					\begin{align*}
						H&=\cbc{a+n\ZZ:a^{u2^{i_0}}\equiv\pm1\mod n}.
					\end{align*}
			\end{itemize}
		\end{block}
		\onslide<3>
		\begin{block}{Beweis (Fortsetzung)}
			\begin{itemize}
				\item $H$ enth\ae lt alle A-L\ue gner $a$.
				\item Denn falls $a^u\equiv1\mod n$, folgt $a^{u2^{i_0}}\equiv1\mod n$.
				\item Sonst gilt $a^{u2^i}\equiv-1\mod n$ f\ue r ein $i\leq i_0$.
				\item Also $a^{u2^{i_0}}\equiv\pm1\mod n$.
				\item Jedenfalls sehen wir, da\ss\ $a\in H$.
			\end{itemize}
		\end{block}
		\onslide<4>
		\begin{block}{Beweis (Fortsetzung)}
			\begin{itemize}
				\item $H$ ist eine Untergruppe von $\ZZ_n^\times$.
				\item Denn wenn $a+n\ZZ,b+n\ZZ\in H$, dann gilt
					\begin{align*}
						\bc{(a+n\ZZ)^{-1}(b+n\ZZ)}^{u2^{i_0}}&= (a+n\ZZ)^{u2^{i_0}}(b+n\ZZ)^{u2^{i_0}}\\&=(\pm1+n\ZZ)\cdot(\pm1+n\ZZ)=\pm1+n\ZZ.
					\end{align*}
				\item Au\ss erdem gilt $1+n\ZZ\in H$.
			\end{itemize}
		\end{block}
		\onslide<5>
		\begin{block}{Beweis (Fortsetzung)}
			\begin{itemize}
				\item Es gilt $H\neq\ZZ_n^\times$.
				\item Denn weil $n$ eine Carmichaelzahl ist, k\oe nnen wir $n$ schreiben als
					\begin{align*}
						n=p\cdot m&&\mbox{mit $p$ prim und }\ggt(p,m)=1.
					\end{align*}
				\item Nach dem Chinesischen Restsatz gibt es $a\in\NN$ mit
					\begin{align*}
						a\equiv a_0\mod p&&\mbox{und}&&a\equiv1\mod m.
					\end{align*}
				\item Wir behaupten, da\ss\ $a+n\ZZ\in\ZZ_n^\times \setminus H$.
			\end{itemize}
		\end{block}
		\onslide<6>
		\begin{block}{Beweis (Fortsetzung)}
			\begin{itemize}
				\item Denn einerseits gilt $\ggt(a,n)=1$, also $a+n\ZZ\in\ZZ_n^\times$.
				\item Andererseits erhalten wir
					\begin{align*}
						a^{u2^{i_0}}&\equiv-1\mod p\\
						a^{u2^{i_0}}&\equiv1\mod m.
					\end{align*}
				\item Also schlie\ss en wir, da\ss
					\begin{align*}
						a\not\equiv-1\mod n&&\mbox{und}&&a\not\equiv1\mod n.
					\end{align*}
				\item Folglich gilt $a\not\in H$.	
			\end{itemize}
		\end{block}
		\onslide<7>
		\begin{block}{Beweis (Fortsetzung)}
			\begin{itemize}
				\item Aus dem Satz von Lagrange folgt also
					\begin{align*}
						|H|\leq|\ZZ_n^\times|/2.
					\end{align*}
				\item Folglich trifft der Algorithmus mit Wahrscheinlichkeit mindestens $\frac{1}{2}$ einen A-Zeugen.
			\end{itemize}
		\end{block}
	\end{overprint}
\end{frame}

\begin{frame}\frametitle{Der Miller-Rabin-Test}
\begin{block}{Erzeugen gro\ss er Primzahlen}
\begin{itemize}
\item Angenommen wir m\oe chten eine Primzahl $p\leq n$ zuf\ae llig erzeugen.
\item W\ae hle dazu einfach $1\leq p\leq n$ zuf\ae llig.
\item Wende den Miller-Rabin-Test an, um zu pr\ue fen, ob $p$ eine Primzahl ist.
\item Um die Erfolgswahrscheinlichkeit zu erh\oe hen, wiederhole den Test mehrfach.
\item Wenn nicht, wiederhole.
\item Aus dem Primzahlsatz folgt, da\ss\ wir nach etwa $\log n$ Versuchen wahrscheinlich auf eine Primzahl sto\ss en.
\item Das Verfahren ist also effizient.
\end{itemize}
\end{block}
\end{frame}

\begin{frame}\frametitle{Zusammenfassung}
\begin{itemize}
\item Der Miller-Rabin-Test ist ein effizienter Monte-Carlo-Primzahltest.
\item Er kann verwendet werden, um gro\ss e Primzahlen effizient zuf\ae llig zu erzeugen.
\end{itemize}
\end{frame}

\end{document}
