\documentclass{beamer}
\usepackage{amsmath,graphics}
\usepackage{amssymb}

\usetheme{default}
\usepackage{xcolor}

\definecolor{solarizedBase03}{HTML}{002B36}
\definecolor{solarizedBase02}{HTML}{073642}
\definecolor{solarizedBase01}{HTML}{586e75}
\definecolor{solarizedBase00}{HTML}{657b83}
\definecolor{solarizedBase0}{HTML}{839496}
\definecolor{solarizedBase1}{HTML}{93a1a1}
\definecolor{solarizedBase2}{HTML}{EEE8D5}
\definecolor{solarizedBase3}{HTML}{FDF6E3}
\definecolor{solarizedYellow}{HTML}{B58900}
\definecolor{solarizedOrange}{HTML}{CB4B16}
\definecolor{solarizedRed}{HTML}{DC322F}
\definecolor{solarizedMagenta}{HTML}{D33682}
\definecolor{solarizedViolet}{HTML}{6C71C4}
%\definecolor{solarizedBlue}{HTML}{268BD2}
\definecolor{solarizedBlue}{HTML}{134676}
\definecolor{solarizedCyan}{HTML}{2AA198}
\definecolor{solarizedGreen}{HTML}{859900}
\definecolor{myBlue}{HTML}{162DB0}%{261CA4}
\setbeamercolor*{item}{fg=myBlue}
\setbeamercolor{normal text}{fg=solarizedBase03, bg=solarizedBase3}
\setbeamercolor{alerted text}{fg=myBlue}
\setbeamercolor{example text}{fg=myBlue, bg=solarizedBase3}
\setbeamercolor*{frametitle}{fg=solarizedRed}
\setbeamercolor*{title}{fg=solarizedRed}
\setbeamercolor{block title}{fg=myBlue, bg=solarizedBase3}
\setbeameroption{hide notes}
\setbeamertemplate{note page}[plain]
\beamertemplatenavigationsymbolsempty
\usefonttheme{professionalfonts}
\usefonttheme{serif}

\usepackage{fourier}

\def\vec#1{\mathchoice{\mbox{\boldmath$\displaystyle#1$}}
{\mbox{\boldmath$\textstyle#1$}}
{\mbox{\boldmath$\scriptstyle#1$}}
{\mbox{\boldmath$\scriptscriptstyle#1$}}}
\definecolor{OwnGrey}{rgb}{0.560,0.000,0.000} % #999999
\definecolor{OwnBlue}{rgb}{0.121,0.398,0.711} % #1f64b0
\definecolor{red4}{rgb}{0.5,0,0}
\definecolor{blue4}{rgb}{0,0,0.5}
\definecolor{Blue}{rgb}{0,0,0.66}
\definecolor{LightBlue}{rgb}{0.9,0.9,1}
\definecolor{Green}{rgb}{0,0.5,0}
\definecolor{LightGreen}{rgb}{0.9,1,0.9}
\definecolor{Red}{rgb}{0.9,0,0}
\definecolor{LightRed}{rgb}{1,0.9,0.9}
\definecolor{White}{gray}{1}
\definecolor{Black}{gray}{0}
\definecolor{LightGray}{gray}{0.8}
\definecolor{Orange}{rgb}{0.1,0.2,1}
\setbeamerfont{sidebar right}{size=\scriptsize}
\setbeamercolor{sidebar right}{fg=Black}

\renewcommand{\emph}[1]{{\textcolor{solarizedRed}{\itshape #1}}}

\newcommand\cA{\mathcal A}
\newcommand\cB{\mathcal B}
\newcommand\cC{\mathcal C}
\newcommand\cD{\mathcal D}
\newcommand\cE{\mathcal E}
\newcommand\cF{\mathcal F}
\newcommand\cG{\mathcal G}
\newcommand\cH{\mathcal H}
\newcommand\cI{\mathcal I}
\newcommand\cJ{\mathcal J}
\newcommand\cK{\mathcal K}
\newcommand\cL{\mathcal L}
\newcommand\cM{\mathcal M}
\newcommand\cN{\mathcal N}
\newcommand\cO{\mathcal O}
\newcommand\cP{\mathcal P}
\newcommand\cQ{\mathcal Q}
\newcommand\cR{\mathcal R}
\newcommand\cS{\mathcal S}
\newcommand\cT{\mathcal T}
\newcommand\cU{\mathcal U}
\newcommand\cV{\mathcal V}
\newcommand\cW{\mathcal W}
\newcommand\cX{\mathcal X}
\newcommand\cY{\mathcal Y}
\newcommand\cZ{\mathcal Z}

\newcommand\fA{\mathfrak A}
\newcommand\fB{\mathfrak B}
\newcommand\fC{\mathfrak C}
\newcommand\fD{\mathfrak D}
\newcommand\fE{\mathfrak E}
\newcommand\fF{\mathfrak F}
\newcommand\fG{\mathfrak G}
\newcommand\fH{\mathfrak H}
\newcommand\fI{\mathfrak I}
\newcommand\fJ{\mathfrak J}
\newcommand\fK{\mathfrak K}
\newcommand\fL{\mathfrak L}
\newcommand\fM{\mathfrak M}
\newcommand\fN{\mathfrak N}
\newcommand\fO{\mathfrak O}
\newcommand\fP{\mathfrak P}
\newcommand\fQ{\mathfrak Q}
\newcommand\fR{\mathfrak R}
\newcommand\fS{\mathfrak S}
\newcommand\fT{\mathfrak T}
\newcommand\fU{\mathfrak U}
\newcommand\fV{\mathfrak V}
\newcommand\fW{\mathfrak W}
\newcommand\fX{\mathfrak X}
\newcommand\fY{\mathfrak Y}
\newcommand\fZ{\mathfrak Z}

\newcommand\fa{\mathfrak a}
\newcommand\fb{\mathfrak b}
\newcommand\fc{\mathfrak c}
\newcommand\fd{\mathfrak d}
\newcommand\fe{\mathfrak e}
\newcommand\ff{\mathfrak f}
\newcommand\fg{\mathfrak g}
\newcommand\fh{\mathfrak h}
%\newcommand\fi{\mathfrak i}
\newcommand\fj{\mathfrak j}
\newcommand\fk{\mathfrak k}
\newcommand\fl{\mathfrak l}
\newcommand\fm{\mathfrak m}
\newcommand\fn{\mathfrak n}
\newcommand\fo{\mathfrak o}
\newcommand\fp{\mathfrak p}
\newcommand\fq{\mathfrak q}
\newcommand\fr{\mathfrak r}
\newcommand\fs{\mathfrak s}
\newcommand\ft{\mathfrak t}
\newcommand\fu{\mathfrak u}
\newcommand\fv{\mathfrak v}
\newcommand\fw{\mathfrak w}
\newcommand\fx{\mathfrak x}
\newcommand\fy{\mathfrak y}
\newcommand\fz{\mathfrak z}

\newcommand\vA{\vec A}
\newcommand\vB{\vec B}
\newcommand\vC{\vec C}
\newcommand\vD{\vec D}
\newcommand\vE{\vec E}
\newcommand\vF{\vec F}
\newcommand\vG{\vec G}
\newcommand\vH{\vec H}
\newcommand\vI{\vec I}
\newcommand\vJ{\vec J}
\newcommand\vK{\vec K}
\newcommand\vL{\vec L}
\newcommand\vM{\vec M}
\newcommand\vN{\vec N}
\newcommand\vO{\vec O}
\newcommand\vP{\vec P}
\newcommand\vQ{\vec Q}
\newcommand\vR{\vec R}
\newcommand\vS{\vec S}
\newcommand\vT{\vec T}
\newcommand\vU{\vec U}
\newcommand\vV{\vec V}
\newcommand\vW{\vec W}
\newcommand\vX{\vec X}
\newcommand\vY{\vec Y}
\newcommand\vZ{\vec Z}

\newcommand\va{\vec a}
\newcommand\vb{\vec b}
\newcommand\vc{\vec c}
\newcommand\vd{\vec d}
\newcommand\ve{\vec e}
\newcommand\vf{\vec f}
\newcommand\vg{\vec g}
\newcommand\vh{\vec h}
\newcommand\vi{\vec i}
\newcommand\vj{\vec j}
\newcommand\vk{\vec k}
\newcommand\vl{\vec l}
\newcommand\vm{\vec m}
\newcommand\vn{\vec n}
\newcommand\vo{\vec o}
\newcommand\vp{\vec p}
\newcommand\vq{\vec q}
\newcommand\vr{\vec r}
\newcommand\vs{\vec s}
\newcommand\vt{\vec t}
\newcommand\vu{\vec u}
\newcommand\vv{\vec v}
\newcommand\vw{\vec w}
\newcommand\vx{\vec x}
\newcommand\vy{\vec y}
\newcommand\vz{\vec z}

\renewcommand\AA{\mathbb A}
\newcommand\NN{\mathbb N}
\newcommand\ZZ{\mathbb Z}
\newcommand\PP{\mathbb P}
\newcommand\QQ{\mathbb Q}
\newcommand\RR{\mathbb R}
\renewcommand\SS{\mathbb S}
\newcommand\CC{\mathbb C}

\newcommand{\ord}{\mathrm{ord}}
\newcommand{\id}{\mathrm{id}}
\newcommand{\pr}{\mathrm{P}}
\newcommand{\Vol}{\mathrm{vol}}
\newcommand\norm[1]{\left\|{#1}\right\|} 
\newcommand\sign{\mathrm{sign}}
\newcommand{\eps}{\varepsilon}
\newcommand{\abs}[1]{\left|#1\right|}
\newcommand\bc[1]{\left({#1}\right)} 
\newcommand\cbc[1]{\left\{{#1}\right\}} 
\newcommand\bcfr[2]{\bc{\frac{#1}{#2}}} 
\newcommand{\bck}[1]{\left\langle{#1}\right\rangle} 
\newcommand\brk[1]{\left\lbrack{#1}\right\rbrack} 
\newcommand\scal[2]{\bck{{#1},{#2}}} 
\newcommand{\vecone}{\mathbb{1}}
\newcommand{\tensor}{\otimes}
\newcommand{\diag}{\mathrm{diag}}
\newcommand{\ggt}{\mathrm{ggT}}
\newcommand{\kgv}{\mathrm{kgV}}
\newcommand{\trans}{\top}

\newcommand{\Karonski}{Karo\'nski}
\newcommand{\Erdos}{Erd\H{o}s}
\newcommand{\Renyi}{R\'enyi}
\newcommand{\Lovasz}{Lov\'asz}
\newcommand{\Juhasz}{Juh\'asz}
\newcommand{\Bollobas}{Bollob\'as}
\newcommand{\Furedi}{F\"uredi}
\newcommand{\Komlos}{Koml\'os}
\newcommand{\Luczak}{\L uczak}
\newcommand{\Kucera}{Ku\v{c}era}
\newcommand{\Szemeredi}{Szemer\'edi}

\renewcommand{\ae}{\"a}
\renewcommand{\oe}{\"o}
\newcommand{\ue}{\"u}
\newcommand{\Ae}{\"A}
\newcommand{\Oe}{\"O}
\newcommand{\Ue}{\"U}

\newcommand{\im}{\mathrm{im}}
\newcommand{\rrk}{\mathrm{zrg}}
\newcommand{\crk}{\mathrm{srg}}
\newcommand{\rk}{\mathrm{rg}}
\newcommand{\GL}{\mathrm{GL}}

\newcommand{\mytitle}{Die inverse Matrix}

\title[Linadi]{\mytitle}
\author[Amin Coja-Oghlan]{Amin Coja-Oghlan}
\institute[Frankfurt]{JWGUFFM}
\date{}

\begin{document}

\frame[plain]{\titlepage}

\begin{frame}\frametitle{\mytitle}
	\begin{block}{Quadratische Matrizen}
	\begin{itemize}
		\item Eine Matrix, die gleichviele Zeilen und Spalten hat, nennt man \emph{quadratisch}.
		\item Quadratische Matrizen kann man mit sich selbst multiplizieren.
		\item Wir definieren daher f\ue r eine quadratische Matrix $A$ und eine Zahl $\ell\in\NN$:
			\begin{align*}
				A^\ell&=\underbrace{A\cdot\ \cdots\ \cdot A}_{\mbox{$\ell$ mal}}
			\end{align*}
		\item Wir wissen bereits, da\ss\ die Menge $\RR^{n\times n}$ der $n\times n$-Matrizen ein Ring ist.
		\item Das neutrale Element bzgl.\ der Multiplikation ist die \emph{Einheitsmatrix} $\id_n$.
	\end{itemize}
	\end{block}
\end{frame}

\begin{frame}\frametitle{\mytitle}
	\begin{block}{Proposition}
		Die Einheitengruppe des Rings $\RR^{n\times n}$ besteht aus allen $n\times n$-Matrizen $A$, die vollen Rang haben.
	\end{block}
	\begin{block}{Die inverse Matrix}
		\begin{itemize}
			\item Zu einer Matrix $A\in\RR^{n\times n}$ mit $\rk(A)=n$ gibt es also eine $n\times n$-Matrix $A^{-1}$ mit
				\begin{align*}
					A^{-1}\cdot A=\id_n.
				\end{align*}
			\item Wir nennen $A^{-1}$ die \emph{inverse Matrix} von $A$.
		\end{itemize}
	\end{block}
\end{frame}

\begin{frame}\frametitle{\mytitle}
	\begin{block}{Berechnung der inversen Matrix}
	\begin{itemize}
	\item Sei $A$ eine $n\times n$-Matrix mit Rang $n$. 
	\item Bilde die $n\times(2n)$-Matrix
		\begin{align*}
			\bc{A\quad\id_n}
		\end{align*}
	\item Wende Zeilenumformungen an, um diese Matrix in die Form
		\begin{align*}
			\bc{\id_n\quad B}
		\end{align*}
		zu \ue berf\ue hren.
	\item Dann ist $A^{-1}=B$.
	\item Falls eine \Ue berf\ue hrung in die Form $(\id_n\quad B)$ nicht m\oe glich ist, hat $A$ nicht vollen Rang.
	\end{itemize}	
	\end{block}
\end{frame}

\begin{frame}\frametitle{\mytitle}
	\begin{block}{Beispiel}
	\begin{itemize}
		\item Sei
			\begin{align*}
				A&=\begin{pmatrix}
					-1&1&1\\1&-2&1\\1&1&-3
				\end{pmatrix}
			\end{align*}
		\item Wir bilden die $3\times 6$-Matrix
\begin{align*}
				\begin{pmatrix}
					-1&1&1&1&0&0\\1&-2&1&0&1&0\\1&1&-3&0&0&1 
				\end{pmatrix}
			\end{align*}
		\item Zun\ae chst bringen wir die Matrix in Zeilenstufenform.
	\end{itemize}	
	\end{block}
\end{frame}

\begin{frame}\frametitle{\mytitle}
	\begin{block}{Beispiel}
	\begin{itemize}
		\item Addiere die erste Zeile zur zweiten und zur dritten Zeile:
\begin{align*}
				\begin{pmatrix}
					-1&1&1&1&0&0\\0&-1&2&1&1&0\\0&2&-2&1&0&1 
				\end{pmatrix}
			\end{align*}
		\item Addiere das doppelte der zweiten Zeile zur dritten Zeile:
\begin{align*}
				\begin{pmatrix}
					-1&1&1&1&0&0\\0&-1&2&1&1&0\\0&0&2&3&2&1 
				\end{pmatrix}
			\end{align*}
		\item Die Matrix ist in Zeilenstufenform, aber wir sind noch nicht fertig.
	\end{itemize}	
	\end{block}
\end{frame}

\begin{frame}\frametitle{\mytitle}
	\begin{block}{Beispiel}
	\begin{itemize}
		\item Wir m\oe chten die ersten drei Spalten in die Form
			\begin{align*}
				\id_3&=\begin{pmatrix}1&0&0\\0&1&0\\0&0&1 \end{pmatrix}
			\end{align*}
			bringen.
		\item Wir beginnen mit der letzten Zeile.
		\item Dividiere die letzte Zeile der Matrix durch zwei:
\begin{align*}
				\begin{pmatrix}
					-1&1&1&1&0&0\\0&-1&2&1&1&0\\0&0&1&\frac{3}{2}&1&\frac{1}{2} 
				\end{pmatrix}
			\end{align*}
	\end{itemize}	
	\end{block}
\end{frame}

\begin{frame}\frametitle{\mytitle}
	\begin{block}{Beispiel}
	\begin{itemize}
		\item Nun subtrahiere das doppelte der letzten Zeile von der zweiten Zeile:
\begin{align*}
				\begin{pmatrix}
					-1&1&1&1&0&0\\0&-1&0&-2&-1&-1\\0&0&1&\frac{3}{2}&1&\frac{1}{2} 
				\end{pmatrix}
			\end{align*}
		\item Ferner subtrahiere die dritte Zeile von der ersten:
\begin{align*}
				\begin{pmatrix}
					-1&1&0&-\frac{1}{2}&-1&-\frac{1}{2}\\0&-1&0&-2&-1&-1\\0&0&1&\frac{3}{2}&1&\frac{1}{2} 
				\end{pmatrix}
			\end{align*}
	\end{itemize}	
	\end{block}
\end{frame}

\begin{frame}\frametitle{\mytitle}
	\begin{block}{Beispiel}
	\begin{itemize}
		\item Multipliziere die zweite Zeile mit $-1$:
\begin{align*}
				\begin{pmatrix}
					-1&1&0&-\frac{1}{2}&-1&-\frac{1}{2}\\0&1&0&2&1&1\\0&0&1&\frac{3}{2}&1&\frac{1}{2} 
				\end{pmatrix}
			\end{align*}
		\item Subtrahiere die zweite Zeile von der ersten Zeile:
\begin{align*}
				\begin{pmatrix}
					-1&0&0&-\frac{5}{2}&-2&-\frac{3}{2}\\0&1&0&2&1&1\\0&0&1&\frac{3}{2}&1&\frac{1}{2} 
				\end{pmatrix}
			\end{align*}
	\end{itemize}	
	\end{block}
\end{frame}

\begin{frame}\frametitle{\mytitle}
	\begin{block}{Beispiel}
	\begin{itemize}
		\item Multipliziere die erste Zeile mit $-1$:
\begin{align*}
				\begin{pmatrix}
					1&0&0&\frac{5}{2}&2&\frac{3}{2}\\0&1&0&2&1&1\\0&0&1&\frac{3}{2}&1&\frac{1}{2} 
				\end{pmatrix}
			\end{align*}
		\item Die inverse Matrix ist
			\begin{align*}
				A^{-1}&=\begin{pmatrix}
					\frac{5}{2}&2&\frac{3}{2}\\2&1&1\\\frac{3}{2}&1&\frac{1}{2} 
				\end{pmatrix}
			\end{align*}
	\end{itemize}	
	\end{block}
\end{frame}

\begin{frame}\frametitle{\mytitle}
	\begin{block}{Beispiel}
	\begin{itemize}
		\item Zur Probe rechnen wir $A^{-1}\cdot A$:
			\begin{align*}
				\begin{pmatrix}
					\frac{5}{2}&2&\frac{3}{2}\\2&1&1\\\frac{3}{2}&1&\frac{1}{2} 
				\end{pmatrix}\cdot\begin{pmatrix} -1&1&1\\1&-2&1\\1&1&-3 \end{pmatrix}
				=\begin{pmatrix}
					1&0&0\\0&1&0\\0&0&1
				\end{pmatrix}
			\end{align*}
	\end{itemize}	
	\end{block}
\end{frame}

\begin{frame}\frametitle{\mytitle}
	\begin{block}{Rechenregeln}
	\begin{itemize}
		\item Sei $A$ eine $n\times n$-Matrix mit $\rk(A)=n$.
		\item Solche Matrizen nennt man auch \emph{invertierbar} oder \emph{regul\ae r}.
		\item Es gilt $A\cdot A^{-1}=\id_n$.
		\item Wenn $A$ invertierbar ist, dann ist auch $A^\trans$ invertierbar und
			\begin{align*}
				(A^\trans)^{-1}&=(A^{-1})^\trans
			\end{align*}
	\end{itemize}	
	\end{block}
\end{frame}

\begin{frame}\frametitle{\mytitle}
	\begin{block}{Rechenregeln}
	\begin{itemize}
		\item Wenn $A,B$ invertierbare $n\times n$-Matrizen sind, dann gilt
			\begin{align*}
				(A\cdot B)^{-1}=B^{-1}\cdot A^{-1}.
			\end{align*}
		\item Wenn $A$ invertierbar ist und $Au=y$, dann gilt
			\begin{align*}
				u=A^{-1}y.
			\end{align*}
	\end{itemize}	
	\end{block}
\end{frame}

\begin{frame}\frametitle{\mytitle}
	\begin{block}{Die allgemeine lineare Gruppe}
		\begin{itemize}
			\item Die Menge der invertierbaren $n\times n$-Matrizen wird mit $\GL(n)$ bezeichnet.
			\item Sie hei\ss t die \emph{allgemeine lineare Gruppe}.
			\item Zusammen mit der Matrixmultiplikation ist $\GL(n)$ eine Gruppe.
			\item $\GL(n)$ ist die Einheitengruppe des Rings $\RR^{n\times n}$.
			\item \alert{Erinnerung:} die Matrixmultiplikation ist \emph{nicht kommutativ!}
		\end{itemize}	
	\end{block}
\end{frame}

\begin{frame}\frametitle{\mytitle}
	\begin{block}{Zusammenfassung}
		\begin{itemize}
			\item Die Menge der invertierbaren $n\times n$-Matrizen wird mit $\GL(n)$ bezeichnet.
			\item Sie hei\ss t die \emph{allgemeine lineare Gruppe}.
			\item Zusammen mit der Matrixmultiplikation ist $\GL(n)$ eine Gruppe.
			\item $\GL(n)$ ist die Einheitengruppe des Rings $\RR^{n\times n}$.
			\item \alert{Erinnerung:} die Matrixmultiplikation ist \emph{nicht kommutativ!}
		\end{itemize}	
	\end{block}
\end{frame}

\end{document}
