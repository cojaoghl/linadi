\documentclass{beamer}
\usepackage{amsmath,graphics}
\usepackage{amssymb}

\usetheme{default}
\usepackage{xcolor}

\definecolor{solarizedBase03}{HTML}{002B36}
\definecolor{solarizedBase02}{HTML}{073642}
\definecolor{solarizedBase01}{HTML}{586e75}
\definecolor{solarizedBase00}{HTML}{657b83}
\definecolor{solarizedBase0}{HTML}{839496}
\definecolor{solarizedBase1}{HTML}{93a1a1}
\definecolor{solarizedBase2}{HTML}{EEE8D5}
\definecolor{solarizedBase3}{HTML}{FDF6E3}
\definecolor{solarizedYellow}{HTML}{B58900}
\definecolor{solarizedOrange}{HTML}{CB4B16}
\definecolor{solarizedRed}{HTML}{DC322F}
\definecolor{solarizedMagenta}{HTML}{D33682}
\definecolor{solarizedViolet}{HTML}{6C71C4}
%\definecolor{solarizedBlue}{HTML}{268BD2}
\definecolor{solarizedBlue}{HTML}{134676}
\definecolor{solarizedCyan}{HTML}{2AA198}
\definecolor{solarizedGreen}{HTML}{859900}
\definecolor{myBlue}{HTML}{162DB0}%{261CA4}
\setbeamercolor*{item}{fg=myBlue}
\setbeamercolor{normal text}{fg=solarizedBase03, bg=solarizedBase3}
\setbeamercolor{alerted text}{fg=myBlue}
\setbeamercolor{example text}{fg=myBlue, bg=solarizedBase3}
\setbeamercolor*{frametitle}{fg=solarizedRed}
\setbeamercolor*{title}{fg=solarizedRed}
\setbeamercolor{block title}{fg=myBlue, bg=solarizedBase3}
\setbeameroption{hide notes}
\setbeamertemplate{note page}[plain]
\beamertemplatenavigationsymbolsempty
\usefonttheme{professionalfonts}
\usefonttheme{serif}

\usepackage{fourier}

\def\vec#1{\mathchoice{\mbox{\boldmath$\displaystyle#1$}}
{\mbox{\boldmath$\textstyle#1$}}
{\mbox{\boldmath$\scriptstyle#1$}}
{\mbox{\boldmath$\scriptscriptstyle#1$}}}
\definecolor{OwnGrey}{rgb}{0.560,0.000,0.000} % #999999
\definecolor{OwnBlue}{rgb}{0.121,0.398,0.711} % #1f64b0
\definecolor{red4}{rgb}{0.5,0,0}
\definecolor{blue4}{rgb}{0,0,0.5}
\definecolor{Blue}{rgb}{0,0,0.66}
\definecolor{LightBlue}{rgb}{0.9,0.9,1}
\definecolor{Green}{rgb}{0,0.5,0}
\definecolor{LightGreen}{rgb}{0.9,1,0.9}
\definecolor{Red}{rgb}{0.9,0,0}
\definecolor{LightRed}{rgb}{1,0.9,0.9}
\definecolor{White}{gray}{1}
\definecolor{Black}{gray}{0}
\definecolor{LightGray}{gray}{0.8}
\definecolor{Orange}{rgb}{0.1,0.2,1}
\setbeamerfont{sidebar right}{size=\scriptsize}
\setbeamercolor{sidebar right}{fg=Black}

\renewcommand{\emph}[1]{{\textcolor{solarizedRed}{\itshape #1}}}

\newcommand\cA{\mathcal A}
\newcommand\cB{\mathcal B}
\newcommand\cC{\mathcal C}
\newcommand\cD{\mathcal D}
\newcommand\cE{\mathcal E}
\newcommand\cF{\mathcal F}
\newcommand\cG{\mathcal G}
\newcommand\cH{\mathcal H}
\newcommand\cI{\mathcal I}
\newcommand\cJ{\mathcal J}
\newcommand\cK{\mathcal K}
\newcommand\cL{\mathcal L}
\newcommand\cM{\mathcal M}
\newcommand\cN{\mathcal N}
\newcommand\cO{\mathcal O}
\newcommand\cP{\mathcal P}
\newcommand\cQ{\mathcal Q}
\newcommand\cR{\mathcal R}
\newcommand\cS{\mathcal S}
\newcommand\cT{\mathcal T}
\newcommand\cU{\mathcal U}
\newcommand\cV{\mathcal V}
\newcommand\cW{\mathcal W}
\newcommand\cX{\mathcal X}
\newcommand\cY{\mathcal Y}
\newcommand\cZ{\mathcal Z}

\newcommand\fA{\mathfrak A}
\newcommand\fB{\mathfrak B}
\newcommand\fC{\mathfrak C}
\newcommand\fD{\mathfrak D}
\newcommand\fE{\mathfrak E}
\newcommand\fF{\mathfrak F}
\newcommand\fG{\mathfrak G}
\newcommand\fH{\mathfrak H}
\newcommand\fI{\mathfrak I}
\newcommand\fJ{\mathfrak J}
\newcommand\fK{\mathfrak K}
\newcommand\fL{\mathfrak L}
\newcommand\fM{\mathfrak M}
\newcommand\fN{\mathfrak N}
\newcommand\fO{\mathfrak O}
\newcommand\fP{\mathfrak P}
\newcommand\fQ{\mathfrak Q}
\newcommand\fR{\mathfrak R}
\newcommand\fS{\mathfrak S}
\newcommand\fT{\mathfrak T}
\newcommand\fU{\mathfrak U}
\newcommand\fV{\mathfrak V}
\newcommand\fW{\mathfrak W}
\newcommand\fX{\mathfrak X}
\newcommand\fY{\mathfrak Y}
\newcommand\fZ{\mathfrak Z}

\newcommand\fa{\mathfrak a}
\newcommand\fb{\mathfrak b}
\newcommand\fc{\mathfrak c}
\newcommand\fd{\mathfrak d}
\newcommand\fe{\mathfrak e}
\newcommand\ff{\mathfrak f}
\newcommand\fg{\mathfrak g}
\newcommand\fh{\mathfrak h}
%\newcommand\fi{\mathfrak i}
\newcommand\fj{\mathfrak j}
\newcommand\fk{\mathfrak k}
\newcommand\fl{\mathfrak l}
\newcommand\fm{\mathfrak m}
\newcommand\fn{\mathfrak n}
\newcommand\fo{\mathfrak o}
\newcommand\fp{\mathfrak p}
\newcommand\fq{\mathfrak q}
\newcommand\fr{\mathfrak r}
\newcommand\fs{\mathfrak s}
\newcommand\ft{\mathfrak t}
\newcommand\fu{\mathfrak u}
\newcommand\fv{\mathfrak v}
\newcommand\fw{\mathfrak w}
\newcommand\fx{\mathfrak x}
\newcommand\fy{\mathfrak y}
\newcommand\fz{\mathfrak z}

\newcommand\vA{\vec A}
\newcommand\vB{\vec B}
\newcommand\vC{\vec C}
\newcommand\vD{\vec D}
\newcommand\vE{\vec E}
\newcommand\vF{\vec F}
\newcommand\vG{\vec G}
\newcommand\vH{\vec H}
\newcommand\vI{\vec I}
\newcommand\vJ{\vec J}
\newcommand\vK{\vec K}
\newcommand\vL{\vec L}
\newcommand\vM{\vec M}
\newcommand\vN{\vec N}
\newcommand\vO{\vec O}
\newcommand\vP{\vec P}
\newcommand\vQ{\vec Q}
\newcommand\vR{\vec R}
\newcommand\vS{\vec S}
\newcommand\vT{\vec T}
\newcommand\vU{\vec U}
\newcommand\vV{\vec V}
\newcommand\vW{\vec W}
\newcommand\vX{\vec X}
\newcommand\vY{\vec Y}
\newcommand\vZ{\vec Z}

\newcommand\va{\vec a}
\newcommand\vb{\vec b}
\newcommand\vc{\vec c}
\newcommand\vd{\vec d}
\newcommand\ve{\vec e}
\newcommand\vf{\vec f}
\newcommand\vg{\vec g}
\newcommand\vh{\vec h}
\newcommand\vi{\vec i}
\newcommand\vj{\vec j}
\newcommand\vk{\vec k}
\newcommand\vl{\vec l}
\newcommand\vm{\vec m}
\newcommand\vn{\vec n}
\newcommand\vo{\vec o}
\newcommand\vp{\vec p}
\newcommand\vq{\vec q}
\newcommand\vr{\vec r}
\newcommand\vs{\vec s}
\newcommand\vt{\vec t}
\newcommand\vu{\vec u}
\newcommand\vv{\vec v}
\newcommand\vw{\vec w}
\newcommand\vx{\vec x}
\newcommand\vy{\vec y}
\newcommand\vz{\vec z}

\renewcommand\AA{\mathbb A}
\newcommand\NN{\mathbb N}
\newcommand\ZZ{\mathbb Z}
\newcommand\PP{\mathbb P}
\newcommand\QQ{\mathbb Q}
\newcommand\RR{\mathbb R}
\renewcommand\SS{\mathbb S}
\newcommand\CC{\mathbb C}

\newcommand{\ord}{\mathrm{ord}}
\newcommand{\id}{\mathrm{id}}
\newcommand{\pr}{\mathrm{P}}
\newcommand{\Vol}{\mathrm{vol}}
\newcommand\norm[1]{\left\|{#1}\right\|} 
\newcommand\sign{\mathrm{sign}}
\newcommand{\eps}{\varepsilon}
\newcommand{\abs}[1]{\left|#1\right|}
\newcommand\bc[1]{\left({#1}\right)} 
\newcommand\cbc[1]{\left\{{#1}\right\}} 
\newcommand\bcfr[2]{\bc{\frac{#1}{#2}}} 
\newcommand{\bck}[1]{\left\langle{#1}\right\rangle} 
\newcommand\brk[1]{\left\lbrack{#1}\right\rbrack} 
\newcommand\scal[2]{\bck{{#1},{#2}}} 
\newcommand{\vecone}{\mathbb{1}}
\newcommand{\tensor}{\otimes}
\newcommand{\diag}{\mathrm{diag}}
\newcommand{\ggt}{\mathrm{ggT}}
\newcommand{\kgv}{\mathrm{kgV}}
\newcommand{\trans}{\top}

\newcommand{\Karonski}{Karo\'nski}
\newcommand{\Erdos}{Erd\H{o}s}
\newcommand{\Renyi}{R\'enyi}
\newcommand{\Lovasz}{Lov\'asz}
\newcommand{\Juhasz}{Juh\'asz}
\newcommand{\Bollobas}{Bollob\'as}
\newcommand{\Furedi}{F\"uredi}
\newcommand{\Komlos}{Koml\'os}
\newcommand{\Luczak}{\L uczak}
\newcommand{\Kucera}{Ku\v{c}era}
\newcommand{\Szemeredi}{Szemer\'edi}

\renewcommand{\ae}{\"a}
\renewcommand{\oe}{\"o}
\newcommand{\ue}{\"u}
\newcommand{\Ae}{\"A}
\newcommand{\Oe}{\"O}
\newcommand{\Ue}{\"U}

\newcommand{\im}{\mathrm{im}}
\newcommand{\rrk}{\mathrm{zrg}}
\newcommand{\crk}{\mathrm{srg}}
\newcommand{\rk}{\mathrm{rg}}

\newcommand{\mytitle}{Der Rang}

\title[Linadi]{\mytitle}
\author[Amin Coja-Oghlan]{Amin Coja-Oghlan}
\institute[Frankfurt]{JWGUFFM}
\date{}

\begin{document}

\frame[plain]{\titlepage}

\begin{frame}\frametitle{\mytitle}
	\begin{block}{Der Zeilenrang}
	\begin{itemize}
		\item Sei $A$ eine $m\times n$-Matrix.
		\item Mit dem Gau\ss verfahren k\oe nnen wir $A$ in Zeilenstufenform bringen.
		\item Sei $k$ die verbleibende Zahl von Zeilen, die nicht nur Nullen enthalten.
		\item Wir nennen $k$ den \emph{Zeilenrang} von $A$:
			\begin{align*}
				\rrk(A)&=k.
			\end{align*}
	\end{itemize}
	\end{block}
\end{frame}

\begin{frame}\frametitle{\mytitle}
	\begin{block}{Beispiel}
		\begin{itemize}
			\item Die Matrix
				\begin{align*}
					A&=\begin{pmatrix}1&-1&2\\1&7&0\\3&-4&6\end{pmatrix}
				\end{align*}
				hat die Zeilenstufenform
				\begin{align*}
					B&=\begin{pmatrix} 1&-1&2\\0&1&0\\0&0&1 \end{pmatrix}
				\end{align*}
				(vgl.\ die letzte Vorlesung).
			\item Folglich gilt $\rrk(A)=3$.
		\end{itemize}
	\end{block}
\end{frame}

\begin{frame}\frametitle{\mytitle}
	\begin{block}{Beispiel}
		\begin{itemize}
			\item Sei
				\begin{align*}
					A&=\begin{pmatrix}
						-1&1&1\\2&2&-4\\-3&-3&6
					\end{pmatrix}
					\end{align*}
				\item In der letzten VL haben wir $A$ in Zeilenstufenform gebracht:
					\begin{align*}
						B=\begin{pmatrix} -1&1&1\\0&4&2\\0&0&0 \end{pmatrix}
					\end{align*}
			\item Zwei der Zeilen enthalten nicht ausschlie\ss lich Nullen.
			\item Also gilt $\rrk(A)=2$.
			\end{itemize}
		\end{block}
\end{frame}

\begin{frame}\frametitle{\mytitle}
	\begin{block}{Proposition}
		Wenn die Matrizen $A,B$ zeilen\ae quivalent sind, dann gilt $$\rrk(A)=\rrk(B).$$	
	\end{block}
\end{frame}

\begin{frame}\frametitle{\mytitle}
	\begin{block}{Spaltenumformungen}
		\begin{itemize}
			\item Den Zeilenumformungen entsprechend kann man Spaltenumformungen einf\ue hren.
			\item Sei also $A$ eine $m\times n$-Matrix; das Ergebnis der Umformung ist jeweils wieder eine $m\times n$-Matrix $B$.
			\item \alert{Spaltenvertauschungen:} wir erhalten $B$ durch Vertauschen von zwei Spalten von $A$.
			\item \alert{Skalierung:} wir erhalten $B$, indem wir die $h$-te Spalte von $A$ mit $c\in\RR\setminus\cbc 0$ multiplizieren.
			\item \alert{Pivot:} $B$ entsteht aus $A$, indem wir ein Vielfaches einer Spalte zu einer anderen addieren.
		\end{itemize}
	\end{block}
\end{frame}

\begin{frame}\frametitle{\mytitle}
	\begin{block}{Definition}
		Zwei Matrizen $A$ und $B$ hei\ss en \emph{spalten\ae quivalent}, wenn $B$ aus $A$ durch eine Abfolge von Spaltenumformungen entsteht.	
	\end{block}
\end{frame}

\begin{frame}\frametitle{\mytitle}
	\begin{block}{Spaltenstufenform}
		Eine $m\times n$-Matrix $A=(a_{ij})$ ist in \emph{Spaltenstufenform}, wenn es ein $k\in\{0,\ldots,n\}$ und Zahlen $h_1,\ldots,h_k\in\{1,\ldots,m\}$ gibt, so da\ss
	\begin{itemize}
		\item $h_1<h_2<\cdots<h_k$,
		\item f\ue r alle $j\in\{1,\ldots,k\}$ gilt $a_{h_j\,j}\neq0$,
		\item f\ue r alle $j\in\{1,\ldots,k\}$ und alle $1\leq i<h_j$ gilt $a_{ij}=0$,
		\item f\ue r alle $k<j\leq n$ und alle $1\leq i\leq m$ gilt $a_{ij}=0$.
	\end{itemize}
	\emph{Beispiel:}
	\begin{align*}
	\begin{pmatrix}
		-1&0&0&0&0\\
		2&0&0&0&0\\
		3&5&0&0&0\\
		4&-6&7&0&0\\
		0&-3&0&0&0
	\end{pmatrix}&&
	k=3,\ h_1=1,\ h_2=3,\ h_3=4
	\end{align*}
	\end{block}
\end{frame}

\begin{frame}\frametitle{\mytitle}
	\begin{block}{Spaltenstufenform}
	\begin{itemize}
	\item Das Gau\ss verfahren kann man auf Spaltenumformungen \ue bertragen.
	\item Wir k\oe nnen damit jede Matrix $A$ in eine Matrix in Spaltenstufenform \ue berf\ue hren.
	\item \emph{Alternative:} transponiere die Matrix $A$; bringe die transponierte Matrix $A^\trans$ in Zeilenstufenform $B$; dann ist $B^\trans$ in Spaltenstufenform und spalten\ae quivalent zu $A$.
	\end{itemize}
	\end{block}
\end{frame}

\begin{frame}\frametitle{\mytitle}
	\begin{block}{Der Spaltenrang}
	\begin{itemize}
		\item Sei $A$ eine $m\times n$-Matrix und $B$ eine zu $A$ spalten\ae quivalente Matrix in Spaltenstufenform.
		\item Der \emph{Spaltenrang} von $A$ ist definiert als die Anzahl der Spalten von $B$, die nicht nur Nullen enthalten.
		\item Mit anderen Worten: der Spaltenrang von $A$ ist der Zeilenrang von $A^\trans$:
			\begin{align*}
				\crk(A)=\rrk(A^\trans)
			\end{align*}
	\end{itemize}
	\end{block}
\end{frame}

\begin{frame}\frametitle{\mytitle}
	\begin{block}{Beispiel}
	\begin{itemize}
		\item Sei $A$ die $3\times 3$-Matrix 
			\begin{align*}
				A&=\begin{pmatrix} 1&5&2\\0&2&1\\1&1&0 \end{pmatrix}
			\end{align*}
		\item Wir bringen $A$ in Spaltenstufenform.
		\item Dazu subtrahieren wir 5 Mal die erste Spalte von der zweiten Spalte:
			\begin{align*}
				\begin{pmatrix} 1&0&2\\0&2&1\\1&-4&0 \end{pmatrix}
			\end{align*}
	\end{itemize}
	\end{block}
\end{frame}

\begin{frame}\frametitle{\mytitle}
	\begin{block}{Beispiel}
	\begin{itemize}
		\item Anschlie\ss end subtrahieren wir 2 Mal die erste Spalte von der dritten Spalte:
			\begin{align*}
				\begin{pmatrix} 1&0&0\\0&2&1\\1&-4&-2 \end{pmatrix}
			\end{align*}
	\end{itemize}
	\end{block}
\end{frame}

\begin{frame}\frametitle{\mytitle}
	\begin{block}{Beispiel}
	\begin{itemize}
		\item Jetzt subtrahieren wir das doppelte der letzten Spalte von der zweiten Spalte:
			\begin{align*}
				\begin{pmatrix} 1&0&0\\0&0&1\\1&0&-2 \end{pmatrix}
			\end{align*}
		\item Schlie\ss lich vertauschen wir die dritte und die zweite Spalte:
			\begin{align*}
				B=\begin{pmatrix} 1&0&0\\0&1&0\\1&-2&0 \end{pmatrix}
			\end{align*}
		\item Die Matrix ist jetzt in Spaltenstufenform.
		\item Wir lesen ab, da\ss\ $\crk(A)=\crk(B)=2$.
	\end{itemize}
	\end{block}
\end{frame}

\begin{frame}\frametitle{\mytitle}
	\begin{block}{Beispiel (alternative Herleitung)}
	\begin{itemize}
		\item Wie zuvor sei $A$ die $3\times 3$-Matrix 
			\begin{align*}
				A&=\begin{pmatrix} 1&5&2\\0&2&1\\1&1&0 \end{pmatrix}
			\end{align*}
		\item Wir bringen $A$ in Spaltenstufenform.
		\item Dazu bringen wir
\begin{align*} 
	A^\trans&=\begin{pmatrix} 1&0&1\\ 5&2&1\\ 2&1&0 \end{pmatrix}
			\end{align*}
		in Zeilenstufenform.
	\end{itemize}
	\end{block}
\end{frame}

\begin{frame}\frametitle{\mytitle}
	\begin{block}{Beispiel}
	\begin{itemize}
		\item Subtrahiere 5 Mal die erste Zeile von der zweiten Zeile:
			\begin{align*}
\begin{pmatrix} 1&0&1\\ 0&2&-4\\ 2&1&0 \end{pmatrix}
			\end{align*}
		\item Subtrahiere 2 Mal die erste Zeile von der dritten Zeile:
			\begin{align*}
\begin{pmatrix} 1&0&1\\ 0&2&-4\\ 0&1&-2 \end{pmatrix}
			\end{align*}
	\end{itemize}
	\end{block}
\end{frame}

\begin{frame}\frametitle{\mytitle}
	\begin{block}{Beispiel}
	\begin{itemize}
		\item Jetzt subtrahieren wir das doppelte der letzten Zeile von der zweiten Zeile:
			\begin{align*}
\begin{pmatrix} 1&0&1\\ 0&0&0\\ 0&1&-2 \end{pmatrix}
			\end{align*}
		\item Schlie\ss lich vertauschen wir die dritte und die zweite Zeile:
			\begin{align*}
B=\begin{pmatrix} 1&0&1\\  0&1&-2\\ 0&0&0\end{pmatrix}
			\end{align*}
		\item Die Matrix ist jetzt in Zeilenstufenform.
	\end{itemize}
	\end{block}
\end{frame}

\begin{frame}\frametitle{\mytitle}
	\begin{block}{Beispiel}
	\begin{itemize}
		\item Die Matrix ist jetzt in Zeilenstufenform.
		\item Um die Spaltenstufenform von $A$ zu erhalten, transponieren wir $B$:
\begin{align*}
				B^\trans=\begin{pmatrix} 1&0&0\\0&1&0\\1&-2&0 \end{pmatrix}
			\end{align*}
		\item Wie zuvor lesen wir $\crk(A)=2$ ab.
	\end{itemize}
	\end{block}
\end{frame}

\begin{frame}\frametitle{\mytitle}
	\begin{block}{Satz}
		F\ue r jede Matrix $A$ gilt $\rrk(A)=\crk(A)$.
	\end{block}
	\begin{block}{Definition}
		Der \emph{Rang} einer Matrix $A$ ist definiert als
		\begin{align*}
			\rk(A)=\rrk(A)=\crk(A).
		\end{align*}
	\end{block}
\end{frame}

\begin{frame}\frametitle{\mytitle}
	\begin{block}{Berechnen des Rangs}
	\begin{itemize}
	\item Weil Zeilenrang und Spaltenrang \ue bereinstimmen, gilt
		\begin{align*}
			\rk(A)&=\rk(A^\trans).
		\end{align*}
	\item Deshalb d\ue rfen wir bei der Berechnung des Rangs Zeilen- und Spaltenumformungen \alert{mischen}!
	\end{itemize}
	\end{block}
\end{frame}

\begin{frame}\frametitle{\mytitle}
	\begin{block}{Beispiel}
	\begin{itemize}
		\item Sei
			\begin{align*}
				A&=\begin{pmatrix}
					1&8&8&1\\
					0&1&1&0\\
					0&1&1&0\\
					1&8&8&1
				\end{pmatrix}
			\end{align*}
		\item Wir subtrahieren die erste Zeile von der letzten Zeile:
\begin{align*}
				\begin{pmatrix}
					1&8&8&1\\
					0&1&1&0\\
					0&1&1&0\\
					0&0&0&0
				\end{pmatrix}
			\end{align*}
	\end{itemize}
	\end{block}
\end{frame}

\begin{frame}\frametitle{\mytitle}
	\begin{block}{Beispiel}
	\begin{itemize}
		\item Jezte subtrahiere die zweite Spalte von der dritten Spalte und die erste Spalte von der letzten Spalte:
\begin{align*}
				\begin{pmatrix}
					1&8&0&0\\
					0&1&0&0\\
					0&1&0&0\\
					0&0&0&0
				\end{pmatrix}
			\end{align*}
		\item Als n\ae chstes subtrahiere die zweite Zeile von der dritten:
			\begin{align*}
				\begin{pmatrix}
					1&8&0&0\\
					0&1&0&0\\
					0&0&0&0\\
					0&0&0&0
				\end{pmatrix}
			\end{align*}
		\item Die Matrix ist in Zeilenstufenform und $\rk(A)=2$.
	\end{itemize}
	\end{block}
\end{frame}

\begin{frame}\frametitle{\mytitle}
	\begin{block}{Definition}
		Zwei $m\times n$-Matrizen $A,B$ hei\ss en \emph{\ae quivaelnt}, falls $\rk(A)=\rk(B)$.
	\end{block}
\begin{block}{Proposition}
		Zwei $m\times n$-Matrizen $A,B$ sind genau dann \ae quivalent, wenn $A$ durch Zeilen- und Spaltenumformungen in $B$ \ue berf\ue hrt werden kann.
	\end{block}
\end{frame}

\begin{frame}\frametitle{\mytitle}
\begin{block}{Proposition}
	Jede $m\times n$-Matrix $A=(a_{ij})$ ist \ae quivalent zu der $m\times n$-Matrix $B=(b_{ij})$ mit Eintr\ae gen
	\begin{align*}
		b_{ij}&=\begin{cases}
			1&\mbox{ falls }i=j\leq\rk(A)\\
			0&\mbox{ sonst}
		\end{cases}
	\end{align*}
	\emph{$B$ ist eine Matrix, deren erste $\rk(A)$ Diagonaleintr\ae ge gleich Eins sind und deren andere Eintr\ae ge gleich Null sind.}
	\end{block}
\end{frame}

\begin{frame}\frametitle{\mytitle}
\begin{block}{Korollar}
	Sei $A$ eine $m\times n$-Matrix.
	Dann gelten die Ungleichungen
	\begin{align*}
		\rk(A)&\geq 0&\rk(A)&\leq m&\rk(A)&\leq n. 
	\end{align*}
	\end{block}
	\begin{block}{Definition}
		Eine $m\times n$-Matrix $A$ hat \emph{vollen Rang}, wenn $\rk(A)=m$ oder $\rk(A)=n$.
	\end{block}
\end{frame}

\begin{frame}\frametitle{\mytitle}
\begin{block}{Lineare Gleichungssysteme und der Rang}
\begin{itemize}
	\item Sei $A$ eine $m\times n$-Matrix und $y\in\RR^m$.
	\item Sei $(A\ y)$ die Matrix, die aus $A$ entsteht, indem $y$ als Spalte hinzugef\ue gt wird.
	\item Es gilt
		\begin{align*}
			y&\in\im(A)&\Leftrightarrow&&\rk(A)&=\rk(A\ y).
		\end{align*}
	\item Mit dem Rang k\oe nnen wir also einfach feststellen, ob das lineare Gleichungssystem $ Au=y $ eine L\oe sung besitzt.
\end{itemize}
	\end{block}
\end{frame}

\begin{frame}\frametitle{\mytitle}
\begin{block}{Zusammenfassung}
\begin{itemize}
	\item Der Zeilenrang einer Matrix ist die Anzahl von Null verschiedener Zeilen, die am Ende des Gau\ss verfahrens verbleiben.
	\item Der Spaltenrang einer Matrix ist der Zeilenrang der transponierten Matrix.
	\item Zeilenrang und Spaltenrang stimmen \ue berein.
	\item Um den Rang zu berechnen, d\ue rfen Zeilen- und Spaltenumformungen gemischt werden.
	\item Mit dem Rang kann man bestimmen, ob ein lineares Gleichungssystem l\oe sbar ist.
\end{itemize}
	\end{block}
\end{frame}

\end{document}
