\documentclass{beamer}
\usepackage{amsmath,graphics}
\usepackage{amssymb}

\usetheme{default}
\usepackage{xcolor}

\definecolor{solarizedBase03}{HTML}{002B36}
\definecolor{solarizedBase02}{HTML}{073642}
\definecolor{solarizedBase01}{HTML}{586e75}
\definecolor{solarizedBase00}{HTML}{657b83}
\definecolor{solarizedBase0}{HTML}{839496}
\definecolor{solarizedBase1}{HTML}{93a1a1}
\definecolor{solarizedBase2}{HTML}{EEE8D5}
\definecolor{solarizedBase3}{HTML}{FDF6E3}
\definecolor{solarizedYellow}{HTML}{B58900}
\definecolor{solarizedOrange}{HTML}{CB4B16}
\definecolor{solarizedRed}{HTML}{DC322F}
\definecolor{solarizedMagenta}{HTML}{D33682}
\definecolor{solarizedViolet}{HTML}{6C71C4}
%\definecolor{solarizedBlue}{HTML}{268BD2}
\definecolor{solarizedBlue}{HTML}{134676}
\definecolor{solarizedCyan}{HTML}{2AA198}
\definecolor{solarizedGreen}{HTML}{859900}
\definecolor{myBlue}{HTML}{162DB0}%{261CA4}
\setbeamercolor*{item}{fg=myBlue}
\setbeamercolor{normal text}{fg=solarizedBase03, bg=solarizedBase3}
\setbeamercolor{alerted text}{fg=myBlue}
\setbeamercolor{example text}{fg=myBlue, bg=solarizedBase3}
\setbeamercolor*{frametitle}{fg=solarizedRed}
\setbeamercolor*{title}{fg=solarizedRed}
\setbeamercolor{block title}{fg=myBlue, bg=solarizedBase3}
\setbeameroption{hide notes}
\setbeamertemplate{note page}[plain]
\beamertemplatenavigationsymbolsempty
\usefonttheme{professionalfonts}
\usefonttheme{serif}

\usepackage{fourier}

\def\vec#1{\mathchoice{\mbox{\boldmath$\displaystyle#1$}}
{\mbox{\boldmath$\textstyle#1$}}
{\mbox{\boldmath$\scriptstyle#1$}}
{\mbox{\boldmath$\scriptscriptstyle#1$}}}

\definecolor{OwnGrey}{rgb}{0.560,0.000,0.000} % #999999
\definecolor{OwnBlue}{rgb}{0.121,0.398,0.711} % #1f64b0
\definecolor{red4}{rgb}{0.5,0,0}
\definecolor{blue4}{rgb}{0,0,0.5}
\definecolor{Blue}{rgb}{0,0,0.66}
\definecolor{LightBlue}{rgb}{0.9,0.9,1}
\definecolor{Green}{rgb}{0,0.5,0}
\definecolor{LightGreen}{rgb}{0.9,1,0.9}
\definecolor{Red}{rgb}{0.9,0,0}
\definecolor{LightRed}{rgb}{1,0.9,0.9}
\definecolor{White}{gray}{1}
\definecolor{Black}{gray}{0}
\definecolor{LightGray}{gray}{0.8}
\definecolor{Orange}{rgb}{0.1,0.2,1}
\setbeamerfont{sidebar right}{size=\scriptsize}
\setbeamercolor{sidebar right}{fg=Black}

\renewcommand{\emph}[1]{{\textcolor{solarizedRed}{\itshape #1}}}

\newcommand\cA{\mathcal A}
\newcommand\cB{\mathcal B}
\newcommand\cC{\mathcal C}
\newcommand\cD{\mathcal D}
\newcommand\cE{\mathcal E}
\newcommand\cF{\mathcal F}
\newcommand\cG{\mathcal G}
\newcommand\cH{\mathcal H}
\newcommand\cI{\mathcal I}
\newcommand\cJ{\mathcal J}
\newcommand\cK{\mathcal K}
\newcommand\cL{\mathcal L}
\newcommand\cM{\mathcal M}
\newcommand\cN{\mathcal N}
\newcommand\cO{\mathcal O}
\newcommand\cP{\mathcal P}
\newcommand\cQ{\mathcal Q}
\newcommand\cR{\mathcal R}
\newcommand\cS{\mathcal S}
\newcommand\cT{\mathcal T}
\newcommand\cU{\mathcal U}
\newcommand\cV{\mathcal V}
\newcommand\cW{\mathcal W}
\newcommand\cX{\mathcal X}
\newcommand\cY{\mathcal Y}
\newcommand\cZ{\mathcal Z}

\newcommand\fA{\mathfrak A}
\newcommand\fB{\mathfrak B}
\newcommand\fC{\mathfrak C}
\newcommand\fD{\mathfrak D}
\newcommand\fE{\mathfrak E}
\newcommand\fF{\mathfrak F}
\newcommand\fG{\mathfrak G}
\newcommand\fH{\mathfrak H}
\newcommand\fI{\mathfrak I}
\newcommand\fJ{\mathfrak J}
\newcommand\fK{\mathfrak K}
\newcommand\fL{\mathfrak L}
\newcommand\fM{\mathfrak M}
\newcommand\fN{\mathfrak N}
\newcommand\fO{\mathfrak O}
\newcommand\fP{\mathfrak P}
\newcommand\fQ{\mathfrak Q}
\newcommand\fR{\mathfrak R}
\newcommand\fS{\mathfrak S}
\newcommand\fT{\mathfrak T}
\newcommand\fU{\mathfrak U}
\newcommand\fV{\mathfrak V}
\newcommand\fW{\mathfrak W}
\newcommand\fX{\mathfrak X}
\newcommand\fY{\mathfrak Y}
\newcommand\fZ{\mathfrak Z}

\newcommand\fa{\mathfrak a}
\newcommand\fb{\mathfrak b}
\newcommand\fc{\mathfrak c}
\newcommand\fd{\mathfrak d}
\newcommand\fe{\mathfrak e}
\newcommand\ff{\mathfrak f}
\newcommand\fg{\mathfrak g}
\newcommand\fh{\mathfrak h}
%\newcommand\fi{\mathfrak i}
\newcommand\fj{\mathfrak j}
\newcommand\fk{\mathfrak k}
\newcommand\fl{\mathfrak l}
\newcommand\fm{\mathfrak m}
\newcommand\fn{\mathfrak n}
\newcommand\fo{\mathfrak o}
\newcommand\fp{\mathfrak p}
\newcommand\fq{\mathfrak q}
\newcommand\fr{\mathfrak r}
\newcommand\fs{\mathfrak s}
\newcommand\ft{\mathfrak t}
\newcommand\fu{\mathfrak u}
\newcommand\fv{\mathfrak v}
\newcommand\fw{\mathfrak w}
\newcommand\fx{\mathfrak x}
\newcommand\fy{\mathfrak y}
\newcommand\fz{\mathfrak z}

\newcommand\vA{\vec A}
\newcommand\vB{\vec B}
\newcommand\vC{\vec C}
\newcommand\vD{\vec D}
\newcommand\vE{\vec E}
\newcommand\vF{\vec F}
\newcommand\vG{\vec G}
\newcommand\vH{\vec H}
\newcommand\vI{\vec I}
\newcommand\vJ{\vec J}
\newcommand\vK{\vec K}
\newcommand\vL{\vec L}
\newcommand\vM{\vec M}
\newcommand\vN{\vec N}
\newcommand\vO{\vec O}
\newcommand\vP{\vec P}
\newcommand\vQ{\vec Q}
\newcommand\vR{\vec R}
\newcommand\vS{\vec S}
\newcommand\vT{\vec T}
\newcommand\vU{\vec U}
\newcommand\vV{\vec V}
\newcommand\vW{\vec W}
\newcommand\vX{\vec X}
\newcommand\vY{\vec Y}
\newcommand\vZ{\vec Z}

\newcommand\va{\vec a}
\newcommand\vb{\vec b}
\newcommand\vc{\vec c}
\newcommand\vd{\vec d}
\newcommand\ve{\vec e}
\newcommand\vf{\vec f}
\newcommand\vg{\vec g}
\newcommand\vh{\vec h}
\newcommand\vi{\vec i}
\newcommand\vj{\vec j}
\newcommand\vk{\vec k}
\newcommand\vl{\vec l}
\newcommand\vm{\vec m}
\newcommand\vn{\vec n}
\newcommand\vo{\vec o}
\newcommand\vp{\vec p}
\newcommand\vq{\vec q}
\newcommand\vr{\vec r}
\newcommand\vs{\vec s}
\newcommand\vt{\vec t}
\newcommand\vu{\vec u}
\newcommand\vv{\vec v}
\newcommand\vw{\vec w}
\newcommand\vx{\vec x}
\newcommand\vy{\vec y}
\newcommand\vz{\vec z}

\newcommand\NN{\mathbb N}
\newcommand\ZZ{\mathbb Z}
\newcommand\PP{\mathbb P}
\newcommand\QQ{\mathbb Q}
\newcommand\RR{\mathbb R}
\newcommand\CC{\mathbb C}

\newcommand{\pr}{\mathrm{P}}
\newcommand{\Vol}{\mathrm{vol}}
\newcommand\norm[1]{\left\|{#1}\right\|} 
\newcommand\sign{\mathrm{sign}}
\newcommand{\eps}{\varepsilon}
\newcommand{\abs}[1]{\left|#1\right|}
\newcommand\bc[1]{\left({#1}\right)} 
\newcommand\cbc[1]{\left\{{#1}\right\}} 
\newcommand\bcfr[2]{\bc{\frac{#1}{#2}}} 
\newcommand{\bck}[1]{\left\langle{#1}\right\rangle} 
\newcommand\brk[1]{\left\lbrack{#1}\right\rbrack} 
\newcommand\scal[2]{\bck{{#1},{#2}}} 
\newcommand{\vecone}{\mathbb{1}}
\newcommand{\tensor}{\otimes}
\newcommand{\diag}{\mathrm{diag}}
\newcommand{\ggt}{\mathrm{ggT}}
\newcommand{\kgv}{\mathrm{kgV}}

\newcommand{\Karonski}{Karo\'nski}
\newcommand{\Erdos}{Erd\H{o}s}
\newcommand{\Renyi}{R\'enyi}
\newcommand{\Lovasz}{Lov\'asz}
\newcommand{\Juhasz}{Juh\'asz}
\newcommand{\Bollobas}{Bollob\'as}
\newcommand{\Furedi}{F\"uredi}
\newcommand{\Komlos}{Koml\'os}
\newcommand{\Luczak}{\L uczak}
\newcommand{\Kucera}{Ku\v{c}era}
\newcommand{\Szemeredi}{Szemer\'edi}

\renewcommand{\ae}{\"a}
\renewcommand{\oe}{\"o}
\newcommand{\ue}{\"u}
\newcommand{\Ae}{\"A}
\newcommand{\Oe}{\"O}
\newcommand{\Ue}{\"U}

\title[Linadi]{Primzahlen}
\author[Amin Coja-Oghlan]{Amin Coja-Oghlan}
\institute[Frankfurt]{JWGUFFM}
\date{}

\begin{document}

\frame[plain]{\titlepage}

\begin{frame}\frametitle{Unzerlegbare Zahlen}
	\begin{block}{Definition}
		Eine Zahl $z\in\ZZ\setminus\{1,-1\}$ hei\ss t \emph{unzerlegbar}, falls $y\nmid z$ f\ue r alle $1<y<|z|$.
	\end{block}

	\bigskip
	{\itshape Die Zahl $z$ ist also genau dann unzerlegbar, wenn $\pm1$ und $\pm z$ ihre einzigen Teiler sind.}
\end{frame}

\begin{frame}\frametitle{Unzerlegbare Zahlen}
	\begin{block}{Beispiele}
		\begin{itemize}
			\item Die Zahl $13$ ist unzerlegbar, wie man durch Ausprobieren aller m\oe glichen Teiler feststellt.
			\item Die Zahl $0$ ist zerlegbar, weil sie von jeder anderen Zahl geteilt wird.
			\item Die Zahl $-2$ ist unzerlegbar.
			\item Die Zahl $4$ ist zerlegbar, weil sie von $2$ geteilt wird.
		\end{itemize}
	\end{block}
\end{frame}

\begin{frame}\frametitle{Unzerlegbare Zahlen}
	\begin{block}{Lemma}
		Jede Zahl $z\in\ZZ\setminus\{0,1,-1\}$ hat einen unzerlegbarer Teiler.
	\end{block}
	\begin{block}{Beweis}
		\begin{itemize}
			\item Es gen\ue gt, dies f\ue r $z>1$ zu beweisen.
			\item Wir f\ue hren Induktion.
			\item Induktionsanfang: die Zahl $2$ ist selbst unzerlegbar.
			\item Ist allgemein $z$ unzerlegbar, so ist $z$ selbst der gesuchte Teiler.
			\item Sonst hat $z$ einen Teiler $1<y<z$.
			\item Nach Induktion hat $y$ einen unzerlegbaren Teiler $x$.
			\item Dieser teilt auch $z$.
		\end{itemize}	
	\end{block}
\end{frame}

\begin{frame}\frametitle{Unzerlegbare Zahlen}
	\begin{block}{Satz}
		Es gibt unendlich viele unzerlegbare Zahlen.
	\end{block}
	\begin{block}{Beweis}
		\begin{itemize}
			\item Angenommen nicht: seien $u_1,\ldots,u_\ell$ alle positiven unzerlegbaren Zahlen.
			\item Die Zahl
				\begin{align*}
					z=1+\prod_{i=1}^\ell u_i
				\end{align*}
				wird von keiner der Zahlen $u_1,\ldots,u_\ell$ geteilt.
			\item Aber nach dem Lemma hat $z$ einen unzerlegbaren Teiler $y>1$.
			\item Widerspruch.
		\end{itemize}	
	\end{block}
\end{frame}

\begin{frame}\frametitle{Primzahlen}
	\begin{block}{Definition}
		Eine Zahl $p\in\ZZ\setminus\{-1,0,1\}$ hei\ss t \emph{Primzahl}, falls f\ue r alle $x,y\in\ZZ$ gilt:
		\begin{align*}
			\mbox{Wenn $p\mid x\cdot y$, dann gilt $p\mid x$ oder $p\mid y$.}
		\end{align*}
		Mit $\PP$ wird die Menge aller Primzahlen bezeichnet.
	\end{block}
\end{frame}

\begin{frame}\frametitle{Primzahlen}
	\begin{block}{Beispiel}
		\begin{itemize}
			\item $2$ ist eine Primzahl; denn ist $x\cdot y$ gerade, so auch $x$ oder $y$.
			\item $\pm1$ sind per Definition keine Primzahlen.
		\end{itemize}
	\end{block}
\end{frame}

\begin{frame}\frametitle{Primzahlen}
	\begin{block}{Lemma}
		Jede Primzahl ist unzerlegbar.
	\end{block}
	\begin{block}{Beweis}
		\begin{itemize}
			\item Sei $p>1$ prim.
			\item Angenommen $x>1$ teilt $p$.
			\item Dann existiert $y>0$, so da\ss\ $p=x\cdot y$.
			\item Weil $p$ eine Primzahl ist, folgt $p|x$ oder $p|y$.
			\item Also gilt $p=x$ oder $p=y$.
			\item Weil $x>1$, folgt $p=x$.
		\end{itemize}
	\end{block}
\end{frame}

\begin{frame}\frametitle{Primzahlen}
	\begin{block}{Lemma}
		Jede unzerlegbare Zahl ist eine Primzahl.
	\end{block}
	\begin{block}{Beweis}
		\begin{itemize}
			\item Angenommen nicht; sei $p>1$ das kleinste Gegenbeispiel.
			\item W\ae hle $a,b>1$ mit $p|ab$, $p\nmid a$, $p\nmid b$ und $a+b$ minimal.
			\item Dann gilt $a<p$; denn sonst w\ae re $a-p,b$ ein kleineres Gegenbeispiel.
			\item Analog gilt $b<p$.
			\item Die Zahl $a$ hat einen unzerlegbaren Teiler $1<q\leq a<p$.
			\item Diese Zahl $q$ ist prim, weil $p$ minimal war.
			\item Folglich gibt es $c>0$, so da\ss\ $q|a|ab=cp$.
			\item Weil $q$ prim ist, folgt $q|c$ oder $q|p$.
			\item Weil $p$ unzerlegbar ist, folgt $q|c$.
		\end{itemize}
	\end{block}
\end{frame}

\begin{frame}\frametitle{Primzahlen}
	\begin{block}{Lemma}
		Jede unzerlegbare Zahl ist eine Primzahl.
	\end{block}
	\begin{block}{Beweis (Fortsetzung)}
		\begin{itemize}
			\item Ferner gilt $q<a$; denn sonst h\ae tten wir $q=a$ und aus $q|c$ und $cp=ab$ folgte
\begin{align*}
	b=\frac{c}{a}\cdot p=\frac{c}{q}\cdot p\quad\mbox{also}\quad q|b.
\end{align*}
			\item Weil $q<a$ und $q|c$, folgt $p|cp/q=ab/q$ und $a/q+b<a+b$.
			\item Somit ist $a/q,b$ ein kleineres Gegenbeispiel.
			\item Widerspruch.
		\end{itemize}
	\end{block}

	\bigskip{\itshape Primzahlen und unzerlegbare Zahlen sind also dasselbe!}
\end{frame}

\begin{frame}\frametitle{Die Primfaktorisierung}
	\begin{block}{Satz}
		Zu jeder Primzahl $1<p\in\PP$ gibt es eine Abbildung $w_p:\NN\to\NN_0$, so da\ss\ f\ue r jede Zahl $z\in\NN$ gilt
		\begin{align*}
			z=\prod_{p\in\PP}p^{w_p(z)}.
		\end{align*}
		Die Abbildungen $w_p$ sind eindeutig bestimmt.
	\end{block}
	\begin{block}{Beweis (Existenz)}
		\begin{itemize}
			\item Wir f\ue hren Induktion: f\ue r $z=1$ setze $w_p(z)=0$ f\ue r alle $p$.
			\item Jedes $z>1$ hat einen Primteiler $q>1$.
			\item F\ue r $y=z/q$ wissen wir nach Induktion, da\ss\ $ y=\prod_{p\in\PP}p^{w_p(y)}.  $
			\item Definiere $w_p(z)=w_p(y)$ f\ue r alle $p\neq q$ und $w_q(z)=w_q(y)+1$.
			\item Dann gilt $z=\prod_{p\in\PP}p^{w_p(z)}.$
		\end{itemize}
	\end{block}
\end{frame}

\begin{frame}\frametitle{Die Primfaktorisierung}
	\begin{block}{Beweis (Eindeutigkeit)}
		\begin{itemize}
			\item Angenommen $z>1$ hat zwei verschiedene Zerlegungen
				\begin{align*}
					z=\prod_{p\in\PP}p^{w_p(z)}=\prod_{p\in\PP}p^{v_p(z)}.
				\end{align*}
			\item W\ae hle ein minimales Gegenbeispiel $z$ und betrachte
				\begin{align*}
					P&=\cbc{p\in\PP:w_p(z)>0},&Q&=\cbc{p\in\PP:v_p(z)>0}.
				\end{align*}
			\item Dann gilt $P\cap Q=\emptyset$ und $P\neq\emptyset$, $Q\neq\emptyset$.
			\item F\ue r jedes $p\in P$ gilt $p\mid\prod_{p\in\PP}p^{v_p(z)}$.
			\item Weil $p$ prim ist, gibt es folglich $q\in Q$ mit $p|q$.
			\item Aber dann gilt $q=p$, weil $q$ unzerlegbar ist.
			\item Also gilt $p=q$ und somit $P\cap Q\neq\emptyset$.
			\item Widerspruch.
		\end{itemize}
	\end{block}
\end{frame}

\begin{frame}\frametitle{Die Primfaktorisierung}
	\begin{block}{Satz}
		Zu jeder Primzahl $1<p\in\PP$ gibt es eine Abbildung $w_p:\NN\to\NN_0$, so da\ss\ f\ue r jede Zahl $z\in\NN$ gilt
		\begin{align*}
			z=\prod_{p\in\PP}p^{w_p(z)}.
		\end{align*}
		Die Abbildungen $w_p$ sind eindeutig bestimmt.
	\end{block}
	\begin{block}{}
		\begin{itemize}
			\item Jede nat\ue rliche Zahl hat eine eindeutige Primfaktorisierung.
			\item F\ue r $z<0$ definieren wir $w_p(z)=w_p(-z)$.
		\end{itemize}
	\end{block}
\end{frame}

\begin{frame}\frametitle{Zusammenfassung}
	\begin{itemize}
		\item Unzerlegbare Zahlen und Primzahl stimmen \ue berein.
		\item Jede nat\ue rliche Zahl kann eindeutig in Primfaktoren zerlegt werden.
		\item \emph{Frage:} gibt es einen effizienten Algorithmus, der herausfindet, ob eine gegebene Zahl prim ist?
	\end{itemize}
\end{frame}

\end{document}
