\documentclass{beamer}
\usepackage{amsmath,graphics}
\usepackage{amssymb}

\usetheme{default}
\usepackage{xcolor}

\definecolor{solarizedBase03}{HTML}{002B36}
\definecolor{solarizedBase02}{HTML}{073642}
\definecolor{solarizedBase01}{HTML}{586e75}
\definecolor{solarizedBase00}{HTML}{657b83}
\definecolor{solarizedBase0}{HTML}{839496}
\definecolor{solarizedBase1}{HTML}{93a1a1}
\definecolor{solarizedBase2}{HTML}{EEE8D5}
\definecolor{solarizedBase3}{HTML}{FDF6E3}
\definecolor{solarizedYellow}{HTML}{B58900}
\definecolor{solarizedOrange}{HTML}{CB4B16}
\definecolor{solarizedRed}{HTML}{DC322F}
\definecolor{solarizedMagenta}{HTML}{D33682}
\definecolor{solarizedViolet}{HTML}{6C71C4}
%\definecolor{solarizedBlue}{HTML}{268BD2}
\definecolor{solarizedBlue}{HTML}{134676}
\definecolor{solarizedCyan}{HTML}{2AA198}
\definecolor{solarizedGreen}{HTML}{859900}
\definecolor{myBlue}{HTML}{162DB0}%{261CA4}
\setbeamercolor*{item}{fg=myBlue}
\setbeamercolor{normal text}{fg=solarizedBase03, bg=solarizedBase3}
\setbeamercolor{alerted text}{fg=myBlue}
\setbeamercolor{example text}{fg=myBlue, bg=solarizedBase3}
\setbeamercolor*{frametitle}{fg=solarizedRed}
\setbeamercolor*{title}{fg=solarizedRed}
\setbeamercolor{block title}{fg=myBlue, bg=solarizedBase3}
\setbeameroption{hide notes}
\setbeamertemplate{note page}[plain]
\beamertemplatenavigationsymbolsempty
\usefonttheme{professionalfonts}
\usefonttheme{serif}

\usepackage{fourier}

\def\vec#1{\mathchoice{\mbox{\boldmath$\displaystyle#1$}}
{\mbox{\boldmath$\textstyle#1$}}
{\mbox{\boldmath$\scriptstyle#1$}}
{\mbox{\boldmath$\scriptscriptstyle#1$}}}

\definecolor{OwnGrey}{rgb}{0.560,0.000,0.000} % #999999
\definecolor{OwnBlue}{rgb}{0.121,0.398,0.711} % #1f64b0
\definecolor{red4}{rgb}{0.5,0,0}
\definecolor{blue4}{rgb}{0,0,0.5}
\definecolor{Blue}{rgb}{0,0,0.66}
\definecolor{LightBlue}{rgb}{0.9,0.9,1}
\definecolor{Green}{rgb}{0,0.5,0}
\definecolor{LightGreen}{rgb}{0.9,1,0.9}
\definecolor{Red}{rgb}{0.9,0,0}
\definecolor{LightRed}{rgb}{1,0.9,0.9}
\definecolor{White}{gray}{1}
\definecolor{Black}{gray}{0}
\definecolor{LightGray}{gray}{0.8}
\definecolor{Orange}{rgb}{0.1,0.2,1}
\setbeamerfont{sidebar right}{size=\scriptsize}
\setbeamercolor{sidebar right}{fg=Black}

\renewcommand{\emph}[1]{{\textcolor{solarizedRed}{\itshape #1}}}

\newcommand\cA{\mathcal A}
\newcommand\cB{\mathcal B}
\newcommand\cC{\mathcal C}
\newcommand\cD{\mathcal D}
\newcommand\cE{\mathcal E}
\newcommand\cF{\mathcal F}
\newcommand\cG{\mathcal G}
\newcommand\cH{\mathcal H}
\newcommand\cI{\mathcal I}
\newcommand\cJ{\mathcal J}
\newcommand\cK{\mathcal K}
\newcommand\cL{\mathcal L}
\newcommand\cM{\mathcal M}
\newcommand\cN{\mathcal N}
\newcommand\cO{\mathcal O}
\newcommand\cP{\mathcal P}
\newcommand\cQ{\mathcal Q}
\newcommand\cR{\mathcal R}
\newcommand\cS{\mathcal S}
\newcommand\cT{\mathcal T}
\newcommand\cU{\mathcal U}
\newcommand\cV{\mathcal V}
\newcommand\cW{\mathcal W}
\newcommand\cX{\mathcal X}
\newcommand\cY{\mathcal Y}
\newcommand\cZ{\mathcal Z}

\newcommand\fA{\mathfrak A}
\newcommand\fB{\mathfrak B}
\newcommand\fC{\mathfrak C}
\newcommand\fD{\mathfrak D}
\newcommand\fE{\mathfrak E}
\newcommand\fF{\mathfrak F}
\newcommand\fG{\mathfrak G}
\newcommand\fH{\mathfrak H}
\newcommand\fI{\mathfrak I}
\newcommand\fJ{\mathfrak J}
\newcommand\fK{\mathfrak K}
\newcommand\fL{\mathfrak L}
\newcommand\fM{\mathfrak M}
\newcommand\fN{\mathfrak N}
\newcommand\fO{\mathfrak O}
\newcommand\fP{\mathfrak P}
\newcommand\fQ{\mathfrak Q}
\newcommand\fR{\mathfrak R}
\newcommand\fS{\mathfrak S}
\newcommand\fT{\mathfrak T}
\newcommand\fU{\mathfrak U}
\newcommand\fV{\mathfrak V}
\newcommand\fW{\mathfrak W}
\newcommand\fX{\mathfrak X}
\newcommand\fY{\mathfrak Y}
\newcommand\fZ{\mathfrak Z}

\newcommand\fa{\mathfrak a}
\newcommand\fb{\mathfrak b}
\newcommand\fc{\mathfrak c}
\newcommand\fd{\mathfrak d}
\newcommand\fe{\mathfrak e}
\newcommand\ff{\mathfrak f}
\newcommand\fg{\mathfrak g}
\newcommand\fh{\mathfrak h}
%\newcommand\fi{\mathfrak i}
\newcommand\fj{\mathfrak j}
\newcommand\fk{\mathfrak k}
\newcommand\fl{\mathfrak l}
\newcommand\fm{\mathfrak m}
\newcommand\fn{\mathfrak n}
\newcommand\fo{\mathfrak o}
\newcommand\fp{\mathfrak p}
\newcommand\fq{\mathfrak q}
\newcommand\fr{\mathfrak r}
\newcommand\fs{\mathfrak s}
\newcommand\ft{\mathfrak t}
\newcommand\fu{\mathfrak u}
\newcommand\fv{\mathfrak v}
\newcommand\fw{\mathfrak w}
\newcommand\fx{\mathfrak x}
\newcommand\fy{\mathfrak y}
\newcommand\fz{\mathfrak z}

\newcommand\vA{\vec A}
\newcommand\vB{\vec B}
\newcommand\vC{\vec C}
\newcommand\vD{\vec D}
\newcommand\vE{\vec E}
\newcommand\vF{\vec F}
\newcommand\vG{\vec G}
\newcommand\vH{\vec H}
\newcommand\vI{\vec I}
\newcommand\vJ{\vec J}
\newcommand\vK{\vec K}
\newcommand\vL{\vec L}
\newcommand\vM{\vec M}
\newcommand\vN{\vec N}
\newcommand\vO{\vec O}
\newcommand\vP{\vec P}
\newcommand\vQ{\vec Q}
\newcommand\vR{\vec R}
\newcommand\vS{\vec S}
\newcommand\vT{\vec T}
\newcommand\vU{\vec U}
\newcommand\vV{\vec V}
\newcommand\vW{\vec W}
\newcommand\vX{\vec X}
\newcommand\vY{\vec Y}
\newcommand\vZ{\vec Z}

\newcommand\va{\vec a}
\newcommand\vb{\vec b}
\newcommand\vc{\vec c}
\newcommand\vd{\vec d}
\newcommand\ve{\vec e}
\newcommand\vf{\vec f}
\newcommand\vg{\vec g}
\newcommand\vh{\vec h}
\newcommand\vi{\vec i}
\newcommand\vj{\vec j}
\newcommand\vk{\vec k}
\newcommand\vl{\vec l}
\newcommand\vm{\vec m}
\newcommand\vn{\vec n}
\newcommand\vo{\vec o}
\newcommand\vp{\vec p}
\newcommand\vq{\vec q}
\newcommand\vr{\vec r}
\newcommand\vs{\vec s}
\newcommand\vt{\vec t}
\newcommand\vu{\vec u}
\newcommand\vv{\vec v}
\newcommand\vw{\vec w}
\newcommand\vx{\vec x}
\newcommand\vy{\vec y}
\newcommand\vz{\vec z}

\newcommand\NN{\mathbb N}
\newcommand\ZZ{\mathbb Z}
\newcommand\PP{\mathbb P}
\newcommand\QQ{\mathbb Q}
\newcommand\RR{\mathbb R}
\newcommand\CC{\mathbb C}

\newcommand{\pr}{\mathrm{P}}
\newcommand{\Vol}{\mathrm{vol}}
\newcommand\norm[1]{\left\|{#1}\right\|} 
\newcommand\sign{\mathrm{sign}}
\newcommand{\eps}{\varepsilon}
\newcommand{\abs}[1]{\left|#1\right|}
\newcommand\bc[1]{\left({#1}\right)} 
\newcommand\cbc[1]{\left\{{#1}\right\}} 
\newcommand\bcfr[2]{\bc{\frac{#1}{#2}}} 
\newcommand{\bck}[1]{\left\langle{#1}\right\rangle} 
\newcommand\brk[1]{\left\lbrack{#1}\right\rbrack} 
\newcommand\scal[2]{\bck{{#1},{#2}}} 
\newcommand{\vecone}{\mathbb{1}}
\newcommand{\tensor}{\otimes}
\newcommand{\diag}{\mathrm{diag}}
\newcommand{\ggt}{\mathrm{ggT}}
\newcommand{\kgv}{\mathrm{kgV}}

\newcommand{\Karonski}{Karo\'nski}
\newcommand{\Erdos}{Erd\H{o}s}
\newcommand{\Renyi}{R\'enyi}
\newcommand{\Lovasz}{Lov\'asz}
\newcommand{\Juhasz}{Juh\'asz}
\newcommand{\Bollobas}{Bollob\'as}
\newcommand{\Furedi}{F\"uredi}
\newcommand{\Komlos}{Koml\'os}
\newcommand{\Luczak}{\L uczak}
\newcommand{\Kucera}{Ku\v{c}era}
\newcommand{\Szemeredi}{Szemer\'edi}

\renewcommand{\ae}{\"a}
\renewcommand{\oe}{\"o}
\newcommand{\ue}{\"u}
\newcommand{\Ae}{\"A}
\newcommand{\Oe}{\"O}
\newcommand{\Ue}{\"U}

\title[Linadi]{Die rationalen Zahlen}
\author[Amin Coja-Oghlan]{Amin Coja-Oghlan}
\institute[Frankfurt]{JWGUFFM}
\date{}

\begin{document}

\frame[plain]{\titlepage}

\begin{frame}\frametitle{Die rationalen Zahlen}
	\begin{block}{Die Menge $\QQ$}
		\begin{itemize}
			\item Die Menge $\QQ$ besteht aus allen Br\ue chen der Form 
				\begin{align*}
					\frac{z}{n}
				\end{align*}
				mit $z\in\ZZ$ und $n\in\ZZ\setminus\cbc 0$.
			\item Die Zahl $z$ wird \emph{Z\ae hler} genannt.
			\item Die Zahl $n$ hei\ss t \emph{Nenner}.
		\end{itemize}
	\end{block}
\end{frame}

\begin{frame}\frametitle{Die rationalen Zahlen}
	\begin{block}{K\ue rzen und Erweitern}
		\begin{itemize}
			\item zwei rationale Zahlen $\frac{a}{b}$ und $\frac{c}{d}$ sind genau dann gleich, wenn
				\begin{align*}
				ad=bc.
				\end{align*}
			\item Dieselbe rationale Zahl wird also durch verschiedene Br\ue che dargestellt!
		\end{itemize}
	\end{block}
\end{frame}

\begin{frame}\frametitle{Die rationalen Zahlen}
	\begin{block}{Beispiel}
		\begin{itemize}
			\item F\ue r jede ganze Zahl $z\in\ZZ$ ist $\frac{z}{1}$ eine rationale Zahl.
				Wir identifizieren daher $\ZZ$ mit $\{\frac{z}{1}:z\in\ZZ\}\subset\QQ$.
			\item Die Zahl $3\in\ZZ$ hat verschiedene Darstellungen:
				\begin{align*}
					\frac{3}{1}&=\frac{6}{2}=\frac{9}{3}=\frac{12}{4}=\cdots
				\end{align*}
			\item Die Zahl $-3\in\ZZ$ ebenfalls:
				\begin{align*}
					\frac{-3}{1}&=\frac{6}{-2}=\frac{-9}{3}=\frac{12}{-4}=\cdots
				\end{align*}
		\end{itemize}
	\end{block}
\end{frame}

\begin{frame}\frametitle{Die rationalen Zahlen}
	\begin{block}{Die gek\ue rzte Darstellung}
		\begin{itemize}
			\item Jede rationale Zahl $\frac{z}{n}$ mit eine eindeutige Darstellung $\frac{z}{n}=\frac{s}{t}$, so da\ss\ $t\in\NN$ und so da\ss\ $|s|+t$ minimal ist.
			\item Diese Darstellung erhalten wir, indem wir $\ggt(z,n)$ herausk\ue rzen:
				\begin{align*}
					\frac{z}{n}=\frac{\pm|z|/\ggt(z,n)}{|n|/\ggt(z,n)}.
				\end{align*}
			\item Weil $\ggt(z,n)$ mit dem Euklidischen Algorithmus effizient berechnet werden kann, k\oe nnen wir diese gek\ue rzte Darstellung ebenfalls effizient berechnen.
			\item Die Darstellung von $0=\frac{0}{1}\in\QQ$ ist bis auf das Vorzeichen eindeutig.
			\item Werden rationale Zahlen im Computer implementiert, ist nach jeder Operation zu k\ue rzen.
		\end{itemize}
	\end{block}
\end{frame}

\begin{frame}\frametitle{Bruchrechnung}
	\begin{block}{Grundrechenarten}
		\begin{itemize}
			\item Addition
			\item Subtraktion
			\item Multiplikation
			\item Division
		\end{itemize}
	\end{block}
\end{frame}

\begin{frame}\frametitle{Bruchrechnung}
	\begin{block}{Die Addition}
		\begin{itemize}
			\item  $\displaystyle\frac{z}{n}+\frac{y}{m}=\frac{m\cdot z+n\cdot y}{m\cdot n}	$
		\end{itemize}
	\end{block}
	\begin{block}{Beispiel}
	\begin{align*}
		\frac{2}{3}+\frac{8}{5}&=\frac{5\cdot 2+3\cdot 8}{15}=\frac{34}{15}
	\end{align*}
	\end{block}
\end{frame}

\begin{frame}\frametitle{Bruchrechnung}
	\begin{block}{Die Subtraktion}
		\begin{itemize}
			\item  $\displaystyle\frac{z}{n}-\frac{y}{m}=\frac{m\cdot z-n\cdot y}{m\cdot n}	$
		\end{itemize}
	\end{block}
	\begin{block}{Beispiel}
		\begin{align*}
			\frac{2}{3}-\frac{8}{5}&=\frac{5\cdot 2-3\cdot 8}{15}=\frac{-14}{15}
		\end{align*}
	\end{block}
\end{frame}

\begin{frame}\frametitle{Bruchrechnung}
	\begin{block}{Die Multiplikation}
		\begin{itemize}
			\item  $\displaystyle\frac{z}{n}\cdot\frac{y}{m}=\frac{z\cdot y}{m\cdot n}$
		\end{itemize}
	\end{block}
\begin{block}{Beispiel}
		\begin{align*}
			\frac{2}{3}\cdot\frac{8}{5}&=\frac{2\cdot 8}{3\cdot 5}=\frac{16}{15}
		\end{align*}
	\end{block}
\end{frame}

\begin{frame}\frametitle{Bruchrechnung}
	\begin{block}{Die Division}
		\begin{itemize}
			\item  $\displaystyle\frac{z}{n}/\frac{y}{m}=\frac{z\cdot m}{n\cdot y}$\hfill($y\neq0$)
		\end{itemize}
	\end{block}
\begin{block}{Beispiel}
		\begin{align*}
			\frac{2}{3}/\frac{8}{5}&=\frac{2\cdot 5}{3\cdot 8}=\frac{10}{24}=\frac{5}{12}
		\end{align*}
	\end{block}
\end{frame}

\begin{frame}\frametitle{Rechenregeln}
	\begin{block}{Addition}
		\begin{description}
			\item[Assoziativgesetz] F\ue r alle $a,b,c\in\QQ$ gilt $(a+b)+c=a+(b+c)$.
			\item[Kommutativgesetz] F\ue r alle $a,b\in\QQ$ gilt $a+b=b+a$.
			\item[Nullelement] F\ue r alle $a\in\QQ$ gilt $0+a=a$.
			\item[Inverses Element] Zu jeder Zahl $a\in\QQ$ existiert $-a\in\QQ$ mit $$-a+a=0,$$ n\ae mlich
				\begin{align*}
				-\frac{z}{n}=\frac{-z}{n}.
				\end{align*}
		\end{description}
	\end{block}
\end{frame}

\begin{frame}\frametitle{Rechenregeln}
	\begin{block}{Multiplikation}
		\begin{description}
			\item[Assoziativgesetz] F\ue r alle $a,b,c\in\QQ$ gilt $(a\cdot b)\cdot c=a\cdot (b\cdot c)$.
			\item[Kommutativgesetz] F\ue r alle $a,b\in\QQ$ gilt $a\cdot b=b\cdot a$.
			\item[Einselement] F\ue r alle $a\in\QQ$ gilt $1\cdot a=a$.
			\item[Inverses Element] Zu jeder Zahl $a\in\QQ\setminus\cbc 0$ existiert $a^{-1}\in\QQ$ mit $$a^{-1}\cdot a=1,$$ n\ae mlich
				\begin{align*}
					\bcfr{z}{n}^{-1}=\frac{n}{z}.
				\end{align*}
		\end{description}
	\end{block}
\end{frame}

\begin{frame}\frametitle{Rechenregeln}
	\begin{block}{Distributivgesetz}
		F\ue r alle $a,b,c\in\QQ$ gilt	
		\begin{align*}
			a\cdot(b+c)=a\cdot b+a\cdot c.
		\end{align*}
	\end{block}
\end{frame}

\begin{frame}\frametitle{Anordnung}
	\begin{block}{}
	\begin{itemize}
	\item F\ue r rationale Zahlen $\frac{z}{n},\frac{y}{m}$ mit $m,n>0$ definieren wir
		\begin{align*}
			\frac{z}{n}<\frac{y}{m}\quad\mbox{ genau dann, wenn }\quad z\cdot m<y\cdot n.
		\end{align*}
	\item Damit sind implizit auch die anderen Vergleichsrelationen
		\begin{align*}
		\leq\quad >\quad \geq
		\end{align*}
		definiert.
	\end{itemize}
	\end{block}
	\begin{block}{Beispiel}
	\begin{align*}
		\frac{-7}{8}<\frac{-2}{3}\qquad\mbox{weil}\qquad -7\cdot 3=-21<-16=-2\cdot 8.
	\end{align*}	
	\end{block}
\end{frame}

\begin{frame}\frametitle{Zusammenfassung}
\begin{itemize}
\item Wir haben die Grundrechenarten f\ue r rationale Zahlen kennengelernt.
\item Diese k\oe nnen auf die Grundrechenarten in $\ZZ$ und auf den Euklidischen Algorithmus zur\ue ckgef\ue hrt werden.
\item Daher k\oe nnen wir mit rationalen Zahlen effizient operieren.
\item Das Computeralgebrasystem Sage unterst\ue tzt Berechnungen mit rationalen Zahlen.
\end{itemize}
\end{frame}

\end{document}
