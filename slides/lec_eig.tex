\documentclass{beamer}
\usepackage{amsmath,graphics}
\usepackage{amssymb}

\usetheme{default}
\usepackage{xcolor}

\definecolor{solarizedBase03}{HTML}{002B36}
\definecolor{solarizedBase02}{HTML}{073642}
\definecolor{solarizedBase01}{HTML}{586e75}
\definecolor{solarizedBase00}{HTML}{657b83}
\definecolor{solarizedBase0}{HTML}{839496}
\definecolor{solarizedBase1}{HTML}{93a1a1}
\definecolor{solarizedBase2}{HTML}{EEE8D5}
\definecolor{solarizedBase3}{HTML}{FDF6E3}
\definecolor{solarizedYellow}{HTML}{B58900}
\definecolor{solarizedOrange}{HTML}{CB4B16}
\definecolor{solarizedRed}{HTML}{DC322F}
\definecolor{solarizedMagenta}{HTML}{D33682}
\definecolor{solarizedViolet}{HTML}{6C71C4}
%\definecolor{solarizedBlue}{HTML}{268BD2}
\definecolor{solarizedBlue}{HTML}{134676}
\definecolor{solarizedCyan}{HTML}{2AA198}
\definecolor{solarizedGreen}{HTML}{859900}
\definecolor{myBlue}{HTML}{162DB0}%{261CA4}
\setbeamercolor*{item}{fg=myBlue}
\setbeamercolor{normal text}{fg=solarizedBase03, bg=solarizedBase3}
\setbeamercolor{alerted text}{fg=myBlue}
\setbeamercolor{example text}{fg=myBlue, bg=solarizedBase3}
\setbeamercolor*{frametitle}{fg=solarizedRed}
\setbeamercolor*{title}{fg=solarizedRed}
\setbeamercolor{block title}{fg=myBlue, bg=solarizedBase3}
\setbeameroption{hide notes}
\setbeamertemplate{note page}[plain]
\beamertemplatenavigationsymbolsempty
\usefonttheme{professionalfonts}
\usefonttheme{serif}

\usepackage{fourier}

\def\vec#1{\mathchoice{\mbox{\boldmath$\displaystyle#1$}}
{\mbox{\boldmath$\textstyle#1$}}
{\mbox{\boldmath$\scriptstyle#1$}}
{\mbox{\boldmath$\scriptscriptstyle#1$}}}
\definecolor{OwnGrey}{rgb}{0.560,0.000,0.000} % #999999
\definecolor{OwnBlue}{rgb}{0.121,0.398,0.711} % #1f64b0
\definecolor{red4}{rgb}{0.5,0,0}
\definecolor{blue4}{rgb}{0,0,0.5}
\definecolor{Blue}{rgb}{0,0,0.66}
\definecolor{LightBlue}{rgb}{0.9,0.9,1}
\definecolor{Green}{rgb}{0,0.5,0}
\definecolor{LightGreen}{rgb}{0.9,1,0.9}
\definecolor{Red}{rgb}{0.9,0,0}
\definecolor{LightRed}{rgb}{1,0.9,0.9}
\definecolor{White}{gray}{1}
\definecolor{Black}{gray}{0}
\definecolor{LightGray}{gray}{0.8}
\definecolor{Orange}{rgb}{0.1,0.2,1}
\setbeamerfont{sidebar right}{size=\scriptsize}
\setbeamercolor{sidebar right}{fg=Black}

\renewcommand{\emph}[1]{{\textcolor{solarizedRed}{\itshape #1}}}

\newcommand\cA{\mathcal A}
\newcommand\cB{\mathcal B}
\newcommand\cC{\mathcal C}
\newcommand\cD{\mathcal D}
\newcommand\cE{\mathcal E}
\newcommand\cF{\mathcal F}
\newcommand\cG{\mathcal G}
\newcommand\cH{\mathcal H}
\newcommand\cI{\mathcal I}
\newcommand\cJ{\mathcal J}
\newcommand\cK{\mathcal K}
\newcommand\cL{\mathcal L}
\newcommand\cM{\mathcal M}
\newcommand\cN{\mathcal N}
\newcommand\cO{\mathcal O}
\newcommand\cP{\mathcal P}
\newcommand\cQ{\mathcal Q}
\newcommand\cR{\mathcal R}
\newcommand\cS{\mathcal S}
\newcommand\cT{\mathcal T}
\newcommand\cU{\mathcal U}
\newcommand\cV{\mathcal V}
\newcommand\cW{\mathcal W}
\newcommand\cX{\mathcal X}
\newcommand\cY{\mathcal Y}
\newcommand\cZ{\mathcal Z}

\newcommand\fA{\mathfrak A}
\newcommand\fB{\mathfrak B}
\newcommand\fC{\mathfrak C}
\newcommand\fD{\mathfrak D}
\newcommand\fE{\mathfrak E}
\newcommand\fF{\mathfrak F}
\newcommand\fG{\mathfrak G}
\newcommand\fH{\mathfrak H}
\newcommand\fI{\mathfrak I}
\newcommand\fJ{\mathfrak J}
\newcommand\fK{\mathfrak K}
\newcommand\fL{\mathfrak L}
\newcommand\fM{\mathfrak M}
\newcommand\fN{\mathfrak N}
\newcommand\fO{\mathfrak O}
\newcommand\fP{\mathfrak P}
\newcommand\fQ{\mathfrak Q}
\newcommand\fR{\mathfrak R}
\newcommand\fS{\mathfrak S}
\newcommand\fT{\mathfrak T}
\newcommand\fU{\mathfrak U}
\newcommand\fV{\mathfrak V}
\newcommand\fW{\mathfrak W}
\newcommand\fX{\mathfrak X}
\newcommand\fY{\mathfrak Y}
\newcommand\fZ{\mathfrak Z}

\newcommand\fa{\mathfrak a}
\newcommand\fb{\mathfrak b}
\newcommand\fc{\mathfrak c}
\newcommand\fd{\mathfrak d}
\newcommand\fe{\mathfrak e}
\newcommand\ff{\mathfrak f}
\newcommand\fg{\mathfrak g}
\newcommand\fh{\mathfrak h}
%\newcommand\fi{\mathfrak i}
\newcommand\fj{\mathfrak j}
\newcommand\fk{\mathfrak k}
\newcommand\fl{\mathfrak l}
\newcommand\fm{\mathfrak m}
\newcommand\fn{\mathfrak n}
\newcommand\fo{\mathfrak o}
\newcommand\fp{\mathfrak p}
\newcommand\fq{\mathfrak q}
\newcommand\fr{\mathfrak r}
\newcommand\fs{\mathfrak s}
\newcommand\ft{\mathfrak t}
\newcommand\fu{\mathfrak u}
\newcommand\fv{\mathfrak v}
\newcommand\fw{\mathfrak w}
\newcommand\fx{\mathfrak x}
\newcommand\fy{\mathfrak y}
\newcommand\fz{\mathfrak z}

\newcommand\vA{\vec A}
\newcommand\vB{\vec B}
\newcommand\vC{\vec C}
\newcommand\vD{\vec D}
\newcommand\vE{\vec E}
\newcommand\vF{\vec F}
\newcommand\vG{\vec G}
\newcommand\vH{\vec H}
\newcommand\vI{\vec I}
\newcommand\vJ{\vec J}
\newcommand\vK{\vec K}
\newcommand\vL{\vec L}
\newcommand\vM{\vec M}
\newcommand\vN{\vec N}
\newcommand\vO{\vec O}
\newcommand\vP{\vec P}
\newcommand\vQ{\vec Q}
\newcommand\vR{\vec R}
\newcommand\vS{\vec S}
\newcommand\vT{\vec T}
\newcommand\vU{\vec U}
\newcommand\vV{\vec V}
\newcommand\vW{\vec W}
\newcommand\vX{\vec X}
\newcommand\vY{\vec Y}
\newcommand\vZ{\vec Z}

\newcommand\va{\vec a}
\newcommand\vb{\vec b}
\newcommand\vc{\vec c}
\newcommand\vd{\vec d}
\newcommand\ve{\vec e}
\newcommand\vf{\vec f}
\newcommand\vg{\vec g}
\newcommand\vh{\vec h}
\newcommand\vi{\vec i}
\newcommand\vj{\vec j}
\newcommand\vk{\vec k}
\newcommand\vl{\vec l}
\newcommand\vm{\vec m}
\newcommand\vn{\vec n}
\newcommand\vo{\vec o}
\newcommand\vp{\vec p}
\newcommand\vq{\vec q}
\newcommand\vr{\vec r}
\newcommand\vs{\vec s}
\newcommand\vt{\vec t}
\newcommand\vu{\vec u}
\newcommand\vv{\vec v}
\newcommand\vw{\vec w}
\newcommand\vx{\vec x}
\newcommand\vy{\vec y}
\newcommand\vz{\vec z}

\renewcommand\AA{\mathbb A}
\newcommand\NN{\mathbb N}
\newcommand\ZZ{\mathbb Z}
\newcommand\PP{\mathbb P}
\newcommand\QQ{\mathbb Q}
\newcommand\RR{\mathbb R}
\renewcommand\SS{\mathbb S}
\newcommand\CC{\mathbb C}

\newcommand{\ord}{\mathrm{ord}}
\newcommand{\id}{\mathrm{id}}
\newcommand{\pr}{\mathrm{P}}
\newcommand{\Vol}{\mathrm{vol}}
\newcommand\norm[1]{\left\|{#1}\right\|} 
\newcommand\sign{\mathrm{sign}}
\newcommand{\eps}{\varepsilon}
\newcommand{\abs}[1]{\left|#1\right|}
\newcommand\bc[1]{\left({#1}\right)} 
\newcommand\cbc[1]{\left\{{#1}\right\}} 
\newcommand\bcfr[2]{\bc{\frac{#1}{#2}}} 
\newcommand{\bck}[1]{\left\langle{#1}\right\rangle} 
\newcommand\brk[1]{\left\lbrack{#1}\right\rbrack} 
\newcommand\scal[2]{\bck{{#1},{#2}}} 
\newcommand{\vecone}{\mathbb{1}}
\newcommand{\tensor}{\otimes}
\newcommand{\diag}{\mathrm{diag}}
\newcommand{\ggt}{\mathrm{ggT}}
\newcommand{\kgv}{\mathrm{kgV}}
\newcommand{\trans}{\top}

\newcommand{\Karonski}{Karo\'nski}
\newcommand{\Erdos}{Erd\H{o}s}
\newcommand{\Renyi}{R\'enyi}
\newcommand{\Lovasz}{Lov\'asz}
\newcommand{\Juhasz}{Juh\'asz}
\newcommand{\Bollobas}{Bollob\'as}
\newcommand{\Furedi}{F\"uredi}
\newcommand{\Komlos}{Koml\'os}
\newcommand{\Luczak}{\L uczak}
\newcommand{\Kucera}{Ku\v{c}era}
\newcommand{\Szemeredi}{Szemer\'edi}

\renewcommand{\ae}{\"a}
\renewcommand{\oe}{\"o}
\newcommand{\ue}{\"u}
\newcommand{\Ae}{\"A}
\newcommand{\Oe}{\"O}
\newcommand{\Ue}{\"U}

\newcommand{\im}{\mathrm{im}}
\newcommand{\rrk}{\mathrm{zrg}}
\newcommand{\crk}{\mathrm{srg}}
\newcommand{\rk}{\mathrm{rg}}
\newcommand{\GL}{\mathrm{GL}}
\newcommand{\SL}{\mathrm{SL}}
\newcommand{\SO}{\mathrm{SO}}
\newcommand{\nul}{\mathrm{nul}}
\newcommand{\eig}{\mathrm{eig}}

\newcommand{\mytitle}{Eigenwerte}

\title[Linadi]{\mytitle}
\author[Amin Coja-Oghlan]{Amin Coja-Oghlan}
\institute[Frankfurt]{JWGUFFM}
\date{}

\begin{document}

\frame[plain]{\titlepage}

\begin{frame}\frametitle{\mytitle}
	\begin{block}{Definition}
		\begin{itemize}
			\item Sei $A$ eine $n\times n$-Matrix.
			\item Eine Zahl $\lambda\in\RR$ hei\ss t \emph{Eigenwert} von $A$, falls
				\begin{align*}
					\nul(A-\lambda\id_n)>0;
				\end{align*}
			d.h.\ es gibt einen Vektor $u\in\RR^n\setminus\cbc 0$, so da\ss\
				\begin{align*}
				Au=\lambda u.
				\end{align*}
			\item Ein solcher Vektor $u$ hei\ss t \emph{Eigenvektor} von $A$ zum Eigenwert $\lambda$.
		\end{itemize}
	\end{block}
\end{frame}

\begin{frame}\frametitle{\mytitle}
	\begin{block}{Proposition}
		Sei $A$ eine $n\times n$-Matrix und sei $\lambda\in\RR$ ein Eigenwert von $A$.
		Die Menge
		\begin{align*}
			\eig(A,\lambda)&=\cbc{u\in\RR^n:Au=\lambda u}
		\end{align*}
		ist ein Untervektorraum von $\RR^n$.
	\end{block}
\end{frame}

\begin{frame}\frametitle{\mytitle}
	\begin{block}{Beispiel}
		\begin{itemize}
			\item Die Matrix
				\begin{align*}
					A&=\begin{pmatrix} 1&0\\0&2 \end{pmatrix}
				\end{align*}
				hat die Eigenwerte $1$ und $2$.
			\item Entsprechende Eigenvektoren sind
				\begin{align*}
					\begin{pmatrix} 1\\0 \end{pmatrix}\quad\mbox{und}\quad
				\begin{pmatrix} 0\\1 \end{pmatrix}
				\end{align*}
			\item Allgemein sind die Eigenwerte einer Diagonalmatrix
				\begin{align*}
					\diag(c_1,\ldots,c_n)
				\end{align*}
				immer genau die Zahlen $c_1,\ldots,c_n$.
		\end{itemize}
	\end{block}
\end{frame}

\begin{frame}\frametitle{\mytitle}
	\begin{block}{Definition}
		Sei $A$ eine $n\times n$-Matrix.
		Die Funktion
		\begin{align*}
			\chi_A:\RR\to\RR&&x\mapsto\det(A-x\cdot\id_n)
		\end{align*}
		hei\ss t das \emph{charakteristische Polynom} von $A$.
	\end{block}
\begin{block}{Proposition}
		Sei $A$ eine $n\times n$-Matrix.
		Eine Zahl $\lambda\in\RR$ ist ein Eigenwert von $A$ genau dann, wenn $\chi_A(\lambda)=0$.
	\end{block}
\end{frame}

\begin{frame}\frametitle{\mytitle}
	\begin{block}{Rechenschema f\ue r Eigenwerte und Eigenvektoren}
	\begin{itemize}
		\item Bestimme das charakteristische Polynom $\chi_A(x)$.
		\item Finde die Nullstellen $\lambda_1,\ldots,\lambda_\ell$ von $\chi_A(x)$.
		\item Bestimme die Eigenr\ae ume
			\begin{align*}
				\eig(A,\lambda_i)&=\ker(A-\lambda_i\cdot\id_n)&&\mbox{f\ue r }i=1,\ldots,\ell.
			\end{align*}
	\end{itemize}
	\end{block}
\end{frame}

\begin{frame}\frametitle{\mytitle}
	\begin{block}{Beispiel}
	\begin{itemize}
	\item Betrachte die Matrix
		\begin{align*}
			A&=\begin{pmatrix}1&2\\2&-1\end{pmatrix}
		\end{align*}
	\item Das charakteristische Polynom lautet
		\begin{align*}
			\chi_A(x)&=\det\bc{ \begin{pmatrix}1&2\\2&-1\end{pmatrix}-x\cdot\begin{pmatrix}1&0\\0&1\end{pmatrix} }
									 =\det\begin{pmatrix} 1-x&2\\2&-1-x \end{pmatrix}\\
															 &=(1-x)(-1-x)-4=x^2-5
		\end{align*}
	\item Die Nullstellen lauten also $\lambda_1=-\sqrt 5$ und $\lambda_2=\sqrt 5$.
	\end{itemize}
	\end{block}
\end{frame}

\begin{frame}\frametitle{\mytitle}
	\begin{block}{Beispiel}
	\begin{itemize}
		\item Wir bestimmen nun
			\begin{align*}
				\eig(A,\pm\sqrt 5)&=\ker\begin{pmatrix} 1\mp\sqrt 5&2\\2&-1\mp\sqrt 5 \end{pmatrix}
			\end{align*}
			mit dem Gau\ss schen Eliminationsverfahren.
		\item Wir brigen dazu die Matrizen in Zeilenstufenform; wir beginnen mit
			\begin{align*}
				\begin{pmatrix} 1\mp\sqrt 5&2\\2&-1\mp\sqrt 5 \end{pmatrix}
			\end{align*}
		\item Zun\ae chst vertauschen wir die Zeilen:
\begin{align*}
				\begin{pmatrix} 2&-1\mp\sqrt 5\\1\mp\sqrt 5&2 \end{pmatrix}
			\end{align*}
	\end{itemize}
	\end{block}
\end{frame}

\begin{frame}\frametitle{\mytitle}
	\begin{block}{Beispiel}
	\begin{itemize}
		\item Nun subtrahieren wir das $(1\mp\sqrt 5)/2$-Fache der ersten Zeile von der zweiten Zeile:
\begin{align*}
				\begin{pmatrix} 2&-1\mp\sqrt 5\\0&0 \end{pmatrix}
			\end{align*}
		\item Die Matrix ist nun in Zeilenstufenform.
		\item Wir lesen der Kern ab:
			\begin{align*}
				\ker\begin{pmatrix} 2&-1\mp\sqrt 5\\0&0 \end{pmatrix}&=
				\cbc{\begin{pmatrix}
						(1\pm\sqrt 5)u_2/2\\u_2
				\end{pmatrix}:u_2\in\RR}.
			\end{align*}
		\item Konkrete Eigenvektoren zu den Eigenwerten $\pm\sqrt 5$ sind also
			\begin{align*}
				\begin{pmatrix} 1+\sqrt 5\\2 \end{pmatrix}&&
			\begin{pmatrix} 1-\sqrt 5\\2 \end{pmatrix}
			\end{align*}
	\end{itemize}
	\end{block}
\end{frame}

\begin{frame}\frametitle{\mytitle}
	\begin{block}{Beispiel}
	\begin{itemize}
		\item Zur Probe multiplizieren wir diese beiden Vektoren mit $A$:
			\begin{align*}
				\begin{pmatrix}1&2\\2&-1\end{pmatrix}\begin{pmatrix} 1+\sqrt 5\\2 \end{pmatrix}&= \begin{pmatrix} 5+\sqrt 5\\2\sqrt 5 \end{pmatrix}=\sqrt 5\begin{pmatrix} 1+\sqrt 5\\2 \end{pmatrix}\\
				\begin{pmatrix}1&2\\2&-1\end{pmatrix}\begin{pmatrix} 1-\sqrt 5\\2 \end{pmatrix}&= \begin{pmatrix} 5-\sqrt 5\\-2\sqrt 5 \end{pmatrix}=-\sqrt 5\begin{pmatrix} 1-\sqrt 5\\2 \end{pmatrix}
			\end{align*}
		\item Wir haben also die Eigenwerte und Eigenvektoren der Matrix $A$ bestimmt.
	\end{itemize}
	\end{block}
\end{frame}

\begin{frame}\frametitle{\mytitle}
	\begin{block}{Definition}
		Eine $n\times n$-Matrix $A$ hei\ss t \emph{symmetrisch}, wenn
		\begin{align*}
		A^\trans=A.
		\end{align*}
	\end{block}
\end{frame}

\begin{frame}\frametitle{\mytitle}
	\begin{block}{Satz}
		Angenommen $A$ ist eine symmetrische $n\times n$-Matrix.
		Dann gibt es eine orthogonale $n\times n$-Matrix $U$ und reelle Zahlen
			\begin{align*}
			\lambda_1\leq\lambda_2\leq\cdots\leq\lambda_n
			\end{align*}
		so da\ss\
			\begin{align*}
				A&=U\cdot \diag(\lambda_1,\ldots,\lambda_n)\cdot U^{-1}.
			\end{align*}
		Die Zahlen $\lambda_1,\ldots,\lambda_n$ sind genau die Eigenwerte von $A$.
	\end{block}
\end{frame}

\begin{frame}\frametitle{\mytitle}
	\begin{block}{Anmerkung}
	\begin{itemize}
	\item Man spricht von \emph{Diagonalisierung} der Matrix $A$.
	\item Die Spalten der Matrix $U$ bilden eine Orthonormalbasis des $\RR^n$, die aus Eigenvektoren der Matrix $A$ besteht.
	\item Wichtig ist die Annahme, da\ss\ $A$ symmetrisch ist, d.h.\
		\begin{align*}
		A^\trans=A.
		\end{align*}
	\item Matrizen, die nicht symmetrisch sind, besitzen nicht notwendigerweise eine Orthonormalbasis aus Eigenvektoren.
	\end{itemize}
	\end{block}
\end{frame}

\begin{frame}\frametitle{\mytitle}
	\begin{block}{Rechenschema zum Diagonalisieren}
	\begin{itemize}
		\item Sei $A$ eine symmetrische $n\times n$-Matrix.
		\item Wir bestimmen das charakteristische Polynom $\chi_A(x)$ und seine Nullstellen.
		\item F\ue r jede Nullstelle $\lambda$ von $\chi_A(x)$ berechnen wir $\eig(A,\lambda)$.
		\item Wir verwenden das Gram-Schmidt-Verfahren, um f\ue r jedes $\lambda$ eine Orthonormalbasis von $\eig(A,\lambda)$ zu berechnen.
		\item Diese Basisvektoren, nach Eigenwerten aufsteigend geordnet, liefern die Spalten der Matrix $U$.
	\end{itemize}
	\end{block}
\end{frame}

\begin{frame}\frametitle{\mytitle}
	\begin{block}{Beispiel}
	\begin{itemize}
		\item Betrachte die symmetrische Matrix
			\begin{align*}
				A&=\begin{pmatrix}
					1&-1&-1&-1\\-1&1&-1&-1\\-1&-1&1&-1\\-1&-1&-1&1
				\end{pmatrix}
			\end{align*}
		\item Wir berechnen zun\ae chst
			\begin{align*}
				\chi_A(x)&=\det(A-x\cdot\id_4)=\det\begin{pmatrix}
					1-x&-1&-1&-1\\-1&1-x&-1&-1\\-1&-1&1-x&-1\\-1&-1&-1&1-x
				\end{pmatrix}
			\end{align*}
		\item Wir bringen die Matrix in Zeilenstufen, wobei wir mit $x$ als einer \alert{Unbestimmten} rechnen.
	\end{itemize}
	\end{block}
\end{frame}

\begin{frame}\frametitle{\mytitle}
	\begin{block}{Beispiel}
	\begin{itemize}
		\item Addiere die letzte Zeile zur ersten und subtrahiere sie von der zweiten und dritten Zeile:
			\begin{align*}
				\chi_A(x)&=\det\begin{pmatrix}
					-x&-2&-2&-x\\0&2-x&0&x-2\\0&0&2-x&x-2\\-1&-1&-1&1-x
				\end{pmatrix}
			\end{align*}
		\item Nun skalieren wir die Matrix mit $-1$:
\begin{align*}
				\chi_A(x)&=\det\begin{pmatrix}
					x&2&2&x\\0&x-2&0&2-x\\0&0&x-2&2-x\\1&1&1&x-1
				\end{pmatrix}
			\end{align*}
	\end{itemize}
	\end{block}
\end{frame}

\begin{frame}\frametitle{\mytitle}
	\begin{block}{Beispiel}
	\begin{itemize}
		\item Subtrahiere nun $x$-fache der letzten Zeile von der ersten Zeile:
\begin{align*}
				\chi_A(x)&=\det\begin{pmatrix}
					0&2-x&2-x&2x-x^2\\0&x-2&0&2-x\\0&0&x-2&2-x\\1&1&1&x-1
				\end{pmatrix}
			\end{align*}
		\item Addiere die zweite und die dritte Zeile zur ersten Zeile:
\begin{align*}
				\chi_A(x)&=\det\begin{pmatrix}
					0&0&0&4-x^2\\0&x-2&0&2-x\\0&0&x-2&2-x\\1&1&1&x-1
				\end{pmatrix}
			\end{align*}
	\end{itemize}
	\end{block}
\end{frame}

\begin{frame}\frametitle{\mytitle}
	\begin{block}{Beispiel}
	\begin{itemize}
		\item Vertausche die erste und die letzte Zeile:
\begin{align*}
				\chi_A(x)&=-\det\begin{pmatrix}
					1&1&1&x-1 \\0&x-2&0&2-x\\0&0&x-2&2-x\\0&0&0&4-x^2				\end{pmatrix}
			\end{align*}
		\item Die Matrix ist jetzt in Zeilenstufenform und wir berechnen
			\begin{align*}
				\chi_A(x)=(x-2)^2(x^2-4)=(x-2)^3(x+2).
			\end{align*}
		\item Die Nullstellen von $\chi_A(x)$ sind also $\pm2$.
	\end{itemize}
	\end{block}
\end{frame}

\begin{frame}\frametitle{\mytitle}
	\begin{block}{Beispiel}
	\begin{itemize}
		\item Wir bestimmen nun die Eigenr\ae ume $\ker(A+2\id)$, $\ker(A-2\id)$.
		\item Dazu verwenden wir das Gau\ss verfahren:
			\begin{align*}
				A+2\cdot\id&= \begin{pmatrix} 3&-1&-1&-1\\-1&3&-1&-1\\-1&-1&3&-1\\-1&-1&-1&3
				\end{pmatrix}
			\end{align*}
		\item Addiere die letzte Zeile dreimal zur ersten und subtrahiere sie von den anderen beiden:
\begin{align*}
				 \begin{pmatrix} 0&-4&-4&8\\0&4&0&-4\\0&0&4&-4\\-1&-1&-1&3
				\end{pmatrix}
			\end{align*}
	\end{itemize}
	\end{block}
\end{frame}

\begin{frame}\frametitle{\mytitle}
	\begin{block}{Beispiel}
	\begin{itemize}
		\item Addiere die zweite und die dritte Zeile zur ersten Zeile:
\begin{align*}
				 \begin{pmatrix} 0&0&0&0\\0&4&0&-4\\0&0&4&-4\\-1&-1&-1&3
				\end{pmatrix}
			\end{align*}
		\item Jetzt vertausche die erste und die letzte Zeile:
\begin{align*}
				 \begin{pmatrix}-1&-1&-1&3 \\0&4&0&-4\\0&0&4&-4\\0&0&0&0				\end{pmatrix}
			\end{align*}
	\end{itemize}
	\end{block}
\end{frame}

\begin{frame}\frametitle{\mytitle}
	\begin{block}{Beispiel}
	\begin{itemize}
		\item Der Kern der Matrix ist also
			\begin{align*}
				\eig(A,-2)&=\ker(A+2\id)=\cbc{\begin{pmatrix} u_4\\u_4\\u_4\\u_4 \end{pmatrix}:u_4\in\RR}.
			\end{align*}
	\end{itemize}
	\end{block}
\end{frame}

\begin{frame}\frametitle{\mytitle}
	\begin{block}{Beispiel}
	\begin{itemize}
		\item Weil der Kern von einem einzigen Vektor aufgespannt wird, ist seine Dimension 1.
		\item Daher bildet der einzelne Vektor
			\begin{align*}
			\begin{pmatrix}
			1/2\\1/2\\1/2\\1/2
			\end{pmatrix}
			\end{align*}
			eine Orthonormalbasis von $\eig(A,-2)$.
	\end{itemize}
	\end{block}
\end{frame}

\begin{frame}\frametitle{\mytitle}
	\begin{block}{Beispiel}
	\begin{itemize}
		\item Zur Berechnung von $\ker(A-2\id)$ verwenden wir ebenfalls das Gau\ss verfahren:
			\begin{align*}
				A-2\cdot\id&= \begin{pmatrix} -1&-1&-1&-1\\-1&-1&-1&-1\\-1&-1&-1&-1\\-1&-1&-1&-1
				\end{pmatrix}
			\end{align*}
		\item Subtrahiere die erste Zeile von den anderen drei Zeilen:
\begin{align*}
	\begin{pmatrix}-1&-1&-1&-1\\0&0&0&0\\0&0&0&0\\0&0&0&0\end{pmatrix}
			\end{align*}
	\end{itemize}
	\end{block}
\end{frame}

\begin{frame}\frametitle{\mytitle}
	\begin{block}{Beispiel}
	\begin{itemize}
		\item Der Kern der Matrix lautet also
\begin{align*}
	\ker(A-2\cdot\id)&=\cbc{\begin{pmatrix}-u_2-u_3-u_4\\u_2\\u_3\\u_4\end{pmatrix}:u_2,u_3,u_4\in\RR}.
			\end{align*}
		\item Die Dimension des Kerns ist 3.
		\item Wir erhalten drei linear unabh\ae ngige Vektoren im Kern, indem wir f\ue r $u_2,u_3,u_4$ einsetzen
			\begin{align*}
				\begin{pmatrix} u_2\\u_3\\u_4 \end{pmatrix}&=\begin{pmatrix}1\\0\\0\end{pmatrix}&
				\begin{pmatrix} u_2\\u_3\\u_4 \end{pmatrix}&=\begin{pmatrix}0\\1\\0\end{pmatrix}&
				\begin{pmatrix} u_2\\u_3\\u_4 \end{pmatrix}&=\begin{pmatrix}0\\0\\1\end{pmatrix}
			\end{align*}
	\end{itemize}
	\end{block}
\end{frame}

\begin{frame}\frametitle{\mytitle}
	\begin{block}{Beispiel}
	\begin{itemize}
		\item Dies ergibt die Vektoren
			\begin{align*}
				\begin{pmatrix} -1\\1\\0\\0 \end{pmatrix},
				\begin{pmatrix} -1\\0\\1\\0 \end{pmatrix},
				\begin{pmatrix} -1\\0\\0\\1 \end{pmatrix}\in\ker(A-2\id).
			\end{align*}
		\item Wir wenden nun das Gram-Schmidt-Verfahren an.
		\item Zun\ae chst normieren wir den ersten Vektor:
			\begin{align*}
				\begin{pmatrix} -1/\sqrt 2\\1/\sqrt 2\\0\\0 \end{pmatrix}
			\end{align*}
	\end{itemize}
	\end{block}
\end{frame}

\begin{frame}\frametitle{\mytitle}
	\begin{block}{Beispiel}
	\begin{itemize}
		\item Als n\ae chstes berechnen wir
			\begin{align*}
				\begin{pmatrix} -1\\0\\1\\0 \end{pmatrix}-\begin{pmatrix} -1&0&1&0 \end{pmatrix}\cdot \begin{pmatrix} -1/\sqrt 2\\1/\sqrt 2\\0\\0 \end{pmatrix}\cdot \begin{pmatrix} -1/\sqrt 2\\1/\sqrt 2\\0\\0 \end{pmatrix}&=\begin{pmatrix} -1/2\\-1/2\\1\\0
				\end{pmatrix}
			\end{align*}
		\item Diesen Vektor normieren wir:
			\begin{align*}
				\norm{\begin{pmatrix} -1/2\\-1/2\\1\\0\end{pmatrix}}^{-1} \begin{pmatrix} -1/2\\-1/2\\1\\0\end{pmatrix}=\begin{pmatrix}-1/\sqrt 6\\-1/\sqrt 6\\\sqrt{2/3}\\0\end{pmatrix}
			\end{align*}
	\end{itemize}
	\end{block}
\end{frame}

\begin{frame}\frametitle{\mytitle}
	\begin{block}{Beispiel}
	\begin{itemize}
		\item Wir kommen zum dritten Vektor:	
			\begin{align*}
			\begin{pmatrix} -1\\0\\0\\1 \end{pmatrix}
			&-\begin{pmatrix} -1&0&0&1 \end{pmatrix}\begin{pmatrix} -1/\sqrt 2\\1/\sqrt 2\\0\\0 \end{pmatrix}\begin{pmatrix} -1/\sqrt 2\\1/\sqrt 2\\0\\0 \end{pmatrix}\\
			&-\begin{pmatrix} -1&0&0&1 \end{pmatrix}\begin{pmatrix}-1/\sqrt 6\\-1/\sqrt 6\\\sqrt{2/3}\\0\end{pmatrix}\begin{pmatrix}-1/\sqrt 6\\-1/\sqrt 6\\\sqrt{2/3}\\0\end{pmatrix}\\
			&=\begin{pmatrix}-1/2\\-1/2\\0\\1 \end{pmatrix}+\begin{pmatrix}1/6\\1/6\\-1/3\\0\end{pmatrix}
			=\begin{pmatrix}-1/3\\-1/3\\-1/3\\1 \end{pmatrix}
			\end{align*}
	\end{itemize}
	\end{block}
\end{frame}

\begin{frame}\frametitle{\mytitle}
	\begin{block}{Beispiel}
	\begin{itemize}
		\item Diesen Vektor normieren wir noch:
			\begin{align*}
				\norm{\begin{pmatrix}-1/3\\-1/3\\-1/3\\1 \end{pmatrix}}^{-1}\begin{pmatrix}-1/3\\-1/3\\-1/3\\1 \end{pmatrix}&=\begin{pmatrix}
				-1/(2\sqrt 3)\\-1/(2\sqrt 3)\\-1/(2\sqrt 3)\\\sqrt 3/2
				\end{pmatrix}
			\end{align*}
	\end{itemize}
	\end{block}
\end{frame}

\begin{frame}\frametitle{\mytitle}
	\begin{block}{Beispiel}
	\begin{itemize}
		\item Wir erhalten die Orthonormalbasis
			\begin{align*}
				\begin{pmatrix} -\sqrt 2/2\\\sqrt 2/2\\0\\0 \end{pmatrix},\quad
\begin{pmatrix}-1/\sqrt 6\\-1/\sqrt 6\\\sqrt{2/3}\\0\end{pmatrix},\quad
\begin{pmatrix} -1/(2\sqrt 3)\\-1/(2\sqrt 3)\\-1/(2\sqrt 3)\\\sqrt 3/2 \end{pmatrix}
			\end{align*}
			des Eigenraums $\eig(A,2)$.
	\end{itemize}
	\end{block}
\end{frame}

\begin{frame}\frametitle{\mytitle}
	\begin{block}{Beispiel}
	\begin{itemize}
		\item Um die Matrix $U$ zu erhalten, f\ue gen wir schlu\ss endlich die Orthonormalbasen der Eigenr\ae ume zusammen:
			\begin{align*}
				U&=\begin{pmatrix}
					1/2&-1/\sqrt 2&-1/\sqrt 6&-1/(2\sqrt 3)\\
					1/2&1/\sqrt 2&-1/\sqrt 6&-1/(2\sqrt 3)\\
					1/2&0&\sqrt{2/3}&-1/(2\sqrt 3)\\
					1/2&0&0&\sqrt 3/2
				\end{pmatrix}
			\end{align*}
		\item Die Matrix $U$ ist orthogonal, d.h.\ $U^\trans=U^{-1}$.
	\end{itemize}
	\end{block}
\end{frame}

\begin{frame}\frametitle{\mytitle}
	\begin{block}{Beispiel}
	\begin{itemize}
		\item Wir haben somit die Matrix $A$ diagonalisiert:
			\begin{align*}
				A&=U\begin{pmatrix}
					-2&0&0&0\\0&2&0&0\\0&0&2&0\\0&0&0&2
				\end{pmatrix}
				U^\trans.
			\end{align*}
	\end{itemize}
	\end{block}
\end{frame}

\begin{frame}\frametitle{\mytitle}
	\begin{block}{Rechenregeln}
		Angenommen $A$ ist eine symmetrische $n\times n$-Matrix mit Diagonalisierung $A=U\cdot\diag(\lambda_1,\ldots,\lambda_n)U^\trans$.
	\begin{itemize}
		\item Es gilt $\det(A)=\prod_{i=1}^n\lambda_i$.
		\item F\ue r jede Zahl $\ell\in\NN$ gilt $ A^\ell=U\diag(\lambda_1^\ell,\ldots,\lambda_n^\ell)U^\trans.  $
		\item Falls $\det(A)\neq0$, gilt f\ue r jede Zahl $\ell\in\NN$
			\begin{align*}
				A^{-\ell}&=U\diag(\lambda_1^{-\ell},\ldots,\lambda_n^{-\ell})U^\trans.
			\end{align*}
		\item F\ue r jede Zahl $c\in\RR$ gilt
			\begin{align*}
				A+c\id_n&=U\diag(c+\lambda_1,\ldots,c+\lambda_n)U^\trans.
			\end{align*}
	\end{itemize}
	\end{block}
\end{frame}

\begin{frame}\frametitle{\mytitle}
	\begin{block}{Zusammenfassung}
	\begin{itemize}
		\item Durch Diagonalisierung kann man eine Orthonormalbasis finden, unter der eine symmetrische Matrix $A$ eine besonders einfache Beschreibung zul\ae sst.
		\item Das Diagonalisierungsverfahren beruht darauf, die Nullstellen des charakteristischen Polynoms zu finden.
	\end{itemize}	
	\end{block}
\end{frame}
\end{document}

